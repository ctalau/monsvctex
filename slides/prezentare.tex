\documentclass[]{prezentare}
\graphicspath{{Imagini/}}

\title  {Operational monitoring of ATLAS TDAQ system}

\author [Cristian T\u al\u au]
        {\texorpdfstring
            {{\bf Autor:} Cristian T\u al\u au \\ 
            {\bf Supervisors:} Prof. Roger Hersch, Dr. Wainer Vandelli}
            {Cristian Talau}  
      }
\institute {\texorpdfstring
                {Universitatea "Politehnica" Bucure\c sti}
                {Universitatea "Politehnica" Bucuresti}
           }
\titlegraphic {}

% Marcouri
\newcommand\nat{\textbf{Nat}}
\newcommand\tlist{\textbf{List}}
\newcommand\sfrec{\text{$System\ F^{rec}\ $}}
\newcommand\sfhat{\text{$System\ F$\^{}}}
\newcommand\sysf{\text{$System\ F$}}

\newcommand\vdrec{\vdash_\text{$SysF^{rec}\ $}}
\newcommand\vdhat{\vdash_\text{$SysF$\^{}}}
\newcommand\vdtm{\vdash_\text{$SF$\^{}mon}}

\newcommand\kwd[1]{\textcolor[rgb]{0.54,0.21,0.22}{\textbf{#1}}}
\newcommand\typ[1]{\textcolor[rgb]{0.00,0.59,0.00}{\textit{#1}}}

\newcommand\av[3]{\visible<#1>{\alert<#2>{\text{\tiny #3}}}}

\begin{document}

    \begin{frame}
        \titlepage
    \end{frame}

%-----------------------------------------------------------------------------------
    \begin{frame}
        \frametitle{Problema verific\u arii statice}
        \begin{block}<+->{Instan\c t\u a - Terminarea programelor}
            \begin{itemize}
            \item Solu\c tie corect\u a dar incomplet\u a
            \item \^ In cadrul analizei de tip
            \item \sfrec = \sysf + \textbf{case} + \textbf{letrec}
            \end{itemize}
        \end{block}

        \begin{exampleblock}<+-> {Caracteristici}
            Tipuri \alert<2>{polimorfice}, \alert<3>{recursive}, \alert<4>{de ordin \^ inalt}
            $$\alert<3>{Tree}\ \alert<2>{X} = lf : \alert<3>{Tree}\ \alert<2>{X} \,|\, nd : \alert<4>{(Nat \to \alert<3>{Tree}\ \alert<2>{X})} \to \alert<2>{X} \to \alert<3>{Tree}\ \alert<2>{X}$$
        \end{exampleblock}

    \end{frame}
%-----------------------------------------------------------------------------------
    \begin{frame}
        \frametitle{Ideea \sfhat}

        \begin{block}<1->{Constr\^ angere letrec}
        \begin{itemize}
        \item Apelurile recursive se fac cu argumente mai mici
        \item Rela\c tia de ordine $\Leftrightarrow$ Rela\c tia de subtip
        \end{itemize}

        \end{block}
        \frametitle{Tipuri inductive cu dimensiune}
        \begin{exampleblock}<2->{\visible<3->{\sfhat} \visible<2>{\sfrec} }
                \vskip -0.6cm
                \begin{align*}
                    \textbf{Datatype}\ Nat &= z : Nat^{\visible<3->{\alert{\widehat{\imath}}}} \,|\,
                                              s : Nat^{\visible<3->{\alert{\imath}}} \to Nat^{\visible<3->{\alert{\widehat{\imath}}}} \\
                    \textbf{Datatype}\ List &= nil : List^{\visible<3->{\alert{\widehat{\imath}}}} \,|\, cons : Nat^{\visible<3->{\alert{\infty}}} \to List^{\visible<3->{\alert{\imath}}} \to List^{\visible<3->{\alert{\widehat{\imath}}}}
                \end{align*}
        \end{exampleblock}
        \begin{exampleblock}<2->{Diagrama}
        \end{exampleblock}

    \end{frame}
%-----------------------------------------------------------------------------------
    \begin{frame}
        \frametitle{Expresivitate}
        \begin{block}<1->{Propriet\u ati - \sfhat}
            \begin{itemize}
                \item Compatibilitate \sfrec
                \item Normalizare puternic\u a \visible<2->{\alert<2>{$\Longrightarrow$ Incompletitudine Turing}}
            \end{itemize}
        \end{block}
        \begin{block}<3->{Tipuri monadice - \sfhat-mon}
            \begin{itemize}
                \item Dou\u a categorii de tipuri incompatibile ca $\sqsubseteq$: \textbf{ID} , \textbf{NT}
                \item \textbf{ID} - toate expresiile \sfhat
                \item \textbf{NT} - expresii \sfrec (slab/ne)-normalizabile 
            \end{itemize}
        \end{block}


    \end{frame}
%-----------------------------------------------------------------------------------
    \begin{frame}
        \frametitle{Cum construim expresii NT?}

        \begin{exampleblock}
        {\visible<1-3>{unit}
         {\visible<4>{letrec - $Nat_{nt}^{\imath} \sqsubseteq Nat_{nt}^{\jmath}$}
          \visible<5->{bind}
         }
        }
        
        \end{exampleblock}

        \begin{exampleblock}{Diagram\u a}
        \end{exampleblock}
    \end{frame}
%-----------------------------------------------------------------------------------
    \begin{frame}
        \frametitle{Tipuri monadice - propriet\u a\c ti}
        \begin{block}{Propriet\u ati}
        \begin{itemize}
        \item {\bf Siguran\c t\u a}:
            \begin{itemize}
                \item \textit{(Conservare)} Dac\u a $e:\tau \to e':\tau'$, atunci.
                    \begin{itemize}
                        \item $\tau \in \textbf{ID} \Rightarrow \tau = \tau'$
                        \item $\tau \in \textbf{NT} \Rightarrow |\tau| = |\tau'|$.
                    \end{itemize}
                \item \textit{(Progres)} Evaluarea unei expresii se termin\u a cu o valoare.
            \end{itemize}
        \item {\bf Normalizare puternic\u a}
            \begin{itemize}
                \item Pentru tipuri din \textbf{ID}.
            \end{itemize}
        \item {\bf Completitudine Turing}
            \begin{itemize}
                \item Orice func\c tie $\mu$-recursiv\u a poate fi tipat\u a \^ in \textbf{NT}
            \end{itemize}
        \item {\bf Compatibilitate} cu \sfrec
        \end{itemize}
        \end{block}
    \end{frame}
%-----------------------------------------------------------------------------------

    \begin{frame}
        \frametitle{TBT (Type Based Termination) }
        \begin{block}<+->{Sintax\u a $\approx$ ML}
                \vskip -0.6cm
            \begin{align*}
                \texttt{\kwd{Inductive} \typ{Bool} }& \texttt{= t : \typ{Bool} | f : \typ{Bool}} \\
                \lambda x : Nat . e                 &\mapsto                \texttt{\kwd{fn} \typ{<Nat>} x \kwd{=>} e} \\
                e \ e'                              &\mapsto               \texttt{e' e} \\
                (\lambda x : Nat . e )\ e'           &\mapsto               \texttt{\kwd{let} \typ{<Nat>} x \kwd{:=} e'; e} \\
                \Lambda X. e                        &\mapsto                \texttt{\typ{/{\char`\\} X.} e}\\
                e\ \forall X.X                      &\mapsto                \texttt{e \typ{<{\char`\\}/X.X>} }\\
                \text{letrec}_{Nat\to Nat}\ f = e    &\mapsto               \texttt{\kwd{letrec} \typ{<Nat->Nat>} f \kwd{:=} e}\\
                \text{case}_X\ e \text{ of } \{ c \Rightarrow e\} &\mapsto  \texttt{\kwd{case} \typ{<X>} e \kwd{of} \{c \kwd{=>} e\}} \\
                bind\ f\ e                             &\mapsto               \texttt{ \kwd{bind} f e} \\
                unit\ e                              &\mapsto                \texttt{ \kwd{unit} e} 
            \end{align*}
        \end{block}
    \end{frame}
    
    \begin{frame}
        \frametitle{Compilator}
        \begin{block}<+->{Generare de cod}
            \begin{itemize}
                \item \textit{TBT} $\mapsto$ \textit{Java}
                \item \^ Inchideri func\c tionale
                \item ANTLR + StringTemplate
            \end{itemize}
        \end{block}
        \begin{block}<+->{Typechecker}
            \begin{itemize}
            \item Variabile de inferen\c t\u a
            $$ s : Nat^{\av{3-}{3}{$v_1$}} \to Nat^{\av{3-}{3}{$\widehat{v_1}$}} ,
               z : Nat^{\av{3-}{3}{$\widehat{v_2}$}} \vdash 
               \lambda f : Nat^{\av{3-}{3}{$v_3$}} \to Nat^{\av{3-}{3}{$v_4$}} . f\ z : Nat^{\av{3-}{3}{$v_4$}}$$
            \item<4-> Constr\^ angeri \^ intre variabile $\Rightarrow$ clas\u a de tipuri
            $$ Nat^{\widehat{v_2}} \sqsubseteq Nat^{v_3} \Rightarrow \{\widehat{v_2} \le v_3 \}$$
            \end{itemize}
        \end{block}
    \end{frame}

%-----------------------------------------------------------------------------------
    \begin{frame}
    \setbeamercolor{qa}{fg=block title.fg,bg=block title.bg} %structure.fg

    \begin{beamercolorbox}[rounded=true,shadow=true]{qa}
    \begin{center}
        {\Huge \textbf{Mul\c tumesc!}}
    \end{center}
    \end{beamercolorbox}
    \vfill
    \begin{columns}
            \begin{column}[l]{0.5\textwidth}
                \begin{itemize}
                    \item Normalizare puternic\u a
                    \item Calcul Lambda
                    \item \^ Inchideri func\c tionale
                    \item ANTLR + StringTemplate
                \end{itemize}
            \end{column}
            \begin{column}[l]{0.5\textwidth}
                \begin{itemize}
                    \item (In)completitudine Turing
                    \item Sisteme de tipuri
                    \item Variabile de inferen\c t\u a
                    \item Constr\^ angeri
                \end{itemize}
            \end{column}
    \end{columns}
    \end{frame}

\end{document}
