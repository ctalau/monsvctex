% Capitolul 3 - 7 pag

\chapter{System F\^{}}
\label{Capitolul3}

Se observ\u a c\u a pentru a rec\u apata proprietatea de normalizare puternic\u a, trebuie impuse unele constr\^ angeri asupra System \frec.
\section{Scheme de recursivitate}
\label{scheme_rec}
\^ Inainte de a stabili ce constr\^ angeri trebuie impuse, vom prezenta unele scheme de recursivitate des folosite \citep{varmo_vene}. Acestea trebuie s\u a poat\u a fi exprimate \^ in limbaj \^ intr-un mod natural \c si cu o complexitate mic\u a.

\subsection{Recursivitate structural\u a}
Acest tip de recursivitate este folosit\u a \^ in general pe structuri de date cu un \emph{tip de date algebric} \c si se aplic\u a pentru cazul \^ in care valoarea func\c tiei pentru un termen depinde doar de valorile func\c tiei pe subtermenii direc\c ti ai acestui termen.
\begin{definition}
Un tip de date recursiv se nume\c ste \textbf{tip de date algebric} dac\u a fiecare constructor are tipul $ c_i : \Pi X . T_1\app X_1 \to \dots \to T_n\app X_n\to T X $, unde $T_k$ sunt identificatori de tipuri de date, si $X_k \subset X$.
\end{definition}
Aceste tipuri se mai numesc \emph{sume de produse}. Aceast\u a denumire provine din faptul c\u a un termen cu constructorul principal $c : T_1\to \dots T_n\to T $ poate fi reprezentat de tuplul argumentelor c\u arora acesta le este aplicat $(t_1,\dots,t_n)\in P = T_1 \times \dots \times T_n $, deci de un element din produsul cartezian al $T_1, \dots, T_n$. Cum tipul de date are mai mul\c ti constructori, elementele apartin $P_1 \cup \dots \cup P_m $ cu $P_i$ produse carteziene. \^ In teoria categoriilor reuniunea a doua mul\c timi se cheam\u a \emph{coprodus} sau \emph{sum\u a}, iar produsul cartezian \emph{produs}.
\begin{example}
Un exemplu de astfel de func\c ie recursiv\u a este functia
\begin{align*}
double(succ(x)) &= 2 + double(x)\\
      double(0) &= 0
\end{align*}
\end{example}

\subsection{Recursivitate primitiv\u a}
Recursivitatea primitiv\u a este folosit\u a atunci c\^ and valoarea func\c tiei pentru un termen depinde de valorile func\c tiei pentru subtermenii direc\c ti ai termenului \c si de ace\c sti subtermeni. Se observ\u a c\u a recursivitatea primitiv\u a este mai expresiva decat cea structural\u a.

\begin{example}
Un exemplu de astfel de func\c tie recursiv\u a este func\c tia
\begin{align*}
fact(succ(x)) &= (x+1) \cdot fact(x)\\
      fact(0) &= 1
\end{align*}
\end{example}

\subsection{Recursivitate couse-of-value}
\done\todo{cum sa zic in romana course-of-value}
Recursivitatea {course-of-value} difer\u a de cea primitiv\u a prin faptul c\u a valoarea func\c tiei pentru un termen poate depinde \c si de valoare func\c tiei pentru subtermeni ai acestuia care nu sunt direc\c ti, dar ad\^ ancimea la care apar este finit\u a.

\begin{example}
Un exemplu de astfel de func\c tie recursiv\u a este func\c tia
\begin{align*}
fibo(succ(succ(x))) &= fibo(succ(x)) + fibo(x)\\
      fibo(1) &= 1 \\
      fibo(0) &= 1
\end{align*}
\end{example}

Se observ\u a c\u a printr-un procedeu de memoizare, aceast\u a schem\u a de recursivitate este echivalent\u a cu recursivitatea primitiv\u a.
\begin{example}
Func\c tia din exemplul anterior se poate rescrie ca
\begin{align*}
fibo(n) &= x \text{ unde } (x,y) = fibo'(n)\\
fibo'(succ(x)) &= (x+y, x) \text{ unde } (x,y) = fibo(x)\\
      fibo'(0) &= (1,0)
\end{align*}
\end{example}
Totu\c si, pentru a codifica aceast\u a schem\u a de recursivitate cu func\c tii primitiv recursive, trebuie folosit\u a o tehnic\u a  artificial\u a.
\section{Renun\c tarea la recursivitatea textuala}

Din demonstra\c tia teoremei \ref{compl_sysfrec} se poate observa c\u a pentru introducerea posibilit\u a\c tii de exprimare a unor func\c tii par\c tiale prin operatorul de minimizare s-a folosit recursivitatea textual\u a. O idee natural\u a pentru redob\^ andirea normaliz\u arii puternice este eliminarea acestei construc\c tii din limbaj. Totu\c si urmatoarea propozi\c tie infirm\u a proprietatea normaliz\u arii puternice pentru System \frec f\u ar\u a recursivitate textual\u a.
\begin{proposition}\label{case_nonterm}
Se pot construi \^ in System \frec expresii care nu se reduc la o form\u a normal\u a f\u ara\u  a folosi recursivitatea textual\u a.
\end{proposition}
\begin{proof}[Demonstra\c tie]
Vom porni de la urmatoarea lem\u a.
\begin{lemma}\label{inconsistence_to_nonterm}
Dac\u a exist\u a expresii cu tipul $\bot := \Pi X. X$, atunci System \frec nu are proprietatea de normalizare puternic\u a.
\end{lemma}
\done\todo{demonstratie lema}
Vom \^ incerca apoi s\u a construim un termen de tipul $\bot$. Pentru a demonstra aceast\u a propozi\c tie ne vom baza pe leg\u atura dintre logic\u a \c si teoria tipurilor prin \emph{izomorfisuml Curry-Howard} \citep{130367}. Vom nota prin $T$ propozi\c tia $\exists x . x : T$. Cu aceasta nota\c tie, obiectivul nostru devine acela de a demonstra propozi\c tia $\bot$. Consider\u am tipul de date
$$ \textbf{{Datatype}  } D := c : (D \to \bot) \to D $$
Regula de tip pentru analiza de cazuri pe $D$ corespunde urmatoarei reguli de deduc\c tie :
\begin{prooftree}
        \AxiomC{$D$}
        \AxiomC{$(D \Rightarrow \bot) \Rightarrow X $}
        \LeftLabel{$\forall X$}
        \BinaryInfC{$ X $}
\end{prooftree}
Care se traduce prin urmatoarea propozi\c tie
\begin{equation} \label{case_prop}
\forall X. D \wedge ((D \Rightarrow \bot) \Rightarrow X) \Rightarrow X
\end{equation}
Atunci urmatorul arbore de deduc\c tie demonstreaz\u a propozi\c tia $\bot$.
\begin{prooftree}
\AxiomC{$c : (D \to \bot) \to D$}
\UnaryInfC{$ (D \Rightarrow \bot) \Rightarrow D $ }
        \AxiomC{\eqref{case_prop}}
        \AxiomC{$(D \Rightarrow \bot) \Rightarrow (D \Rightarrow \bot)$}
        \BinaryInfC{$D \Rightarrow (D \Rightarrow \bot)$}
        \UnaryInfC{$D \Rightarrow \bot$}
    \BinaryInfC{$D$}
                    \AxiomC{$\vdots $}
                    \UnaryInfC{$D \Rightarrow \bot$}
        \BinaryInfC{$\bot$}
\end{prooftree}
Am folosit urmatoarea lem\u a clasic\u a ce apare \^ in contextul izomorfismului Curry-Howard.
\begin{lemma}\label{curry_howard_impl}
Propozi\c tia $ A \to B $ implic\u a propozi\c tia $A \Rightarrow B$. \qedhere
\end{lemma}
\end{proof}
In continuare vom da \c si demonstra\c tiile celor dou\u a leme.
\begin{proof}[Demonstra\c tia lemei \ref{inconsistence_to_nonterm}]
Presupunem c\u a System \frec are proprietatea de normalizare puternic\u a \c si c\u a exist\u a o expresie $e : \bot $. Cum $e$ este normalizabil\u a de tipul $\Pi X. X$ inseamn\u a c\u a $\exists e'. e \downarrow \Lambda X.e'$. Prin $e \downarrow v$ am notat faptul c\u a $e$ se reduce la forma normal\u a $v$.

Consider\u am acum expresia $e \app [ \textbf{{Bool}} ]$. Aceasta are tipul \textbf{{Bool}} \c si deci se reduce la o form\u a normal\u a care poate fi \textbf{{true}} sau \textbf{{false}}. Presupunem f\u ar\u a a restr\^ ange generalitatea c\u a aceasta este \textbf{{false}}. Deci
\begin{equation*}
(\Lambda X.e')\app [\textbf{{Bool}}] \downarrow \textbf{{false}} \Rightarrow [X \mapsto\textbf{{Bool}}] e' \downarrow \textbf{{false}}
\end{equation*}

Fie $e''$ astfel incat $e'\downarrow e''$ . Avem
\begin{equation*}
[X \mapsto\textbf{{Bool}}] e' \downarrow [X \mapsto\textbf{{Bool}}] e'' \Rightarrow
[X \mapsto\textbf{{Bool}}] e'' = \textbf{{false}}  \Rightarrow
e'' = \textbf{false}
\end{equation*}

Din proprietatea de conservare, ob\c tinem urm\u atoarele concluzii cu privire la tipul lui $e$.

\begin{equation*}
    e'' : \textbf{{Bool}} \Rightarrow
    e'  : \textbf{{Bool}} \Rightarrow
    e = \Lambda X.e' : \Pi X. \textbf{{Bool}} \neq \Pi X.X = \bot \qedhere
\end{equation*}
\end{proof}

\begin{proof}[Demonstratia lemei \ref{curry_howard_impl}]
Fie $e_{A \to B}$ astfel \^ inc\^ at $e_{A \to B} : A \to B$. Avem urmatoarele propozi\c tii
\[\exists e_A. e_A : A \Rightarrow \exists e_B . e_B = e_{A \to B}\app e_A \Rightarrow \exists e_B : B \qedhere\]
\end{proof}
Se poate da \c si o demonstra\c tie mai artificial\u a, dar constuctiv\u a pentru propozi\c tia \ref{case_prop} (\citep{1614481}).
\begin{proof}[Demonstra\c tia propozi\c tiei \ref{case_prop}]
\begin{align*}
p &:= \lambda x : D. \text{{case}}_{D\to \bot}\ x \text{ {of} } \{ c \Rightarrow \lambda y : D \to \bot . y \} &{ (D \Rightarrow (D \Rightarrow \bot))} \\
\omega_D &:= \lambda x : D. p\app x\app x &{ (D \Rightarrow \bot)}\\
\omega &:= p\app  (c\app \omega_D) \app (c\app \omega_D) &{ (\bot)}
\end{align*}
Expresia $\omega$ este cea cautat\u a. O secven\c t\u a infinit\u a de reduceri este urmatoarea
\[ \omega \to p\app (c\app \omega_D)\app (c\app \omega_D) \to (\lambda y : D \to \bot . y\app  \omega_D)\app (c\app \omega_D) \to \omega_D (c\app \omega_D) \to p\app (c\app \omega_D)\app (c\app \omega_D)\to \dots \qedhere \]
\end{proof}

Pe baza propozi\c tiei \ref{case_nonterm} putem trage concluzia c\u a trebuie restric\c tionate ambele construc\c tii introduse: \c si analiza de cazuri \c si recursivitatea textual\u a.

\section{Recursori}

O alternativ\u a ar fi creearea unor operatori de ordin \^ inalt care s\u a \^ inlocuiasca recursivitatea textual\u a \c si s\u a reprezinte cele trei tipuri de recursivitate indicate la sec\c tiunea \ref{scheme_rec}.

\begin{example}
Pentru a exprima recursivitatea structural\u a pe arbori se poate defini o func\c tie similar\u a func\c tiei \emph{\textbf{fold}} pe liste cu urmatorul comportament
\begin{align*}
\textbf{\emph{Datatype}  } \emph{\textbf{Tree}} &:= leaf : \emph{\textbf{Tree}} \ |\  node : \emph{\textbf{Tree}} \to \emph{\textbf{Nat}} \to \emph{\textbf{Tree}} \to \emph{\textbf{Tree}} \\
foldTree &:= \Lambda X .\lambda l:X. \lambda n : (X\to \emph{\textbf{Nat}} \to X \to X). \\
 &\text{\emph{letrec}}_{\emph{\textbf{Tree}} \to X}\ f =  \lambda t'. \text{\emph{case}}_X \ t' \text{ \emph{of} }\\
 &\qquad leaf \Rightarrow l\\
 &\qquad node \Rightarrow \lambda t_1 : \emph{\textbf{Tree}}.\lambda nr :\emph{\textbf{Nat}}. \lambda t_2 : \emph{\textbf{Tree}} . n\app (f\app t_1)\app nr\app (f\app t_2)
\end{align*}
\end{example}

Problema cu acest\u a alternativ\u a este c\u a folosirea recursorilor nu este deloc intuitiv\u a pentru programatori.
\begin{example}
Pentru a calcula dimensiunea unui arbore se foloseste urmatoarea expresie
$$foldTree \app [\emph{\textbf{Nat}}]\app z \app  (\lambda l : \emph{\textbf{Nat}}.\lambda  n : \emph{\textbf{Nat}}.\lambda r :\emph{\textbf{Nat}} . (add\app l\app r\app (s\app z))) $$
spre deosebire de varianta cu recursivitate textual\u a
\begin{align*}
\text{\emph{letrec}}_{\emph{\textbf{Tree}} \to \emph{\textbf{Nat}}} sz &= \lambda t  :\emph{\textbf{Tree}}.  \text{case}_{\emph{\textbf{Nat}}}\ t \text{ of } \\
&\qquad leaf \Rightarrow z \\
&\qquad node \Rightarrow \lambda l :\emph{\textbf{Tree}} .\lambda n: \emph{\textbf{Nat}}.\lambda r : \emph{\textbf{Tree}}. (add\app (sz\app l) \app (sz \app r)\app (s\app z))
\end{align*}
\end{example}

\begin{remark}
Aceast\u a construc\c tie poate fi generalizat\u a doar pentru o parte din tipurile de date recursive care se pot defini \^ in System \frec. De exemplu, pentru un tip cu un constructor de forma $ c : (T \to \emph{\textbf{Nat}}) \to T$, construc\c tia nu se extinde \^ in mod natural.
\end{remark}

\begin{remark}
O mul\c time de tipuri pentru care extinderea func\c tioneaz\u a sunt tipurile de date \emph{algebrice}.
\end{remark}

\section{Tipuri cu dimensiune}

Ideea din spatele System \fhat \citep{1614481} este de a extinde Sysyem \frec si de a folosi sistemul de tipuri pentru a exprima constr\^ angerile asupra analizei de cazuri \c si a recursivit\u a\c tii textuale.

Pentru a restric\c tiona recursivitatea textual\u a, se folose\c ste reprezentarea tipurilor de date din System \frec ca limita unui \c sir de subtipuri (numite \c si aproximari) astfel \^ incat, \^ in cazul schemelor clasice de recursivitate, apelurile recursive sunt f\u acute asupra unor argumente al c\u aror tip \emph{des\-cres\-te} (\^ in raport cu relatia de subtip). \^ In urma restric\c tiei puse pentru a rezolva problema analizei de cazuri, rela\c tia de subtip devine \fixme{well-founded} (defini\c tia \ref{well_founded}), lucru care asigura terminarea apelurilor recursive.

\begin{example}
Daca pornim de la interpretarea tipului de numere naturale $\textbf{\emph{Nat}} \equiv \{ z, s(z), \dots \}$ atunci interpretarile subtipurilor acestuia sunt
$$\textbf{\emph{Nat}}^p \equiv \{ z, s(z), \dots , s^p(z)\} \subset \textbf{\emph{Nat}}^{p+1} \subset \dots \subset \textbf{\emph{Nat}}^\infty \equiv \textbf{\emph{Nat}} $$
\end{example}

Vom reprezenta aceste aproximari prin \emph{tipuri cu dimensiune} cu urmatoarea sintax\u a:
\begin{gather*}
s := V_s \ |\ \hat{s}\ |\ \infty\\
\overline{T} := X \ |\ \overline{T} \to \overline{T}\ |\ \Pi X. \overline{T}\ |\ d^s\app \overline{T}
\end{gather*}

\begin{example}
Dimensiuni: $\infty$, $\hat{\iota}$. Tipuri cu dimensiune: $\textbf{\emph{Nat}}^\infty$, $\textbf{\emph{Nat}}^{\hat{\iota}}$. Din punct de vedere intuitiv, dac\u a $\textbf{\emph{Nat}}^{\iota}$ are interpretarea $\textbf{\emph{Nat}}^{p}$ atunci $\textbf{\emph{Nat}}^{\hat{\iota}}$ are interpretarea $\textbf{\emph{Nat}}^{p+1}$.
\end{example}
Problema analizei de cazuri este rezolvat\u a prin introducerea \emph{tipurilor inductive cu dimensiune}.
\begin{definition}
Un tip de date se nume\c ste \textbf{tip inductiv cu dimensiune} dac\u a to\c ti constructorii au tipul $c_k : \Pi X. \overline{\theta}_k \to d^{\hat{\iota}} X$, unde
\begin{itemize}\addtolength{\itemsep}{-0.5\baselineskip}
  \item Constructorul de tip de date $d$ apare in $\overline{\theta}_k$ doar pe pozi\c tii pozitive. O pozi\c tie este pozitiv\u a, dac\u a \^ in calea c\u atre r\u adacin\u a, \^ in arborele sintactic al tipului, se afl\u a \^ in st\^ anga constructorului $\to$ de numar par de ori.
  \item Orice apari\c tie a lui $d$ in $\overline{\theta}_k$ are dimensiunea $\iota$.
  \item Orice apari\c tie a altui constructor de tip de date $d' \neq d$ are dimensiuneea $\infty$.
\end{itemize}
\end{definition}\done\todo{poza cu diagrama adj}
Func\c tia $|.|:\overline{T} \to T$ este func\c tia care \c sterge adnot\u arile de dimensiune, de exemplu $|d^\iota X| = d X $.
Sintaxa termenilor si regulile de evaluare sunt identice cu cele ale System \frec.
\begin{example}
\^ In urmatoarea diagram\u a ADJ \citep{adj} este prezentat\u a structura tipului de liste de numere naturale. Discurile reprezint\u a tipuri, cu incluziunea de discuri reprezent\^ and rela\c tia de subtip. S\u agetile albastre reprezint\u a constructori, sursele s\u age\c tilor reprezint\u a tipuri din aritate, iar destina\c tiile reprezinta tipul rezultat. Fiecare constructor apare de mai multe ori pentru fiecare valoare a dimensiunii (polimorfism de dimensiune).
\begin{figure}
\begin{center}
\begin{tikzpicture}[node distance=4cm, auto,>=latex', thick]
    \tikzstyle{type} = [shape=circle, fill=white, draw, thin]
    \tikzstyle{constructor} = [draw, ->,color=blue!50!black ]

% Nat
    \node [type,minimum size=3.6cm,pin={[pin edge={<-,red}]90:$\textbf{\emph{Nat}}^{\infty}$}] at (0cm,0cm) (nat) {};
    \node [type,minimum size=2.4cm,label=above:$\vdots$,pin={[pin edge={<-,red}]50:$\textbf{\emph{Nat}}^{{\hat{\hat{\iota}}}}$}] at (0cm,0cm) (natssp) {};
    \node [type,minimum size=2.0cm,pin={[pin edge={<-,red}]30:$\textbf{\emph{Nat}}^{{{\hat{\iota}}}}$}] at (0cm,0cm) (natsp) {};
    \node [type,minimum size=1.6cm,pin={[pin edge={<-,red}]10:$\textbf{\emph{Nat}}^{{{{\iota}}}}$}] at (0cm,0cm) (natp) {};

    % succesor
    \path [constructor] (natp.south east)  node[above] {\scriptsize s} parabola[bend pos=0.5] bend +(0,-1cm) (natsp.south west) ;
    \path [constructor] (natsp.south east) node[above] {\scriptsize s} parabola[bend pos=0.5] bend +(0,-1cm) (natssp.south west);
    \path [constructor] {(nat.south)+ (0.5cm,0.08cm)}   node[above] {\scriptsize s} parabola[bend pos=0.5] bend +(0,-1cm) (nat.south);

    % zero
    \path [constructor] (0,0) circle (0) node[pin={[pin edge={<-,blue!50!black},pin distance=2cm]180:{\scriptsize z}}] (centru) {};

% Liste
    \tikzstyle{cons} = [shape=rectangle, thin, minimum width=0.03cm, minimum height=0.03cm]

    \node [cons,right of=nat,yshift=-1cm,label=below:{\scriptsize cons}] (consp) {};
    \node [cons,right of=nat,label=below:{\scriptsize cons}] (conssp) {};
    \node [cons,right of=nat,yshift=+1cm,label=below:{\scriptsize cons}] (consssp) {};

    \tikzstyle{type} = [shape=circle, fill=white, draw, thin, xshift=8cm]

    \node [type,minimum size=3.6cm,pin={[pin edge={<-,red}]90:$\textbf{\emph{List}}^{\infty}$}] at (0cm,0cm) (list) {};
    \node [type,minimum size=2.4cm,label=above:$\vdots$,pin={[pin edge={<-,red}]50:$\textbf{\emph{List}}^{{\hat{\hat{\iota}}}}$}] at (0cm,0cm) (listssp) {};
    \node [type,minimum size=2.0cm,pin={[pin edge={<-,red}]30:$\textbf{\emph{List}}^{{{\hat{\iota}}}}$}] at (0cm,0cm) (listsp) {};
    \node [type,minimum size=1.6cm,pin={[pin edge={<-,red}]10:$\textbf{\emph{List}}^{{{{\iota}}}}$}] at (0cm,0cm) (listp) {};

    % cons
    \path [constructor,-] (nat.east) edge[bend right]  (consp.center);
    \path [constructor,-] (nat.east) edge             (conssp.center);
    \path [constructor,-] (nat.east) edge[bend left] (consssp.center);

    \path [constructor] (consp.center)   parabola[bend pos=0.5] bend +(0,+0.5cm) (listsp.south west) ;
    \path [constructor] (conssp.center)  parabola[bend pos=0.5] bend +(0,+0.2cm) (listssp.west);
    \path [constructor] (consssp.center) parabola[bend pos=0.5] bend +(0,-0.5cm) (list.north west);

    \path [constructor,-] (listp.south west)     parabola[bend pos=0.5] bend +(0,-0.5cm) (consp.center) ;
    \path [constructor,-] (listsp.west)     parabola[bend pos=0.5] bend +(0,-0.2cm) (conssp.center);
    \path [constructor,-] (list.north west) parabola[bend pos=0.5] bend +(0,+0.5cm) (consssp.center);

    % nil
    \path [constructor] (list.center) circle (0) node[pin={[pin edge={<-,blue!50!black},pin distance=2cm]0:{\scriptsize nil}}] (centru) {};

\end{tikzpicture}

\end{center}
\caption{Diagrama ADJ pentru List \c si Nat}
\label{adj_list_nat}
\end{figure}
\end{example}


\subsection{Rela\c tia de subtip}
Rela\c tia de subtip \^ intre tipurile cu dimensiune este derivat\u a din rela\c tia de ordine \^ intre dimensiuni
\begin{align*}
s &\le s &  s \le s'  \wedge s' &\le s''  \Rightarrow s \le s''\\
s &\le \hat{s} & s &\le \infty
\end{align*}
Constructorul de tip $\to$ este contravariant \^ in primul argument \c si covariant \^ in al doilea. Intuitiv, o func\c tie care poate fi apelat\u a cu argumente de tip $T$, poate fi apelat\u a \c si cu argumente de tip $T' \sqsubseteq T$ \^ in timp ce rezultatul unei func\c tii $T_r$ poate fi considerat ca fiind de tip $T_r' \sqsupseteq T_r$. Regula pentru constructorii de tip defini\c ti de utilizator combin\u a regula de mo\c stenire a rela\c tiei de subtip din rela\c tia de ordine \^ intre dimensiuni cu regula de covarian\c t\u a a constructorilor.
\begin{multicols}{2}
\setlength\columnseprule{.4pt}
\begin{prooftree}
\AxiomC{$\overline{\tau}'\sqsubseteq \overline{\tau}$}
\AxiomC{$\overline{\sigma}\sqsubseteq \overline{\sigma}'$}
\BinaryInfC{$ \overline{\tau} \to\overline{\sigma} \sqsubseteq \overline{\tau}' \to\overline{\sigma}'$}
\end{prooftree}
\begin{prooftree}
\AxiomC{}
\UnaryInfC{$X \sqsubseteq X $}
\end{prooftree}
\begin{prooftree}
\AxiomC{$s \le s'$}
\AxiomC{$\overline{\tau}\sqsubseteq \overline{\sigma}$}
\BinaryInfC{$ d^s \overline{\tau} \sqsubseteq d^{s'} \overline{\sigma}$}
\end{prooftree}
\begin{prooftree}
\AxiomC{$\overline{\tau}\sqsubseteq \overline{\sigma}$}
\UnaryInfC{$ \Pi X.\overline{\tau} \sqsubseteq \Pi X.\overline{\sigma}$}
\end{prooftree}
\end{multicols}

\subsection{Reguli de tip}
\label{reguli_sysfhat}
Chiar dac\u a la nivel sintactic, adnot\u arile cu tipuri folosesc tipuri din System \frec, tipurile deduse automat \^ in System \fhat sunt tipuri cu dimensiune.
\begin{multicols}{2}
\setlength\columnseprule{.4pt}

\begin{prooftree}
\AxiomC{$x : \overline{\sigma} \in \overline{\Gamma} $}
\RightLabel{\scriptsize (T-VAR)}
\UnaryInfC{$\overline{\Gamma} \vdash x : \overline{\sigma} $}
\end{prooftree}

\begin{prooftree}
\AxiomC{$\overline{\Gamma}, x : \overline{\tau} \vdash e : \overline{\sigma} $}
\RightLabel{\scriptsize (T-ABS)}
\UnaryInfC{$\overline{\Gamma} \vdash \lambda x : |\overline{\tau}|. e : \overline{\tau} \to \overline{\sigma} $}
\end{prooftree}

\begin{prooftree}
\AxiomC{$\overline{\Gamma} \vdash e : \overline{\sigma} $}
\RightLabel{\scriptsize (T-TABS)}
\UnaryInfC{$\overline{\Gamma} \vdash \Lambda X . e : \Pi X. \overline{\sigma} $}
\end{prooftree}

\begin{prooftree}
\AxiomC{$c_k \in \mathcal{C}(d)$}
\RightLabel{\scriptsize (T-CONS)}
\UnaryInfC{$\overline{\Gamma} \vdash c_k : \Pi X. \overline{\theta}_k\to d^{\hat{\iota}} X $}
\end{prooftree}

\columnbreak

\begin{prooftree}
\AxiomC{$\overline{\Gamma} \vdash e : \overline{\sigma}$}
\AxiomC{$\overline{\sigma} \sqsubseteq \overline{\tau} $}
\RightLabel{\scriptsize (T-SUB)}
\BinaryInfC{$\overline{\Gamma} \vdash e : \overline{\tau} $}
\end{prooftree}

\begin{prooftree}
\AxiomC{$\overline{\Gamma} \vdash e : \overline{\tau} \to \overline{\sigma}$}
\AxiomC{$\overline{\Gamma} \vdash e' : \overline{\tau} $}
\RightLabel{\scriptsize (T-APP)}
\BinaryInfC{$\overline{\Gamma} \vdash e\app e' : \overline{\sigma} $}
\end{prooftree}

\begin{prooftree}
\AxiomC{$\overline{\Gamma} \vdash e : \Pi X.\overline{\sigma} $}
\RightLabel{\scriptsize (T-TAPP)}
\UnaryInfC{$\overline{\Gamma} \vdash e \app [|\overline{\tau}|] : [ X \mapsto \overline{\tau}] \overline{\sigma} $}
\end{prooftree}

\begin{prooftree}
\AxiomC{\scriptsize (T-LETREC)}
\noLine
\UnaryInfC{$\overline{\Gamma}, f : d^\iota\tau \to \overline{\theta} \vdash e : d^{\hat{\iota}}\tau \to [\iota \mapsto \hat{\iota}] \overline{ \theta}$}
\AxiomC{$\iota \text{\scriptsize pos} \overline{\theta} $}
\insertBetweenHyps{\hskip 2pt}
\BinaryInfC{$\overline{\Gamma} \vdash (\text{letrec}_{d|\overline{\tau}| \to |\overline{\theta}|} f = e) : d^s \overline{\tau} \to [\iota \mapsto s]\overline{\theta}$}
\end{prooftree}

\end{multicols}
\begin{prooftree}

\AxiomC{$\overline{\Gamma} \vdash e : d^{\hat{s}} \overline{\tau}$}
\AxiomC{$\overline{\Gamma} \vdash e_k : [X \mapsto \overline{\tau}, \iota \mapsto s]\overline{\theta}_k \to \overline{\sigma} $}
\AxiomC{$\overline{\Gamma} \vdash c_k : \Pi X. \overline{\theta}_k \to d^{\hat{\iota}} X$}
\RightLabel{\scriptsize (T-CASE)}
\TrinaryInfC{$\overline{\Gamma} \vdash \text{case}_{|\overline{\sigma}|}\ e \text{ of } \{c_1 \Rightarrow e_1\ |\ \dots \ |\ c_n \Rightarrow e_n\} : \overline{\sigma} $}
\end{prooftree}

\begin{remark}
Regula \textbf{ {\scriptsize (T-CONS)}} spune c\u a aplicarea unui constructor unor valori aflate \^ intr-o aproximatie a tipului de date, d\u a ca rezultat o valoare din aproximarea urmatoare.
\end{remark}

\begin{remark}
Regula \textbf{{\scriptsize (T-CASE)}} spune contrariul, \c si anume c\u a orice valoare care se afl\u a \^ intr-o aproximatie de dimensiune $\hat{s}$ poate fi supus\u a analizei de cazuri pentru a ob\c tine valori din aproxima\c tia $s$.
\end{remark}

\begin{remark}
Regula \textbf{{\scriptsize (T-LETREC)}} spune c\u a dac\u a o func\c tie definit\u a pe o aproxima\c tie poate fi extins\u a pe urmatoarea aproxima\c tie, atunci ea poate fi extins\u a pe intreg tipul de date, ca limit\u a a defini\c tiilor sale pe aproxim\u ari. Variabila $\iota$ trebuie s\u a nu apar\u a \^ in $\overline{\Gamma},\overline{\tau}$.
\end{remark}

\begin{remark}
\^ In procesul de deduc\c tie a tipului pentru o expresie, regula care trebuie aplicat\u a este unic determina\u ta de structura sintactic\u a a expresiei. Totu\c si procesul de stabilire a tipului bazat pe regulile de mai sus are o component\u a de nedeterminism deoarece trebuie s\u a \emph{ghiceasca} dimensiunile tipurilor. Astfel, spre deosebire de regulile System \frec, aceste reguli au doar rol teoretic; \^ in implementare se va folosi alt set de reguli echivalent.
\end{remark}

\subsection{Calita\c ti ca limbaj de programare}

Din punct de vedere al utiliz\u arii practice, System \fhat are aceeasi sintax\u a cu cea a System \frec ca\-re la r\^ andul s\u au se apropie de cea a unor limbaje de programare func\c tionala ca Haskell sau ML. O diferen\c t\u a major\u a fa\c t\u a de acastea este faptul c\u a programatorul trebuie s\u a adnoteze programele cu informa\c tii de tip. Totu\c si dimensiunile tipului se deduc automat.

Un lucru important \^ in cazul programelor despre care compilatorul nu poate decide dac\u a se termin\u a, este c\u a programatorul s\u a aiba \emph{apriori} intui\c tia dac\u a acestea vor trece de verificarea de tip \c si s\u a in\c teleag\u a \emph{aposteriori} din ce cauz\u a nu s-a putut demonstra terminarea programului. \^ In cazul System \fhat, ideea de demonstra\c tie este simpl\u a: dimensiunea argumentelor unei func\c tii textual recursive trebuie s\u a scad\u a. Mai mult, pentru tipuri de date comune precum liste sau arbori, dimensiunea tipului are o semnifica\c tie foarte natural\u a : lungimea listei sau ad\^ ancimea arborelui.

\done\todo{calitati ca limbaj de prog.}

