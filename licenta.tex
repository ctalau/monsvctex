%% ----------------------------------------------------------------
%% licenta.tex Fisierul principal
%% ----------------------------------------------------------------

\documentclass[a4paper, 11pt, oneside]{licenta}
\graphicspath{{Imagini/}}  % Folderul pentru imagini

% Pachete necesare
\usepackage[square, numbers, comma, sort&compress]{natbib}  % Folosim "Natbib" pentru bibliografie

% Macrouri
\newcommand\fhat{$F\hspace{4pt}\widehat{}\hspace{4pt}$}
\newcommand\frec{$F^{\emph{rec}}\hspace{2pt}$}
\newcommand\app{\hspace{2pt}}
%\newcommand\fixme[1]{ {\large \color[rgb]{1.0,0.0,0.0} \textbf{#1}} }
\newcommand\fixme[1]{#1}

%% ----------------------------------------------------------------
\begin{document}

\frontmatter	  % Numerotare cu cifre romane( i, ii, iii, iv...)
% Titlul lucrarii
\title  {Operational monitoring of ATLAS TDAQ evolution}
\fstsupervisor  {\texorpdfstring
            {\href{rd.hersch@epfl.ch} {Prof. R.D. Hersch}}
            {Prof. R.D. Hersch}}
            
\sndsupervisor  {\texorpdfstring
            {\href{wainer.vandelli@cern.ch} {Wainer Vandelli}} 
            {Wainer Vandelli}}

\authors  {\texorpdfstring
            {\href{ctalau@cern.ch}{Cristian T\u al\u au}}
            {Cristian T\u al\u au }
          }
\UNIVERSITY  {\texorpdfstring{\href{http://epfl.ch}
                {\' Ecole Polytechnique Fédérale de Lausanne - EPFL}}
                {\' Ecole Polytechnique Fédérale de Lausanne - EPFL}}
\faculty     {\texorpdfstring{\href{http://ic.epfl.ch}
                {School of Computer and Communication Sciences}}
                {School of Computer and Communication Sciences}}
% \addresses  {\groupname\\\deptname\\\univname}
\date       {\today}
\degree     {Master Thesis}
\subject    {}
\keywords   {}

\maketitle


%% ----------------------------------------------------------------

\pagestyle{fancy}  % Folosim titlul sectiunii in antetul paginii
\tableofcontents

%% ----------------------------------------------------------------
\mainmatter	  % Numerotare normala (1,2,3...)

% Capitolele lucrarii
% Capitoulul 1 - 6 pag

\chapter{Motiva\c tie} % Titlul capotilului
\label{Capitolul1}

\section{Verificarea programelor}

Odat\u a cu cre\c sterea complexit\u a\c tii proiectelor software, verificarea calit\u a\c tii programelor a devenit o problem\u a principal\u a a ingineriei software. Aceast\u a verificare este necesar\u a \^ in mai multe contexte. Odat\u a cu dezvoltarea platformelor de agen\c ti mobili, a ap\u arut nevoia de verificare a codului de la distan\c t\u a care urmeaz\u a s\u a fie executat de c\u atre un sistem gazd\u a. O alt\u a utilizare a metodelor de verificare a programelor este \^ in verificarea conformit\u a\c tii programelor cu o specifica\c tie dat\u a. De aseamenea, compilatoarele trebuie s\u a efectueze o serie de verific\u ari pentru a confirma legitimitatea aplic\u arii unor optimiz\u ari cu p\u astrarea semanticii programului ini\c tial. De exemplu, dac\u a se poate verifica faptul c\u a ambii termeni ai unei adun\u ari au valori constante, cunoscute la compilare, atunci rezultatul ei poate fi calculat, la r\^ andul sau, la momentul compil\u arii.

\section[Problema verific\u arii statice]{Problema verific\u arii statice a programelor}

De\c si in aplica\c tii practice este folosit\u a preponderent verificarea dinamic\u a a programelor sub forma test\u arii, aceast\u a abordare folose\c ste doar pentru demonstra\c tii negative ale unor propriet\u a\c ti. Din aceast\u a cauz\u a, solu\c tia ideal\u a pentru toate problemele enun\c tate in paragraful anterior ar fi existen\c ta unui instrument software care s\u a verifice static, doar pe baza codului surs\u a, anumite propriet\u ati ale programelor.

\subsection[Teorema lui Rice]{Teorema de nedecidabilitate a lui Rice}

Este un fapt \c stiut c\u a solu\c tia indicat\u a \^ in pragraful anterior este imposibil\u a.

\begin{theorem}[Rice]
Orice proprietate extensional\u a \c si netrivial\u a a programelor este nedecidabil\i a.
\end{theorem}

Aceast\u a teorem\u a se bazeaz\u a pe problema opririi ma\c sinilor Turing.

\begin{theorem}[Turing,1936]
Problema opririi unei ma\c sini Turing este nedecidabil\u a.
\end{theorem}

Teorema aceasta st\u a la baza rezultatelor de nedecidabilitate ale multor probleme care apar  \^ in construc\c tia compilatoarelor, de exemplu, \emph{"variabila x poate fi $>$ 0  \^ in punctul P din program"}.

Totu\c si experien\c ta ne arat\u a c\u a majoritatea programelor folosite  \^ i\c si termin\u a execu\c tia , sau prelucreaz\u a un flux de date de intrare \c si prelucrarea fiec\u arui element din fluxul de date de intrare se termin\u a .

\begin{example}
Compilatoarele sau func\c tiile de criptare sunt programe a c\u aror execu\c tie trebuie s\u a se termine. Alte exemple sunt func\c tiile folosite in implementarea unor structuri de date. Din a doua clas\u a, fac parte programele cu interfa\c ta grafic\u a \c si sistemele de operare. Prelucrarea fiec\u arei cereri din partea utilizatorului se termin\u a de fiecare dat\u a.
\end{example}

\subsection{Solu\c tii}

De\c si nedecidabil\u a, problema verific\u arii statice are mai multe solu\c tii par\c tiale.

Un exemplu de astfel de solu\c tie este aceea  \^ in care demonstra\c tia este generat\u a de un instrument software (theorem prover) care are la baz\u a o logic\u a cu o expresivitate ridicat\u a \c si care necesit\u a, eventual, interac\c tiune uman\u a. Aceast\u a variant\u a este folosit\u a de cel care produce programul pentru a demonstra anumite propriet\u a\c ti ale acestuia.  \^ In contextul dezvolt\u arii bazate pe componente, aceast\u a demonstra\c tie trebuie facut\u a o singur\u a dat\u a \c si apoi programul poate fi refolosit f\u ar\u a a reface demonstra\c tia. Un exemplu de astfel de instrument este Isabelle bazat pe logica de ordin  \^ inalt.

O alta variant\u a este cea folosit\u a  \^ in compilatoare pentru a decide propriet\u a\c tile necesare pentru a asigura corectitudinea optimiz\u arilor  \^ in ceea ce prive\u ste semantica limbajului de programare. Aceast\u a solu\c tie folose\c ste o procedur\u a de decizie corect\u a, dar incomplet\u a care poate decide doar anumite tipuri de propriet\u a\c ti \citep{1095594}. Aceast\u a solu\c tie se bazeaz\u a pe algoritmul Kildall \c si are meritul c\u a este rapid\u a \c si este complet automat\u a.

\^ In concluzie trebuie f\u acut un compromis  \^ intre expresivitatea metodei de demonstra\c tie pe de o parte \c si complexitatea algoritmului de generare a demonstra\c tiilor \c si a sistemului logic pe care se bazeaz\u a \c si necesitatea interven\c tiei umane pe de alt\u a parte.

\section{Theorem prover}

O abordare care folose\c ste theorem prover este cea numit\u a Proof-carrying code \citep{ATTAPL}.  \^ In aceast caz cel care scrie codul este responsabil s\u a furnizeze \c si o demonstra\c tie asupra functionalit\u a\c tii codului  \^ in conformitate cu specifica\c tia. Pentru aceast\u a sarcin\u a sunt folosite mai multe elemente:

\begin{itemize*}
  \item un \textbf{limbaj de specificare formal\u a} cu ajutorul c\u aruia se exprim\u a proprietatea de func\-\c ti\-o\-na\-re sigur\u a a programului
  \item o \textbf{logic\u a} pentru a face legatura intre \textbf{specificatie} si \textbf{semantica limbajului} de programare folosit. Se poate folosi, spre exemplu, logica Hoare \citep{363259}.
  \item un \textbf{limbaj pentru scrierea demonstra\c tiei}
  \item un \textbf{algoritm de validare a demonstra\c tiilor},  \^ in general asociat cu limbajul in care se exprima demonstratia.
  \item \textbf{instrumente pentru generarea automat\u a a demonstra\c tiilor (theorem provers)}, dar p\u ar\c tile mai complicate din demonstra\c tie pot fi f\u acute \c si manual.
\end{itemize*}
\done\todo{citez PCC}

Nevoia interac\c tiunii umane nu provine din considerente de expresivitate, ci de complexitate a algoritmului de generare a demonstra\c tiei. Ca principiu, interac\c tiunea uman\u a ar putea fi  \^ inlocuit\u a de o component\u a de nedeterminism a algoritmului. Acesta poate pur  \c si simplu \emph{s\u a ghiceasc\u a} deciziile luate de partea uman\u a. Concluzia c\u a interac\c tiunea uman\u a nu este necesar\u a din punctul de vederee al expresivit\u a\c tii este dat\u a de urmatoarea teorem\u a.

\begin{theorem}
Clasa de limbaje recunoscute de ma\c sini Turing deterministe coincide cu clasa de limbaje recunoscute de ma\c sini Turing nedeterministe.
\end{theorem}

Totu\c si, din punct de vedere al complexit\u a\c tii se crede c\u a ma\c sinile Turing nedeterministe sunt mai puternice dec \^ at cele deterministe,  \^ in sensul c\u a, pot rezolva o clas\u a mai bogat\u a de probleme  \^ in timp polinomial ( celebra problem\u a \emph{P = NP} ).

\subsection{Arhitectura}

\done\todo{comentat poza}
Dup\u a cum se vede din imaginea care prezint\u a un sistem de verificare ce folose\c ste un theorem prover, interac\c tiunea uman\u a este necesar\u a  \^ in mai multe etape ale procesului de demonstra\c tie. Pe lang\u a ac\c tiunea fireasc\u a de scriere de cod folosind un limbaj de programare, programatorul trebuie sa foloseasc\u a \c si un limbaj de specificare  \^ in care s\u a exprime proprietatea dorit\u a \c si un limbaj de scriere a demonstra\c tiei,  \^ in care s\u a completeze p\u ar\c tile mai complicate ale demonstra\c tiei f\u acute automat de c\u atre generatorul de demonstra\c tii.

\^ In general, limbajele de demonstra\c tie au o semantic\u a mult mai complex\u a dec\^ at limbajele de programare \c si sunt bazate pe fundamente logice matematice.

\begin{center}
    \begin{tikzpicture}[node distance=2cm, auto,>=latex', thick]
    \tikzstyle{tool} = [draw, thin, fill=blue!20, minimum width=7em, text width=6em, text centered]
    \tikzstyle{data} = [ellipse, draw, thin, fill=green!20, minimum width=5em, text width=5em, text centered]
    \tikzstyle{human} = [rectangle, thin, minimum height=7em, minimum width=2.5em ,text width=2em]

    \node[human,label=below:{\scriptsize Programator}] (om) {};
    \node[data, above of=om, xshift=3.5cm] (cod) {\scriptsize Cod};
    \node[data, below of=om, xshift=3.5cm] (spec) {\scriptsize Proprietate};
    \node[tool, right of=om, xshift=5cm] (proof_gen) {\scriptsize Generare dem.};
    \node[data, below of=proof_gen] (proof) {\scriptsize Demonstra\c tie};
    \node[tool, right of=proof_gen, xshift=2cm] (proof_check) {\scriptsize Verificare};
    \node[data, below of=proof_check] (yes_no) {\scriptsize DA / NU};
    \node[data, above of=proof_check] (semant) {\scriptsize Semantic\u a};

    \path[->] (om) edge node[above left] {\scriptsize Limbaj de}
                        node[left] {\scriptsize programare} (cod);
    \path[->] (om) edge node[left] {\scriptsize Limbaj de}
                        node[below left] {\scriptsize specificare} (spec);
    \path[->] (om) edge node[above] {\scriptsize Limbaj de}
                        node[below] {\scriptsize demonstra\c tie}  (proof_gen);
    \path[->] (cod) edge (proof_gen);
    \path[->] (spec) edge (proof_gen);
    \path[->] (proof_gen) edge (proof);
    \path[->] (proof) edge (proof_check);
    \path[->] (proof_check) edge (yes_no);
    \path[->] (semant) edge (proof_check);

    \path [draw] (om.north) + (0,-0.7cm) circle (0.2cm);                % cap
    \path [draw] (om.north) + (0,-0.9cm) -- + (0,-1.5cm);               % corp
    \path [draw] (om.north) + (-0.5cm,-1cm) -- + (+0.5cm,-1cm);         % maini
    \path [draw] (om.north) + (0,-1.5cm) -- + (-0.5cm,-2cm);            % picior st
    \path [draw] (om.north) + (0,-1.5cm) -- + (+0.5cm,-2cm);            % picior dr

\end{tikzpicture} 
%    \includegraphics[width=350pt]{thm_prover_arch.png}\\
    \label{thm_prover_arch}
\end{center}

\subsection{Conjectura Collatz}

Un dezavantaj al abord\u arii  \^ in care demonstra\c tiile includ p\u ar\c ti ce trebuie f\u acute de c\u atre programatori este c\u a exist\u a propriet\u a\c ti simple a caror demonstra\c tie se dovede\c ste suficient de complicat\u a. Astfel, programatorii sunt pu\c si  \^ in situa\c tia nefireasc\u a de a stabili dac\u a o proprietate este nedecidabil\u a, sau dac\u a merit\u a investit efort  \^ in demonstra\c tia ei. Un exemplu de astfel de demonstra\c tie care pare simpl\u a este \emph{problema 3n+1}:
\begin{quote}
    Plec\^ and de la un numar $n$ , daca el este par atunci \^il \^imp\u ar\c tim la 2, altfel \^il \^inmultim cu 3 si adun\u am 1. Apoi continu\u am procedeul cu num\u arul ob\c tinut. Este adev\u arat c\u a pornind de la orice numar, ajungem \^in final la 1?
\end{quote}

Aceast\u a problem\u a este echivalent\u a cu terminarea urmatorului program:

\begin{lstlisting}
--- Conjectura Collatz
let collatz |<Nat->Nat>| :=
    letrec |<Nat->Nat>| collatz_ :=
        fn |<Nat->Nat>| n =>
            (if|<Nat>|
                (eqNat n one)
                one
                (if|<Nat>|
                    (eqNat(mod n two) z)
                    (collatz_ (div n two))
                    (collatz_ (add (mul n three) one))
                )
            )
\end{lstlisting}

De\c si pare c\u a raspunsul este \emph{da} \c si propozi\c tia a fost verificat\u a pan\u a la numere foarte mari cu ajutorul calculatorului, problema este \^in stadiul de conjectur\u a din anul 1937. Alte probelme de acela\c si gen sunt teorema Fermat, conjectura Goldbach, etc.


\section{Demonstrarea automat\u a a termin\u arii}

O solu\c tie fezabil\u a pentru problema verific\u arii statice este una \^in care programatorii scriu cod folosind un limbaj de programare, iar propriet\u a\c tile dorite sunt demostrate \^in mod automat de c\u atre unele instrumente software, de exemplu de c\u atre compilator. \^In cele ce urmeaz\u a voi trata \^in principal proprietatea de terminare.

\subsection{Arhitectura}
\done\todo{comentat poza}

Arhitectura unui sistem de automat de verificare este mult mai simpl\u a. Singura ac\c tiune \^in care este implicat\u a partea uman\u a este cea de scriere de cod. Problema principal\u a cu aceast\u a arhitectur\u a este lipsa de flexibilitate, propriet\u a\c tile care se pot verifica fiind integrate \^in compilator. Totu\c si o proprietate de genul \emph{programul \^intoarce mereu rezultatul \textbf{TRUE}}, poate fi folosit\u a pentru a exprima echivalen\c ta a doua programe sau unii invarian\c ti.

\begin{center}
    \begin{tikzpicture}[node distance=1cm, auto,>=latex', thick]
    \tikzstyle{tool} = [draw, thin, fill=blue!20, minimum width=7em, text width=8em, text centered]
    \tikzstyle{data} = [ellipse, draw, thin, fill=green!20, minimum width=3em, text centered]
    \tikzstyle{human} = [rectangle, thin, minimum height=7em, minimum width=2.5em ,text width=2em]
    \tikzstyle{compiler} = [rectangle split, 
                            rectangle split parts=3, 
                            rectangle split draw splits=true, 
                            rectangle split part fill={red!20, blue!20, green!20}, 
                            text width=8em]
    \tikzstyle{sageata} = [ single arrow, 
                            minimum height=1.5em, 
                            minimum width=0.5em, 
                            single arrow head extend=0.3em]
                            
    \node[human,label=above:{\scriptsize Programator}] (om) {};
    \node[data, right of=om, xshift=3cm] (cod) {\scriptsize Cod};
    \node[compiler, right of=cod, xshift=3cm] (comp)
    {\scriptsize Compilator 
        \nodepart {second} {\scriptsize Verificare tip}
        \nodepart {third} {\scriptsize Proprietate $\leftrightarrow$ Tip}
    };
    \node[data, right of=comp, xshift=3cm] (yes_no) {\scriptsize DA / NU};

    \path [->] (om) edge node[above] {\scriptsize Limbaj de}
                         node[below] {\scriptsize programare} (cod);
    \path [->] (cod) edge (comp.text west);
    \path [->] (comp.east) edge (yes_no);


    \node[sageata, draw, shape border rotate=270, xshift=-0.4cm, yshift=-0.1cm] at (comp.text east){};
    \node[sageata, draw, shape border rotate=90 , xshift=-0.4cm, yshift=+0.1cm] at (comp.third east){};

%    \path [draw] (comp.text east) + (-0.3cm,0cm) -- +(-0.3cm , 0cm);
%    \path [draw] (comp.third east) edge (comp.second east);


    \path [draw] (om.north) + (0,-0.7cm) circle (0.2cm);                % cap
    \path [draw] (om.north) + (0,-0.9cm) -- + (0,-1.5cm);               % corp
    \path [draw] (om.north) + (-0.5cm,-1cm) -- + (+0.5cm,-1cm);         % maini
    \path [draw] (om.north) + (0,-1.5cm) -- + (-0.5cm,-2cm);            % picior st
    \path [draw] (om.north) + (0,-1.5cm) -- + (+0.5cm,-2cm);            % picior dr

    \path [draw,ultra thick] (comp.north west) rectangle (comp.text split east);
    \path [draw,ultra thin] (comp.text split west) rectangle (comp.second split east);
    \path [draw,ultra thin] (comp.second split west) rectangle (comp.third split east);
    \path [draw,ultra thick] (comp.north west) rectangle (comp.third split east);

%    \path [draw] (prop.north west) + (-2cm, 1cm) rectangle (type_check.south east) + (1cm, 1cm);
\end{tikzpicture} 
%    \includegraphics[width=350pt,height=100pt]{tbt_arch.png}\\
    \label{tbt_arch}
\end{center}


\subsection{Abord\u ari existente}

O prim\u a simplificare a problemei este reprezentat\u a de \^inlaturarea buclelor de control din limbajul de programare. Astfel singura posibilitate de a avea o expresie a c\u arei evaluare nu se termin\u a este dat\u a de prezen\c ta func\c tiilor textual recursive. Dac\u a impunem asupra acestora restic\c tia ca fiecare apel al unei func\c tii recursive s\u a fie facut asupra unor parametri \emph{mai mici}, atunci stiva de apeluri succesive va fi un sir de descrescator de parametri. Pentru a demonstra finitudinea stivei de apeluri recursive, este suficient s\u a ar\u atam c\u a rela\c tia de \emph{mai mic} este \fixme{well-founded}.

\begin{definition}\label{well_founded}
O rela\c tie de ordine $R$ este \fixme{\textbf{well-founded}} dac\u a nu exist\u a nici un \c sir infinit de elemente $x_1, \dots , x_n, \dots$ astfel \^incat $x_{i+1} R x_i,\forall i$.
\end{definition}

\done\todo{exemplu condi\c tii sintactice}
\begin{example}
Un exemplu de rela\c tie de ordine de natur\u a sintactic\u a \^intre expresii este urmatoarea:
    \begin{itemize*}
      \item Argumetele s\u a apar\c tin\u a unui tip de pentru care relatia de subtermen este \fixme{well-founded} (de exemplu tip de date algebric).
      \item Apelurile recursive ale func\c tiei sunt f\u acute doar asupra unor subtermeni ai argumentelor. Ace\c sti subtermeni sunt ob\c tinuti prin deconstruc\c tia argumentelor, folosind de exemplu analiza de cazuri sau pattern matching (\^in limbajul Haskell).
    \end{itemize*}
\end{example}

\done\todo{dezavantaje condi\c tii sintactice, din 5}
Un dezavantaj \^in folosirea condi\c tiilor de natur\u a sintactic\u a este c\u a acestea sunt foarte sensibile la modul \^in care este scris programul.

\begin{example}
De exemplu programul \ref{syntactic_ok} respect\u a condi\c tia de mai sus, \^in timp ce programul \ref{syntactic_bad}, de\c si echivalent din punct de vedere func\c tional, nu o respect\u a.
\end{example}
\begin{lstlisting}[label=syntactic_ok,captionpos=b,caption=Exemplu corect,float=tb]
letrec const0 |<Nat->Nat>| :=
fn |<Nat>| n => case n of {
    z => z
    s => fn|<Nat>| pn =>
        (const0 pn)
}
\end{lstlisting}
\begin{lstlisting}[label=syntactic_bad,captionpos=b,caption=Exemplu incorect,float=tb]
letrec const0 |<Nat->Nat>| :=
fn |<Nat>| n => case n of {
    z => z
    s => fn|<Nat>| pn =>
        (const0 ((fn |<Nat>| x => x) pn))
}
\end{lstlisting}
\done\todo{recursori pe structuri de date, catamorfism}

\subsection{Integrarea cu teoria tipurilor}

O solu\c tie natural\u a este aceea de a face demonstra\c tiile \^in faza de compilare a programului. Pasul cel mai potrivit \^in cadrul compil\u arii este cel al analizei de tip, c\^and compilatorul oricum face demonstra\c tii ale faptului c\u a o expresie are un anumit tip. Ideea termin\u arii bazate pe tipuri este aceea de a asocia tipuri doar acelor expresii a c\u aror evaluare se termin\u a.

Avantajul principal al acestei solu\c tii este acela c\u a programatorii nu trebuie s\u a inve\c te noi limbaje de specifica\c tie, \c si sisteme logice de demonstra\c tie. Folosirea unor sisteme de tipuri, fie ele \c si mai exotice, reprezint\u a o sarcin\u a mai u\c soar\u a pentru un programator dec\^at folosirea unor formalisme matematice complexe. Sistemele de tipuri ascund \^in general fundamentele teoretice pe care se bazeaz\u a \t si astfel pot fi folosite si de c\u atre cei care nu le \^in\c teleg.

\^ In ciuda simplit\u a\c tii, aceast\u a metod\u a are dezavantajul c\u a atunci c\^and sistemul de tipuri nu este suficient de puternic pentru a demonstra terminarea evalu\u arii unei expresii, aceasta nu poate fi facut\u a manual. Un alt dezavantaj este c\u a limbajul care folose\c ste acest sistem de tipuri nu este Turing complet.  % Motivatie

% Capitolul 2 - 7 pag

\chapter{No\c tiuni preliminarii}
\label{Capitolul2}

\^ In aceasta lucrare voi prezenta construc\c tia unui compilator pentru un limbaj de programare func\c tional care verific\u a static (corect dar incomplet) proprietatea de terminare a programelor pe baza teoriei tipurilor.

\section{Terminologie}
\done\todo{de scris exact defini\c tia normalizarii puternice}
\begin{definition}
Un sistem de tipuri are proprietatea de \textbf{normalizare puternic\u a} dac\u a pentru orice expresie c\u areia \^ ii este asociat un tip exist\u a un num\^ ar $n\in\mathbb{N}$ astfel \^ incat orice evaluare a expresiei p\^ n\u a la o valoare s\u a necesite mai pu\c tin de $n$ pa\c si.
\end{definition}

\begin{definition}
Un sistem are proprietatea de \textbf{conservare} dac\u a pentru orice context de tip $\Gamma$ , daca $\Gamma \vdash t : T$ \c si $t \to t'$, atunci $\Gamma \vdash t' : T$
\end{definition}

\begin{definition}
Un sistem are proprietatea de \textbf{progres} dac\u a orice expresie $t$,  fie $t$ este o \emph{valoare}, fie exist\u a o alt\u a expresie $t'$ astfel \^ incat $t \to t'$.
\end{definition}

\begin{definition}
Un sistem are proprietatea de \textbf{siguran\c ta} dac\u a are \c si proprietatea de conservare, \c si proprietatea de progres.
\end{definition}

\section{Vedere de ansamblu}

Exemple simple de limbaje cu proprietatea de normalizare puternic\u a sunt \emph{calculul lambda cu tipuri simple (simply typed lambda calculus)} sau \emph{calculul lambda polimorfic (System F)}. Totu\c si motivul pentru care aceste sisteme de tipuri nu sunt folosite \^ in practic\u a este c\u a unele concepte comune (de ex. numere naturale) au o reprezentare complicat\u a \c si multi algoritmi sunt expresibili doar cu o complexitate foarte mare.

\begin{example}
De exemplu, pentru a calcula predecesorul unui num\u ar natural, trebuie ca nu\-m\u a\-rul sa fie reprezentat \^ in forma Church, apoi algoritmul de are o complexitate $O(n)$ \^ in valoarea numarului. Un exemplu de astfel de implementare \^ in Haskell este prezentat\u a \^ in programul \ref{church_pred}.
\done\todo{implementare haskell}
\end{example}

\begin{lstlisting}[label=church_pred,captionpos=b,caption=Func\c tia predecesor \^ in codificare Church,language=Haskell,float=tb]
-- perechi in codificare Church
cpair = \x y -> \p -> p x y
cfst  = \p -> p (\x y -> x)
csnd  = \p -> p (\x y -> y)
-- numere naturale \^ in codificare Church
czero = \s z -> z
csucc = \n -> \s z -> s (n s z)
-- func\c tia predecesor \^ in codificare Church
cpred = \n -> n
            (\m -> (cpair (csucc (cfst m)) (cfst m)) )
            (cpair czero czero)
\end{lstlisting}

Din aceast\u a cauza a fost introdus\u a o extensie a System F - System \frec, ce con\c tine \^ in plus tipuri de date \c si func\c tii recursive. Din p\^ acate acest sistem de tipuri nu mai garateaz\^ a finitudinea evalu\u arii expresiilor c\u arora le asociaz\u a un tip. \^In final o s\u a introduc System \fhat care este o extindere a System \frec, cu unele constr\^ angeri \c si care respect\u a proprietarea de normalizare puternic\u a.

\section{System F}

Calculul lambda polimorfic sau System F a fost descoperit de Jean-Yves Girard \citep{64805} \^ in contextul \emph{teoriei demonstra\c tiei} \^ in logica. Acest sistem de tipuri a fost folosit ca un cadru pentru studiul teoretic al polimorfismului \c si ca baz\u a pentru proiectarea mai multor limbaje de programare.

System F mai este numit \c si calcul lambda de ordinul doi. Acest nume provine de la faptul c\u a \^ in transpunerea sa \^ in logica prin izomorfismul Curry-Howard, sunt permise propozi\c tii cuantificate nu doar peste termeni de ordinul \^ intai - lambda expresii, ci \c si peste predicate - tipuri.

\subsection{Sintaxa}

Sintaxa acestui limbaj este una simpl\u a, dat\u a de urmatoarea gramatic\u a:

\begin{multicols}{2}
\setlength\columnseprule{.4pt}
\begin{align*}
t :=  &                  &\textbf{lambda-expresii}\\
      &x                 &\text{variabila}\\
      &\lambda x:T . t   &\text{functie lambda}\\
      &t\app t    &\text{aplicare functie}\\
      &\Lambda X . t     &\text{paramtetrizare tip}\\
      &t\app [T]  &\text{instantiere tip}
\end{align*}
\begin{align*}
v :=  &                  &\textbf{valori}\\
      &\lambda x:T . t   &\text{val. functie}\\
      &\Lambda X . t     &\text{val. param. tip}\\
T :=  &                  &\textbf{tipuri}\\
      &X                 &\text{variabila de tip}\\
      &T\to T            &\text{tip functional}\\
      &\Pi X.T       &\text{tip universal}
\end{align*}
\end{multicols}

Se observ\u a c\u a, spre deosebire de calculul lambda simplu, expresiile pot fi parametrizate \c si de o variabil\u a de tip. Valorile sunt forme normale pentru expresii. Evaluarea unei expresii se \^ incheie atunci c\^ and ea devine o valoare.

\begin{example}\label{church_numbers}
\^ In acest limbaj de programare putem defini func\c tii polimorfice precum $id = \Lambda X . (\lambda x:X . x) $. De asemenea, spre deosebire de calculul lambda cu tipuri simple aici se pot exprima numere naturale \c si valori booleene \^ in forma Church
\begin{align*}
    \emph{true} &= \Lambda X.\lambda t:X.\lambda f:X.t                     \\
    \emph{ifthenelse}   &= \Lambda X . \lambda c : (\forall X. X \to X \to X). \lambda t : X.\lambda f : X . x\app [X]\app y\app z\\
    \emph{two}  &= \Lambda X.\lambda s:(X \to X).\lambda z:X . s\app s\app z \\
\end{align*}
\end{example}

\subsection{Reguli de evaluare}
\done\todo{reguli evaluare sysf}
Semantica opera\c tional\u a a limbajului este dat\u a de urmatoarele reguli de evaluare:

\begin{multicols}{2}
\setlength\columnseprule{.4pt}

\begin{prooftree}
\AxiomC{$t_1 \to t_1'$}
\RightLabel{\scriptsize(CTX-APP)}
\UnaryInfC{$t_1\app t_2 \to t_1' \app t_2$}
\end{prooftree}

$(\lambda x :T_{11} . t_{12})\app v_2 \to [x \mapsto v_2] t_{12}$ {\scriptsize (E-APP)}
\columnbreak

\begin{prooftree}
\AxiomC{$t_1 \to t_1'$}
\RightLabel{\scriptsize(CTX-TAPP)}
\UnaryInfC{$t_1\app [T_2] \to t_1' \app [T_2]$}
\end{prooftree}

$(\Lambda X . t_{12})\app [T_2] \to [X \mapsto T_2] t_{12}$ {\scriptsize(E-TAPP)}
\end{multicols}
\hyphenation{e-va-lu-at}
Aceste reguli corespund unei strategii de evaluare lene\c s\u a, argumentul unei func\c tii nefiind evaluat dec\^ at dup\u a aplicarea func\c tiei. 

\subsection{Reguli de tip}

Pentru determinarea tipului unei expresii trebuie folosit un context de tip \^ in care se re\c tin in\-for\-ma\-\c ti\-i despre variabilele care apar libere \^ in expresie. Tipurile asociate expresiilor \c si contextele de tip se determin\u a dup\u a urmatoarele reguli:

\begin{multicols}{2}
\setlength\columnseprule{.4pt}
\flushleft{\textbf{Context de tip}}
\begin{align*}
\Gamma := &             &\textbf{context de tip}\\
          & \Phi        &\text{context gol}\\
          &\Gamma,x:T   &\text{asociere var-tip}\\
          &\Gamma,X     &\text{var. de tip legata}\\
\end{align*}
\flushleft{\textbf{Reguli de tip}}

\begin{prooftree}
\AxiomC{$x:T\in\Gamma$}
\RightLabel{\scriptsize(T-VAR)}
\UnaryInfC{$\Gamma \vdash x:T$}
\end{prooftree}

\begin{prooftree}
\AxiomC{$\Gamma,x:T_1 \vdash t_2:T_2$}
\RightLabel{\scriptsize(T-ABS)}
\UnaryInfC{$\Gamma \vdash \lambda x :T_1. t_2 : T_1 \to T_2$}
\end{prooftree}

\begin{prooftree}
\AxiomC{$\Gamma \vdash t_1 : T_{1} \to T_{2}$}
\AxiomC{$\Gamma \vdash t_2 : T_{1}$}
\RightLabel{\scriptsize(T-APP)}
\BinaryInfC{$\Gamma \vdash t_1 \app t_2 : T_{2}$}
\end{prooftree}

\begin{prooftree}
\AxiomC{$\Gamma,X \vdash t_2 : T_2 $}
\RightLabel{\scriptsize(T-TABS)}
\UnaryInfC{$\Gamma \vdash \Lambda X.t_2 : \Pi X . T_2$}
\end{prooftree}

\begin{prooftree}
\AxiomC{$\Gamma \vdash t_1 : \Pi X.T_1 $}
\RightLabel{\scriptsize(T-TAPP)}
\UnaryInfC{$\Gamma \vdash t_1\app [T_2] : [X \mapsto T_2] T_1$}
\end{prooftree}

\end{multicols}

Determinarea tipurilor pentru System F este un procedeu simplu, dirijat de sintax\u a. Singura dificultate const\u a \^ in substitu\c tia $[X \mapsto T_2] T_1$ la care trebuie \^ inlocuite doar variabilele $X$ libere \^ in tipul $T_1$.

\begin{example}
Dup\u a cum se vede \^ in exemplul \ref{church_numbers}, tipurile \textbf{\emph{Bool}} \c si \textbf{\emph{Nat}} pot fi definite ca prescurt\u ari pentru $\Pi X. X \to X \to X$ respectiv $\Pi X. (X \to X) \to X\to X$.
\end{example}

\subsection{Propriet\u a\c ti}
\begin{theorem}[Siguranta]
System F are propriet\u a\c tile de conservare \c si progres, deci are proprietatea de siguran\c t\u a.
\end{theorem}
\begin{proof}[Demonstra\c tie]
Demonstra\c tia poate fi gasit\u a \^ in \citep{Pierce:TypeSystems}.
\end{proof}
\hyphenation{pu-ter-ni-ca}
\begin{theorem}[Normalizare puternic\u a]
System F are proprietatea de normalizare puternic\u a.
\end{theorem}
\begin{proof}[Demonstra\c tie]
Demonstra\c tia poate fi gasit\u a \^in \citep{64805}.
\end{proof}

\begin{corollary}
System F nu este Turing complet.
\end{corollary}
\begin{proof}[Demonstra\c tie]
Teorema este o consecin\c t\u a direct\u a a teoremei \ref{turing_incomplete} utilizate \^ in demonstra\c tia incompletitudinii pentru System \fhat.
\end{proof}

\section[System F rec]{System \frec}

System \frec \citep{1614481} este o extensie a System F care are \^ in plus tipuri de date recursive \c si functii recursive texutal. Acest lucru face ca exprimarea unor concepte naturale precum numere, valori booleene, liste, arbori s\u a poata fi facut\u a mai direct dec\^ at prin codificare Church. Mai mult, prin ad\u augarea func\c tiilor recursive, Sytem \frec devine Turing complet.

\subsection{Sintaxa}

\subsubsection{Declara\c tii de tipuri}
Fiecare tip de date este determinat de un identificator (constructor) \c si de aritatea acelui identificator. To\c ti identificatorii fac parte din mul\c timea $\mathcal{D}$.

\begin{example}
Tipul numerelor naturale este are identificatorul \emph{\textbf{Nat}} \c si de aritatea $ar(Nat) = 0$. Tipul listelor polimorfice este dat de identificatorul \textbf{\emph{List}} \c si aritatea $ar(List) = 1$, parametrul reprezent\^ and tipul elementelor ce se afl\u a in lista.
\end{example}

Fiecare tip de date $d \in \mathcal{D}$ are o mul\c time de constructori $\mathcal{C}(d)$. Mul\c timile de constructori sunt disjuncte pentru constructori diferi\c ti $\mathcal{C}(d_1) \cap \mathcal{C}(d_2) = \emptyset $ pentru $d_1 \neq d_2$. Fiecare constructor $c\in \mathcal{C}(d)$ are tipul de forma
$$ \Pi \textbf{X}.\theta_1\to\dots\to\theta_n \to d \textbf{X}$$
unde $\textbf{X}$ este un vector de parametrii de dimensiune egal\u a cu aritatea lui $d$.

\begin{example}
\label{nat_def} Multimea de constructori pentru \emph{\textbf{Nat}} este:
$$\mathcal{C}(\emph{\textbf{Nat}})=\{z : \emph{\textbf{Nat}} , s : \emph{\textbf{Nat}} \to \emph{\textbf{Nat}} \}.$$
Mul\c timea de constructori pentru \emph{\textbf{List}} este
$$\mathcal{C}(\emph{\textbf{List}})=\{ nil : \Pi X. \emph{\textbf{List}}\app X , cons : \Pi X . X \to \emph{\textbf{List}}\app X \to \emph{\textbf{List}}\app X\}.$$
\end{example}

Declara\c tia unui tip care include identificatorul de tip, parametri tipului, constructorii \c si tipurile acestora are urmatoarea sintax\u a:
$$ \textbf{Datatype  } d\app X := c_1 : \Pi X . \theta_1 \to d \app X \ | \
                                 \ldots \ |\
                                 c_k : \Pi X . \theta_k \to d\app X
$$
unde $X = X_1 \dots X_n$, $\ ar(d) = n$, $\mathcal{C}(d) = \{c_1, \dots,c_k\}$, $\theta_i = \theta_{i_1} \to \dots \to \theta_{i_m} $.
\begin{example}
Declara\c tiile pentru valori booleene, numere naturale \c si liste sunt urm\u atoarele:
\begin{align*}
 \textbf{\emph{Datatype}  } \emph{\textbf{Bool}}       &:= true : \emph{\textbf{Bool}} \ |\  false : \emph{\textbf{Bool}} \\
 \textbf{\emph{Datatype}  } \emph{\textbf{Nat}}        &:= z : \emph{\textbf{Nat}} \ |\ s : \emph{\textbf{Nat}} \to \emph{\textbf{Nat}} \\
 \textbf{\emph{Datatype}  } \emph{\textbf{List}}\app X &:= nil : \Pi X. \emph{\textbf{List}}\app X\ |\ cons :  \Pi X.X \to \emph{\textbf{List}}\app X \to \emph{\textbf{List}} \app X
\end{align*}
\end{example}

\subsubsection{Tipuri}

Pe lang\u a tipurile de date ale System F, se mai adaug\u a tipuri de forma
$$ d\app T_1\app T_2\app \dots\app T_n  \text{, unde } ar(d) = n$$
\begin{example}
Un astfel de tip este \emph{\textbf{Nat}} sau \emph{\textbf{List}} \app \emph{\textbf{Nat}}.
\end{example}
Condi\c tia ca num\u arul de argumente s\u a fie egal cu aritatea constructorului de tip este asemanatoare cu aceea care cere ca o func\c tie s\u a fie aplicat\u a unui num\^ ar de argumente egal cu aritatea sa. Aceast\u a a doua conditie se rezolva in general prin verificarea tipului expresiei rezultate. Pentru a face acela\c si lucru pentru tipuri putem acorda fiec\u arui tip un \emph{kind} introduc\^ and practic un nou sistem de tipuri la nivelul tipurilor.
\begin{example}
Kind-ul pentru \emph{\textbf{Nat}} este $*$, kind-ul pentru \emph{\textbf{List}} este $* \Rightarrow *$.
\end{example}
Kind-ul fiec\ uarei expresii trebuie s\u a fie $*$. Din fericire, orice identificator are un kind de genul $* \Rightarrow \dots \Rightarrow *$ \c si deci sistemul de kinduri este mult mai simplu dec\^ at calculul lambda cu tipuri simple.

\subsubsection{Termeni}
La termenii System F se mai adaug\u a urmatoarele construc\c tii sintactice:
\begin{align*}
t :=  &\dots                                    &\textbf{lambda-expresii}\\
      &\mathcal{C}                              &\text{constructori}\\
      &\text{case}_T of \{\mathcal{C} \Rightarrow t\}     &\text{analiza de cazuri}\\
      &\text{letrec}_T \{ x = t \}                       &\text{recursivitate textuala}
\end{align*}
Aceste construc\c tii sunt \^ int\^ alnite \^ in limbaje de programare func\c tional\u a precum Haskell \c si ML. Dintre ace\c sti termeni, contructorii, aplica\c tiile de constructori \c si defini\c tiile de func\c tii textual recursive sunt considerate valori.

\subsection{Reguli de evaluare}

System \frec adaug\u a reguli de evaluare pentru noile construc\c tii introduse.

\begin{prooftree}
\AxiomC{$e \to e' $}
\RightLabel{\scriptsize (CTX-CASE)}
\UnaryInfC{$\text{case}_{\sigma}\ e \text{ of } \{ c_1 \Rightarrow \ e_1 |\ \dots \ |\ c_k \Rightarrow e_k \} \to \text{case}_{\sigma}\ e' \text{ of } \{ c_1 \Rightarrow \ e_1 |\ \dots \ |\ c_k \Rightarrow e_k \} $ }
\end{prooftree}
\begin{prooftree}
\AxiomC{$e \to e' $}
\RightLabel{\scriptsize (CTX-LETREC)}
\UnaryInfC{$(\text{letrec}_{d\app \tau \to \sigma} f = g)\app e  \to (\text{letrec}_{d\app \tau \to \sigma} f = g)\app e'  $ }
\end{prooftree}
$$\text{case}_{\sigma}\ c_i\app \tau\app a \text{ of } \{ c_1 \Rightarrow \ e_1 |\ \dots \ |\ c_k \Rightarrow e_k \} \to e_i \app a \text{ \scriptsize (E-CASE)}$$
$$ (\text{letrec}_{d\app \tau \to \sigma} f = g)(c\app a) \to [ f \mapsto (\text{letrec}_{d\app \tau \to \sigma} f = g)] g\app (c\app a) \text{  \scriptsize (E-LETREC)} $$

Se observ\u a c\u a regula de evaluare pentru analiza de cazuri corespunde unei strategii lene\c s\u a, evalu\^ andu-se \^ int\^ ai expresia care este analizat\u a \c si abia apoi, doar expresia corespunz\u atoare cazului care se aplic\u a.

\subsection{Reguli de tip}
Regulile de tip ad\u augate au ca rol validarea construc\c tiilor sintactice nou introduse
\begin{prooftree}
\AxiomC{}
\RightLabel{\scriptsize(T-CONS)}
\UnaryInfC{$\Gamma \vdash c_k : \Pi X . \theta_k \to d \app X$}
\end{prooftree}

\begin{prooftree}
\AxiomC{ $\Gamma , f : \tau \vdash e : \tau $}
\RightLabel{\scriptsize(T-LETREC)}
\UnaryInfC{$\Gamma \vdash (\text{letrec}_\tau f = e) : \tau $}
\end{prooftree}

\begin{prooftree}
\AxiomC{ $\Gamma \vdash e : d \app \tau $}
\AxiomC{ $\Gamma \vdash e_k : [X \mapsto \tau]\theta_k \to \sigma \  ( 1 \le k \le n) $}
\RightLabel{\scriptsize(T-CASE)}
\BinaryInfC{$\Gamma \vdash \text{case}_{\sigma}\ e \text{ of } \{ c_1 \Rightarrow e_1\  |\ \dots \ |\ c_n \Rightarrow e_n \} : \sigma$}
\end{prooftree}

Se observ\u a c\u a tipul expresiilor \emph{case} \c si \emph{letrec} este deja specificat, astfel regulile au ca rol doar verificarea unor constr\^ angeri asupra expresiei.

\begin{remark}
\^ In cazul \emph{letrec}, este necesar\u a specificarea tipului. Un argument pentru acest fapt este urmatoarea expresie:
$$ \text{letrec } f = \lambda x:{\textbf{Nat}} . f\app x $$
care poate avea orice tip de forma $\bf{Nat} \to X $.

Pentru \emph{case} nu este necesar\u a specificarea tipului. Pentru a elimina anota\c tia de tip putem \^ inlocui regula (E-CASE) cu urmatoarea regul\u a:

\begin{prooftree}
\AxiomC{ $\Gamma \vdash e : d \app \tau $}
\AxiomC{ $\Gamma \vdash e_k : [X \mapsto \tau]\theta_k \to \sigma_k \  ( 1 \le k \le n) $}
\AxiomC{ $\sigma_i = \sigma_j \ \forall i,j $}
\RightLabel{\scriptsize(T-CASE')}
\TrinaryInfC{$\Gamma \vdash \text{case}\ e \text{ of } \{ c_1 \Rightarrow e_1\  |\ \dots \ |\ c_n \Rightarrow e_n \} : \sigma_1$}
\end{prooftree}

\end{remark}

\done\todo{se poate renun\c ta la specifica\c tiile de tip pentru letrec / case?}

\subsection{Propriet\u a\c ti}
\begin{theorem}[Siguran\c ta]
System \frec are propriet\u a\c tile de progres \c si conservare, deci are proprietatea de siguran\c t\u a.
\end{theorem}

\begin{theorem}[Completitudine Turing] \label{compl_sysfrec}
System \frec este Turing complet.
\end{theorem}\done\todo{demonstra\c tie? completitudine turing}
\begin{proof}[Demonstra\c tie]
Voi ar\u ata c\u a orice func\c tie $\mu$-recursiv\u a este caclulabil\u a in System \frec. Consider\u am defini\c tia pentru tipul de numere naturale din exemplul \ref{nat_def}. Func\c tiile constant\u a, succesor, identitate \c si opreatorii de compunere, recursivitatea primitiv\u a \c si minimizare sunt calculabile.

\begin{align*}
\text{const}_{k}  &:= \lambda x_1:{\textbf{Nat}}. \dots \lambda x_k:{\textbf{Nat}}. z \\
\text{succ}         &:= \lambda x:{\textbf{Nat}} . s\app x\\
\text{id}_{k,n}     &:= \lambda x_1:{\textbf{Nat}}. \dots \lambda x_n:{\textbf{Nat}}.x_k \\
\text{comp}_{k,m}   &:= \lambda h: \textbf{Nat}^{m} \to \textbf{Nat}. \lambda g_1: \textbf{Nat}^{k} \to \textbf{Nat} \dots \lambda g_m:\textbf{Nat}^{k} \to \textbf{Nat}. \\
                    & \lambda x_1:\textbf{Nat}. \dots \lambda x_k:\textbf{Nat}. h \app (g_1\app \bar{x} )\app \dots \app (g_m\app \bar{x}) \\
\text{p\_rec}_k     &:= \lambda g:\textbf{Nat}^{k} \to \textbf{Nat}. \lambda h:\textbf{Nat}^{k+2} \to \textbf{Nat}.\\
                    & \text{letrec}_{\textbf{Nat}^{k+1} \to \textbf{Nat}} \ f =\lambda y:\textbf{Nat}.\lambda x_1:\textbf{Nat}. \dots \lambda x_k:\textbf{Nat}. \text{case}_{\textbf{Nat}}\ y \text{ of } \\
                    & \qquad z \Rightarrow (g\app \bar{x}) \\
                    & \qquad s \Rightarrow \lambda y' : \textbf{Nat}. (h\app y'\app (f\app y'\app \bar{x})\app \bar{x})\\
\text{min}_k        &:= \lambda f. \lambda x_1:\textbf{Nat} \dots \lambda x_n:\textbf{Nat}. \\
                    & (\text{letrec}_{\textbf{Nat}^{k+1} \to \textbf{Nat}}\  \mu = \lambda y:\textbf{Nat}. \text{case}_{\textbf{Nat}}\  f(y, \bar{x}) \text{ of } \\
                    & \qquad z \Rightarrow y \\
                    & \qquad s \Rightarrow \lambda x' .  \mu\app (s\app y))\app z
\end{align*}


unde $\bar{x} = x_1\app \dots\app x_k$ si $\textbf{Nat}^k = \textbf{Nat}\to\dots\to\textbf{Nat}$. Pentru simplitate am omis adnotarile de tip. \^ In orice caz toate variabilele sunt fie numere naturale fie func\c tii care lucreaz\u a cu numere naturale.

Cum multimea functiilor $\mu$-recursive coincide cu mul\c timea func\c tiilor calculabile Turing, am ob\c tinut completitudinea limbajului.
\end{proof}
\begin{remark}
O consecint\u a trivial\u a a completitudinii Turing este faptul c\u a System \frec nu are proprietatea de normalizare puternic\u a. Un exemplu de func\c tie a c\u arei evaluare nu se termin\u a este $(min_1\app succ)$.
\end{remark}

\done\todo{mai multe detalii.. Totu\c si se poate formula o versiune f\u ar\u a func\c tii recursive textual care s\u a aib\u a proprietaeta de normalizare puternic\u a}.

 % Notiuni preliminare

% Capitolul 3 - 7 pag

\chapter{System F\^{}}
\label{Capitolul3}

Se observ\u a c\u a pentru a rec\u apata proprietatea de normalizare puternic\u a, trebuie impuse unele constr\^ angeri asupra System \frec.
\section{Scheme de recursivitate}
\label{scheme_rec}
\^ Inainte de a stabili ce constr\^ angeri trebuie impuse, vom prezenta unele scheme de recursivitate des folosite \citep{varmo_vene}. Acestea trebuie s\u a poat\u a fi exprimate \^ in limbaj \^ intr-un mod natural \c si cu o complexitate mic\u a.

\subsection{Recursivitate structural\u a}
Acest tip de recursivitate este folosit\u a \^ in general pe structuri de date cu un \emph{tip de date algebric} \c si se aplic\u a pentru cazul \^ in care valoarea func\c tiei pentru un termen depinde doar de valorile func\c tiei pe subtermenii direc\c ti ai acestui termen.
\begin{definition}
Un tip de date recursiv se nume\c ste \textbf{tip de date algebric} dac\u a fiecare constructor are tipul $ c_i : \Pi X . T_1\app X_1 \to \dots \to T_n\app X_n\to T X $, unde $T_k$ sunt identificatori de tipuri de date, si $X_k \subset X$.
\end{definition}
Aceste tipuri se mai numesc \emph{sume de produse}. Aceast\u a denumire provine din faptul c\u a un termen cu constructorul principal $c : T_1\to \dots T_n\to T $ poate fi reprezentat de tuplul argumentelor c\u arora acesta le este aplicat $(t_1,\dots,t_n)\in P = T_1 \times \dots \times T_n $, deci de un element din produsul cartezian al $T_1, \dots, T_n$. Cum tipul de date are mai mul\c ti constructori, elementele apartin $P_1 \cup \dots \cup P_m $ cu $P_i$ produse carteziene. \^ In teoria categoriilor reuniunea a doua mul\c timi se cheam\u a \emph{coprodus} sau \emph{sum\u a}, iar produsul cartezian \emph{produs}.
\begin{example}
Un exemplu de astfel de func\c ie recursiv\u a este functia
\begin{align*}
double(succ(x)) &= 2 + double(x)\\
      double(0) &= 0
\end{align*}
\end{example}

\subsection{Recursivitate primitiv\u a}
Recursivitatea primitiv\u a este folosit\u a atunci c\^ and valoarea func\c tiei pentru un termen depinde de valorile func\c tiei pentru subtermenii direc\c ti ai termenului \c si de ace\c sti subtermeni. Se observ\u a c\u a recursivitatea primitiv\u a este mai expresiva decat cea structural\u a.

\begin{example}
Un exemplu de astfel de func\c tie recursiv\u a este func\c tia
\begin{align*}
fact(succ(x)) &= (x+1) \cdot fact(x)\\
      fact(0) &= 1
\end{align*}
\end{example}

\subsection{Recursivitate couse-of-value}
\done\todo{cum sa zic in romana course-of-value}
Recursivitatea {course-of-value} difer\u a de cea primitiv\u a prin faptul c\u a valoarea func\c tiei pentru un termen poate depinde \c si de valoare func\c tiei pentru subtermeni ai acestuia care nu sunt direc\c ti, dar ad\^ ancimea la care apar este finit\u a.

\begin{example}
Un exemplu de astfel de func\c tie recursiv\u a este func\c tia
\begin{align*}
fibo(succ(succ(x))) &= fibo(succ(x)) + fibo(x)\\
      fibo(1) &= 1 \\
      fibo(0) &= 1
\end{align*}
\end{example}

Se observ\u a c\u a printr-un procedeu de memoizare, aceast\u a schem\u a de recursivitate este echivalent\u a cu recursivitatea primitiv\u a.
\begin{example}
Func\c tia din exemplul anterior se poate rescrie ca
\begin{align*}
fibo(n) &= x \text{ unde } (x,y) = fibo'(n)\\
fibo'(succ(x)) &= (x+y, x) \text{ unde } (x,y) = fibo(x)\\
      fibo'(0) &= (1,0)
\end{align*}
\end{example}
Totu\c si, pentru a codifica aceast\u a schem\u a de recursivitate cu func\c tii primitiv recursive, trebuie folosit\u a o tehnic\u a  artificial\u a.
\section{Renun\c tarea la recursivitatea textuala}

Din demonstra\c tia teoremei \ref{compl_sysfrec} se poate observa c\u a pentru introducerea posibilit\u a\c tii de exprimare a unor func\c tii par\c tiale prin operatorul de minimizare s-a folosit recursivitatea textual\u a. O idee natural\u a pentru redob\^ andirea normaliz\u arii puternice este eliminarea acestei construc\c tii din limbaj. Totu\c si urmatoarea propozi\c tie infirm\u a proprietatea normaliz\u arii puternice pentru System \frec f\u ar\u a recursivitate textual\u a.
\begin{proposition}\label{case_nonterm}
Se pot construi \^ in System \frec expresii care nu se reduc la o form\u a normal\u a f\u ara\u  a folosi recursivitatea textual\u a.
\end{proposition}
\begin{proof}[Demonstra\c tie]
Vom porni de la urmatoarea lem\u a.
\begin{lemma}\label{inconsistence_to_nonterm}
Dac\u a exist\u a expresii cu tipul $\bot := \Pi X. X$, atunci System \frec nu are proprietatea de normalizare puternic\u a.
\end{lemma}
\done\todo{demonstratie lema}
Vom \^ incerca apoi s\u a construim un termen de tipul $\bot$. Pentru a demonstra aceast\u a propozi\c tie ne vom baza pe leg\u atura dintre logic\u a \c si teoria tipurilor prin \emph{izomorfisuml Curry-Howard} \citep{130367}. Vom nota prin $T$ propozi\c tia $\exists x . x : T$. Cu aceasta nota\c tie, obiectivul nostru devine acela de a demonstra propozi\c tia $\bot$. Consider\u am tipul de date
$$ \textbf{{Datatype}  } D := c : (D \to \bot) \to D $$
Regula de tip pentru analiza de cazuri pe $D$ corespunde urmatoarei reguli de deduc\c tie :
\begin{prooftree}
        \AxiomC{$D$}
        \AxiomC{$(D \Rightarrow \bot) \Rightarrow X $}
        \LeftLabel{$\forall X$}
        \BinaryInfC{$ X $}
\end{prooftree}
Care se traduce prin urmatoarea propozi\c tie
\begin{equation} \label{case_prop}
\forall X. D \wedge ((D \Rightarrow \bot) \Rightarrow X) \Rightarrow X
\end{equation}
Atunci urmatorul arbore de deduc\c tie demonstreaz\u a propozi\c tia $\bot$.
\begin{prooftree}
\AxiomC{$c : (D \to \bot) \to D$}
\UnaryInfC{$ (D \Rightarrow \bot) \Rightarrow D $ }
        \AxiomC{\eqref{case_prop}}
        \AxiomC{$(D \Rightarrow \bot) \Rightarrow (D \Rightarrow \bot)$}
        \BinaryInfC{$D \Rightarrow (D \Rightarrow \bot)$}
        \UnaryInfC{$D \Rightarrow \bot$}
    \BinaryInfC{$D$}
                    \AxiomC{$\vdots $}
                    \UnaryInfC{$D \Rightarrow \bot$}
        \BinaryInfC{$\bot$}
\end{prooftree}
Am folosit urmatoarea lem\u a clasic\u a ce apare \^ in contextul izomorfismului Curry-Howard.
\begin{lemma}\label{curry_howard_impl}
Propozi\c tia $ A \to B $ implic\u a propozi\c tia $A \Rightarrow B$. \qedhere
\end{lemma}
\end{proof}
In continuare vom da \c si demonstra\c tiile celor dou\u a leme.
\begin{proof}[Demonstra\c tia lemei \ref{inconsistence_to_nonterm}]
Presupunem c\u a System \frec are proprietatea de normalizare puternic\u a \c si c\u a exist\u a o expresie $e : \bot $. Cum $e$ este normalizabil\u a de tipul $\Pi X. X$ inseamn\u a c\u a $\exists e'. e \downarrow \Lambda X.e'$. Prin $e \downarrow v$ am notat faptul c\u a $e$ se reduce la forma normal\u a $v$.

Consider\u am acum expresia $e \app [ \textbf{{Bool}} ]$. Aceasta are tipul \textbf{{Bool}} \c si deci se reduce la o form\u a normal\u a care poate fi \textbf{{true}} sau \textbf{{false}}. Presupunem f\u ar\u a a restr\^ ange generalitatea c\u a aceasta este \textbf{{false}}. Deci
\begin{equation*}
(\Lambda X.e')\app [\textbf{{Bool}}] \downarrow \textbf{{false}} \Rightarrow [X \mapsto\textbf{{Bool}}] e' \downarrow \textbf{{false}}
\end{equation*}

Fie $e''$ astfel incat $e'\downarrow e''$ . Avem
\begin{equation*}
[X \mapsto\textbf{{Bool}}] e' \downarrow [X \mapsto\textbf{{Bool}}] e'' \Rightarrow
[X \mapsto\textbf{{Bool}}] e'' = \textbf{{false}}  \Rightarrow
e'' = \textbf{false}
\end{equation*}

Din proprietatea de conservare, ob\c tinem urm\u atoarele concluzii cu privire la tipul lui $e$.

\begin{equation*}
    e'' : \textbf{{Bool}} \Rightarrow
    e'  : \textbf{{Bool}} \Rightarrow
    e = \Lambda X.e' : \Pi X. \textbf{{Bool}} \neq \Pi X.X = \bot \qedhere
\end{equation*}
\end{proof}

\begin{proof}[Demonstratia lemei \ref{curry_howard_impl}]
Fie $e_{A \to B}$ astfel \^ inc\^ at $e_{A \to B} : A \to B$. Avem urmatoarele propozi\c tii
\[\exists e_A. e_A : A \Rightarrow \exists e_B . e_B = e_{A \to B}\app e_A \Rightarrow \exists e_B : B \qedhere\]
\end{proof}
Se poate da \c si o demonstra\c tie mai artificial\u a, dar constuctiv\u a pentru propozi\c tia \ref{case_prop} (\citep{1614481}).
\begin{proof}[Demonstra\c tia propozi\c tiei \ref{case_prop}]
\begin{align*}
p &:= \lambda x : D. \text{{case}}_{D\to \bot}\ x \text{ {of} } \{ c \Rightarrow \lambda y : D \to \bot . y \} &{ (D \Rightarrow (D \Rightarrow \bot))} \\
\omega_D &:= \lambda x : D. p\app x\app x &{ (D \Rightarrow \bot)}\\
\omega &:= p\app  (c\app \omega_D) \app (c\app \omega_D) &{ (\bot)}
\end{align*}
Expresia $\omega$ este cea cautat\u a. O secven\c t\u a infinit\u a de reduceri este urmatoarea
\[ \omega \to p\app (c\app \omega_D)\app (c\app \omega_D) \to (\lambda y : D \to \bot . y\app  \omega_D)\app (c\app \omega_D) \to \omega_D (c\app \omega_D) \to p\app (c\app \omega_D)\app (c\app \omega_D)\to \dots \qedhere \]
\end{proof}

Pe baza propozi\c tiei \ref{case_nonterm} putem trage concluzia c\u a trebuie restric\c tionate ambele construc\c tii introduse: \c si analiza de cazuri \c si recursivitatea textual\u a.

\section{Recursori}

O alternativ\u a ar fi creearea unor operatori de ordin \^ inalt care s\u a \^ inlocuiasca recursivitatea textual\u a \c si s\u a reprezinte cele trei tipuri de recursivitate indicate la sec\c tiunea \ref{scheme_rec}.

\begin{example}
Pentru a exprima recursivitatea structural\u a pe arbori se poate defini o func\c tie similar\u a func\c tiei \emph{\textbf{fold}} pe liste cu urmatorul comportament
\begin{align*}
\textbf{\emph{Datatype}  } \emph{\textbf{Tree}} &:= leaf : \emph{\textbf{Tree}} \ |\  node : \emph{\textbf{Tree}} \to \emph{\textbf{Nat}} \to \emph{\textbf{Tree}} \to \emph{\textbf{Tree}} \\
foldTree &:= \Lambda X .\lambda l:X. \lambda n : (X\to \emph{\textbf{Nat}} \to X \to X). \\
 &\text{\emph{letrec}}_{\emph{\textbf{Tree}} \to X}\ f =  \lambda t'. \text{\emph{case}}_X \ t' \text{ \emph{of} }\\
 &\qquad leaf \Rightarrow l\\
 &\qquad node \Rightarrow \lambda t_1 : \emph{\textbf{Tree}}.\lambda nr :\emph{\textbf{Nat}}. \lambda t_2 : \emph{\textbf{Tree}} . n\app (f\app t_1)\app nr\app (f\app t_2)
\end{align*}
\end{example}

Problema cu acest\u a alternativ\u a este c\u a folosirea recursorilor nu este deloc intuitiv\u a pentru programatori.
\begin{example}
Pentru a calcula dimensiunea unui arbore se foloseste urmatoarea expresie
$$foldTree \app [\emph{\textbf{Nat}}]\app z \app  (\lambda l : \emph{\textbf{Nat}}.\lambda  n : \emph{\textbf{Nat}}.\lambda r :\emph{\textbf{Nat}} . (add\app l\app r\app (s\app z))) $$
spre deosebire de varianta cu recursivitate textual\u a
\begin{align*}
\text{\emph{letrec}}_{\emph{\textbf{Tree}} \to \emph{\textbf{Nat}}} sz &= \lambda t  :\emph{\textbf{Tree}}.  \text{case}_{\emph{\textbf{Nat}}}\ t \text{ of } \\
&\qquad leaf \Rightarrow z \\
&\qquad node \Rightarrow \lambda l :\emph{\textbf{Tree}} .\lambda n: \emph{\textbf{Nat}}.\lambda r : \emph{\textbf{Tree}}. (add\app (sz\app l) \app (sz \app r)\app (s\app z))
\end{align*}
\end{example}

\begin{remark}
Aceast\u a construc\c tie poate fi generalizat\u a doar pentru o parte din tipurile de date recursive care se pot defini \^ in System \frec. De exemplu, pentru un tip cu un constructor de forma $ c : (T \to \emph{\textbf{Nat}}) \to T$, construc\c tia nu se extinde \^ in mod natural.
\end{remark}

\begin{remark}
O mul\c time de tipuri pentru care extinderea func\c tioneaz\u a sunt tipurile de date \emph{algebrice}.
\end{remark}

\section{Tipuri cu dimensiune}

Ideea din spatele System \fhat \citep{1614481} este de a extinde Sysyem \frec si de a folosi sistemul de tipuri pentru a exprima constr\^ angerile asupra analizei de cazuri \c si a recursivit\u a\c tii textuale.

Pentru a restric\c tiona recursivitatea textual\u a, se folose\c ste reprezentarea tipurilor de date din System \frec ca limita unui \c sir de subtipuri (numite \c si aproximari) astfel \^ incat, \^ in cazul schemelor clasice de recursivitate, apelurile recursive sunt f\u acute asupra unor argumente al c\u aror tip \emph{des\-cres\-te} (\^ in raport cu relatia de subtip). \^ In urma restric\c tiei puse pentru a rezolva problema analizei de cazuri, rela\c tia de subtip devine \fixme{well-founded} (defini\c tia \ref{well_founded}), lucru care asigura terminarea apelurilor recursive.

\begin{example}
Daca pornim de la interpretarea tipului de numere naturale $\textbf{\emph{Nat}} \equiv \{ z, s(z), \dots \}$ atunci interpretarile subtipurilor acestuia sunt
$$\textbf{\emph{Nat}}^p \equiv \{ z, s(z), \dots , s^p(z)\} \subset \textbf{\emph{Nat}}^{p+1} \subset \dots \subset \textbf{\emph{Nat}}^\infty \equiv \textbf{\emph{Nat}} $$
\end{example}

Vom reprezenta aceste aproximari prin \emph{tipuri cu dimensiune} cu urmatoarea sintax\u a:
\begin{gather*}
s := V_s \ |\ \hat{s}\ |\ \infty\\
\overline{T} := X \ |\ \overline{T} \to \overline{T}\ |\ \Pi X. \overline{T}\ |\ d^s\app \overline{T}
\end{gather*}

\begin{example}
Dimensiuni: $\infty$, $\hat{\iota}$. Tipuri cu dimensiune: $\textbf{\emph{Nat}}^\infty$, $\textbf{\emph{Nat}}^{\hat{\iota}}$. Din punct de vedere intuitiv, dac\u a $\textbf{\emph{Nat}}^{\iota}$ are interpretarea $\textbf{\emph{Nat}}^{p}$ atunci $\textbf{\emph{Nat}}^{\hat{\iota}}$ are interpretarea $\textbf{\emph{Nat}}^{p+1}$.
\end{example}
Problema analizei de cazuri este rezolvat\u a prin introducerea \emph{tipurilor inductive cu dimensiune}.
\begin{definition}
Un tip de date se nume\c ste \textbf{tip inductiv cu dimensiune} dac\u a to\c ti constructorii au tipul $c_k : \Pi X. \overline{\theta}_k \to d^{\hat{\iota}} X$, unde
\begin{itemize}\addtolength{\itemsep}{-0.5\baselineskip}
  \item Constructorul de tip de date $d$ apare in $\overline{\theta}_k$ doar pe pozi\c tii pozitive. O pozi\c tie este pozitiv\u a, dac\u a \^ in calea c\u atre r\u adacin\u a, \^ in arborele sintactic al tipului, se afl\u a \^ in st\^ anga constructorului $\to$ de numar par de ori.
  \item Orice apari\c tie a lui $d$ in $\overline{\theta}_k$ are dimensiunea $\iota$.
  \item Orice apari\c tie a altui constructor de tip de date $d' \neq d$ are dimensiuneea $\infty$.
\end{itemize}
\end{definition}\done\todo{poza cu diagrama adj}
Func\c tia $|.|:\overline{T} \to T$ este func\c tia care \c sterge adnot\u arile de dimensiune, de exemplu $|d^\iota X| = d X $.
Sintaxa termenilor si regulile de evaluare sunt identice cu cele ale System \frec.
\begin{example}
\^ In urmatoarea diagram\u a ADJ \citep{adj} este prezentat\u a structura tipului de liste de numere naturale. Discurile reprezint\u a tipuri, cu incluziunea de discuri reprezent\^ and rela\c tia de subtip. S\u agetile albastre reprezint\u a constructori, sursele s\u age\c tilor reprezint\u a tipuri din aritate, iar destina\c tiile reprezinta tipul rezultat. Fiecare constructor apare de mai multe ori pentru fiecare valoare a dimensiunii (polimorfism de dimensiune).
\begin{figure}
\begin{center}
\begin{tikzpicture}[node distance=4cm, auto,>=latex', thick]
    \tikzstyle{type} = [shape=circle, fill=white, draw, thin]
    \tikzstyle{constructor} = [draw, ->,color=blue!50!black ]

% Nat
    \node [type,minimum size=3.6cm,pin={[pin edge={<-,red}]90:$\textbf{\emph{Nat}}^{\infty}$}] at (0cm,0cm) (nat) {};
    \node [type,minimum size=2.4cm,label=above:$\vdots$,pin={[pin edge={<-,red}]50:$\textbf{\emph{Nat}}^{{\hat{\hat{\iota}}}}$}] at (0cm,0cm) (natssp) {};
    \node [type,minimum size=2.0cm,pin={[pin edge={<-,red}]30:$\textbf{\emph{Nat}}^{{{\hat{\iota}}}}$}] at (0cm,0cm) (natsp) {};
    \node [type,minimum size=1.6cm,pin={[pin edge={<-,red}]10:$\textbf{\emph{Nat}}^{{{{\iota}}}}$}] at (0cm,0cm) (natp) {};

    % succesor
    \path [constructor] (natp.south east)  node[above] {\scriptsize s} parabola[bend pos=0.5] bend +(0,-1cm) (natsp.south west) ;
    \path [constructor] (natsp.south east) node[above] {\scriptsize s} parabola[bend pos=0.5] bend +(0,-1cm) (natssp.south west);
    \path [constructor] {(nat.south)+ (0.5cm,0.08cm)}   node[above] {\scriptsize s} parabola[bend pos=0.5] bend +(0,-1cm) (nat.south);

    % zero
    \path [constructor] (0,0) circle (0) node[pin={[pin edge={<-,blue!50!black},pin distance=2cm]180:{\scriptsize z}}] (centru) {};

% Liste
    \tikzstyle{cons} = [shape=rectangle, thin, minimum width=0.03cm, minimum height=0.03cm]

    \node [cons,right of=nat,yshift=-1cm,label=below:{\scriptsize cons}] (consp) {};
    \node [cons,right of=nat,label=below:{\scriptsize cons}] (conssp) {};
    \node [cons,right of=nat,yshift=+1cm,label=below:{\scriptsize cons}] (consssp) {};

    \tikzstyle{type} = [shape=circle, fill=white, draw, thin, xshift=8cm]

    \node [type,minimum size=3.6cm,pin={[pin edge={<-,red}]90:$\textbf{\emph{List}}^{\infty}$}] at (0cm,0cm) (list) {};
    \node [type,minimum size=2.4cm,label=above:$\vdots$,pin={[pin edge={<-,red}]50:$\textbf{\emph{List}}^{{\hat{\hat{\iota}}}}$}] at (0cm,0cm) (listssp) {};
    \node [type,minimum size=2.0cm,pin={[pin edge={<-,red}]30:$\textbf{\emph{List}}^{{{\hat{\iota}}}}$}] at (0cm,0cm) (listsp) {};
    \node [type,minimum size=1.6cm,pin={[pin edge={<-,red}]10:$\textbf{\emph{List}}^{{{{\iota}}}}$}] at (0cm,0cm) (listp) {};

    % cons
    \path [constructor,-] (nat.east) edge[bend right]  (consp.center);
    \path [constructor,-] (nat.east) edge             (conssp.center);
    \path [constructor,-] (nat.east) edge[bend left] (consssp.center);

    \path [constructor] (consp.center)   parabola[bend pos=0.5] bend +(0,+0.5cm) (listsp.south west) ;
    \path [constructor] (conssp.center)  parabola[bend pos=0.5] bend +(0,+0.2cm) (listssp.west);
    \path [constructor] (consssp.center) parabola[bend pos=0.5] bend +(0,-0.5cm) (list.north west);

    \path [constructor,-] (listp.south west)     parabola[bend pos=0.5] bend +(0,-0.5cm) (consp.center) ;
    \path [constructor,-] (listsp.west)     parabola[bend pos=0.5] bend +(0,-0.2cm) (conssp.center);
    \path [constructor,-] (list.north west) parabola[bend pos=0.5] bend +(0,+0.5cm) (consssp.center);

    % nil
    \path [constructor] (list.center) circle (0) node[pin={[pin edge={<-,blue!50!black},pin distance=2cm]0:{\scriptsize nil}}] (centru) {};

\end{tikzpicture}

\end{center}
\caption{Diagrama ADJ pentru List \c si Nat}
\label{adj_list_nat}
\end{figure}
\end{example}


\subsection{Rela\c tia de subtip}
Rela\c tia de subtip \^ intre tipurile cu dimensiune este derivat\u a din rela\c tia de ordine \^ intre dimensiuni
\begin{align*}
s &\le s &  s \le s'  \wedge s' &\le s''  \Rightarrow s \le s''\\
s &\le \hat{s} & s &\le \infty
\end{align*}
Constructorul de tip $\to$ este contravariant \^ in primul argument \c si covariant \^ in al doilea. Intuitiv, o func\c tie care poate fi apelat\u a cu argumente de tip $T$, poate fi apelat\u a \c si cu argumente de tip $T' \sqsubseteq T$ \^ in timp ce rezultatul unei func\c tii $T_r$ poate fi considerat ca fiind de tip $T_r' \sqsupseteq T_r$. Regula pentru constructorii de tip defini\c ti de utilizator combin\u a regula de mo\c stenire a rela\c tiei de subtip din rela\c tia de ordine \^ intre dimensiuni cu regula de covarian\c t\u a a constructorilor.
\begin{multicols}{2}
\setlength\columnseprule{.4pt}
\begin{prooftree}
\AxiomC{$\overline{\tau}'\sqsubseteq \overline{\tau}$}
\AxiomC{$\overline{\sigma}\sqsubseteq \overline{\sigma}'$}
\BinaryInfC{$ \overline{\tau} \to\overline{\sigma} \sqsubseteq \overline{\tau}' \to\overline{\sigma}'$}
\end{prooftree}
\begin{prooftree}
\AxiomC{}
\UnaryInfC{$X \sqsubseteq X $}
\end{prooftree}
\begin{prooftree}
\AxiomC{$s \le s'$}
\AxiomC{$\overline{\tau}\sqsubseteq \overline{\sigma}$}
\BinaryInfC{$ d^s \overline{\tau} \sqsubseteq d^{s'} \overline{\sigma}$}
\end{prooftree}
\begin{prooftree}
\AxiomC{$\overline{\tau}\sqsubseteq \overline{\sigma}$}
\UnaryInfC{$ \Pi X.\overline{\tau} \sqsubseteq \Pi X.\overline{\sigma}$}
\end{prooftree}
\end{multicols}

\subsection{Reguli de tip}
\label{reguli_sysfhat}
Chiar dac\u a la nivel sintactic, adnot\u arile cu tipuri folosesc tipuri din System \frec, tipurile deduse automat \^ in System \fhat sunt tipuri cu dimensiune.
\begin{multicols}{2}
\setlength\columnseprule{.4pt}

\begin{prooftree}
\AxiomC{$x : \overline{\sigma} \in \overline{\Gamma} $}
\RightLabel{\scriptsize (T-VAR)}
\UnaryInfC{$\overline{\Gamma} \vdash x : \overline{\sigma} $}
\end{prooftree}

\begin{prooftree}
\AxiomC{$\overline{\Gamma}, x : \overline{\tau} \vdash e : \overline{\sigma} $}
\RightLabel{\scriptsize (T-ABS)}
\UnaryInfC{$\overline{\Gamma} \vdash \lambda x : |\overline{\tau}|. e : \overline{\tau} \to \overline{\sigma} $}
\end{prooftree}

\begin{prooftree}
\AxiomC{$\overline{\Gamma} \vdash e : \overline{\sigma} $}
\RightLabel{\scriptsize (T-TABS)}
\UnaryInfC{$\overline{\Gamma} \vdash \Lambda X . e : \Pi X. \overline{\sigma} $}
\end{prooftree}

\begin{prooftree}
\AxiomC{$c_k \in \mathcal{C}(d)$}
\RightLabel{\scriptsize (T-CONS)}
\UnaryInfC{$\overline{\Gamma} \vdash c_k : \Pi X. \overline{\theta}_k\to d^{\hat{\iota}} X $}
\end{prooftree}

\columnbreak

\begin{prooftree}
\AxiomC{$\overline{\Gamma} \vdash e : \overline{\sigma}$}
\AxiomC{$\overline{\sigma} \sqsubseteq \overline{\tau} $}
\RightLabel{\scriptsize (T-SUB)}
\BinaryInfC{$\overline{\Gamma} \vdash e : \overline{\tau} $}
\end{prooftree}

\begin{prooftree}
\AxiomC{$\overline{\Gamma} \vdash e : \overline{\tau} \to \overline{\sigma}$}
\AxiomC{$\overline{\Gamma} \vdash e' : \overline{\tau} $}
\RightLabel{\scriptsize (T-APP)}
\BinaryInfC{$\overline{\Gamma} \vdash e\app e' : \overline{\sigma} $}
\end{prooftree}

\begin{prooftree}
\AxiomC{$\overline{\Gamma} \vdash e : \Pi X.\overline{\sigma} $}
\RightLabel{\scriptsize (T-TAPP)}
\UnaryInfC{$\overline{\Gamma} \vdash e \app [|\overline{\tau}|] : [ X \mapsto \overline{\tau}] \overline{\sigma} $}
\end{prooftree}

\begin{prooftree}
\AxiomC{\scriptsize (T-LETREC)}
\noLine
\UnaryInfC{$\overline{\Gamma}, f : d^\iota\tau \to \overline{\theta} \vdash e : d^{\hat{\iota}}\tau \to [\iota \mapsto \hat{\iota}] \overline{ \theta}$}
\AxiomC{$\iota \text{\scriptsize pos} \overline{\theta} $}
\insertBetweenHyps{\hskip 2pt}
\BinaryInfC{$\overline{\Gamma} \vdash (\text{letrec}_{d|\overline{\tau}| \to |\overline{\theta}|} f = e) : d^s \overline{\tau} \to [\iota \mapsto s]\overline{\theta}$}
\end{prooftree}

\end{multicols}
\begin{prooftree}

\AxiomC{$\overline{\Gamma} \vdash e : d^{\hat{s}} \overline{\tau}$}
\AxiomC{$\overline{\Gamma} \vdash e_k : [X \mapsto \overline{\tau}, \iota \mapsto s]\overline{\theta}_k \to \overline{\sigma} $}
\AxiomC{$\overline{\Gamma} \vdash c_k : \Pi X. \overline{\theta}_k \to d^{\hat{\iota}} X$}
\RightLabel{\scriptsize (T-CASE)}
\TrinaryInfC{$\overline{\Gamma} \vdash \text{case}_{|\overline{\sigma}|}\ e \text{ of } \{c_1 \Rightarrow e_1\ |\ \dots \ |\ c_n \Rightarrow e_n\} : \overline{\sigma} $}
\end{prooftree}

\begin{remark}
Regula \textbf{ {\scriptsize (T-CONS)}} spune c\u a aplicarea unui constructor unor valori aflate \^ intr-o aproximatie a tipului de date, d\u a ca rezultat o valoare din aproximarea urmatoare.
\end{remark}

\begin{remark}
Regula \textbf{{\scriptsize (T-CASE)}} spune contrariul, \c si anume c\u a orice valoare care se afl\u a \^ intr-o aproximatie de dimensiune $\hat{s}$ poate fi supus\u a analizei de cazuri pentru a ob\c tine valori din aproxima\c tia $s$.
\end{remark}

\begin{remark}
Regula \textbf{{\scriptsize (T-LETREC)}} spune c\u a dac\u a o func\c tie definit\u a pe o aproxima\c tie poate fi extins\u a pe urmatoarea aproxima\c tie, atunci ea poate fi extins\u a pe intreg tipul de date, ca limit\u a a defini\c tiilor sale pe aproxim\u ari. Variabila $\iota$ trebuie s\u a nu apar\u a \^ in $\overline{\Gamma},\overline{\tau}$.
\end{remark}

\begin{remark}
\^ In procesul de deduc\c tie a tipului pentru o expresie, regula care trebuie aplicat\u a este unic determina\u ta de structura sintactic\u a a expresiei. Totu\c si procesul de stabilire a tipului bazat pe regulile de mai sus are o component\u a de nedeterminism deoarece trebuie s\u a \emph{ghiceasca} dimensiunile tipurilor. Astfel, spre deosebire de regulile System \frec, aceste reguli au doar rol teoretic; \^ in implementare se va folosi alt set de reguli echivalent.
\end{remark}

\subsection{Calita\c ti ca limbaj de programare}

Din punct de vedere al utiliz\u arii practice, System \fhat are aceeasi sintax\u a cu cea a System \frec ca\-re la r\^ andul s\u au se apropie de cea a unor limbaje de programare func\c tionala ca Haskell sau ML. O diferen\c t\u a major\u a fa\c t\u a de acastea este faptul c\u a programatorul trebuie s\u a adnoteze programele cu informa\c tii de tip. Totu\c si dimensiunile tipului se deduc automat.

Un lucru important \^ in cazul programelor despre care compilatorul nu poate decide dac\u a se termin\u a, este c\u a programatorul s\u a aiba \emph{apriori} intui\c tia dac\u a acestea vor trece de verificarea de tip \c si s\u a in\c teleag\u a \emph{aposteriori} din ce cauz\u a nu s-a putut demonstra terminarea programului. \^ In cazul System \fhat, ideea de demonstra\c tie este simpl\u a: dimensiunea argumentelor unei func\c tii textual recursive trebuie s\u a scad\u a. Mai mult, pentru tipuri de date comune precum liste sau arbori, dimensiunea tipului are o semnifica\c tie foarte natural\u a : lungimea listei sau ad\^ ancimea arborelui.

\done\todo{calitati ca limbaj de prog.}

 % Prezentarea System F hat

% Chapter 1

\chapter{Propriet\u a\c tile System F\^{}}
\label{Capitolul4}

% 10 pag
\section{Siguran\c ta}
\begin{remark}
\^ In aceast\u a sec\c tiune toate tipurile folosite, vor fi tipuri cu dimensiune. Pentru simplitate, atunci c\^ and nu este pericol de confuzie \^ intre tipurile System \frec \c si tipurile cu dimensiune, voi scrie $\sigma$ \^ in loc de $\overline{\sigma}$.
\end{remark}
\begin{proposition}
System \fhat are proprietatea de progres.
\end{proposition}
\begin{proof}[Demonstra\c tie]
Vom considera o expresie $e$ \^ in System \fhat astfel \^ inc\^ at $\exists \sigma,\Gamma.\: \Gamma \vdash e : \sigma$. \^ In func\c tie de structura lui $e$ avem mai multe cazuri.
\begin{enumerate*}
  \item ${\bf e = e_1\app e_2} $. Cum $e$ este validat\u a de sistemul de tipuri, \^ inseamna c\u a $\exists \sigma , \tau.\,e_1 : \sigma \to \tau$. \^ In cazul \^ in care $e_1$ nu este valoare, $\exists e_1'. e_1 \to e_1'$ si deci $e \to e_1'\app e_2$ propozi\c tia este demonstrat\u a. Altfel $e_1$ poate fi:
        \begin{enumerate*}
            \item Func\c tie lambda $\lambda x : |\sigma|.e_1'$ caz in care $e \to [x \mapsto e_2] e_1'$
            \item Func\c tie textual recursiv\u a $(\text{letrec}_{d\app|\tau|\to|\sigma|} f := g)$, caz \^ in care
                $$e \to [f \mapsto (\text{letrec}_{d\app|\tau|\to|\sigma|} f := g)]g \app e_2 $$
        \end{enumerate*}
  \item ${\bf e = e_1\app [|{\sigma}|]}$ . Daca $e_1$ este valoare, atunci singura form\u a pe care poate s\u a o aib\u a este  $\Lambda X.e_1'$ \c si deci $e\to [X \mapsto {\sigma}]e_1'$. Dac\u a $e_1$ nu este valoare, propozi\c tia se demonstreaz\u a analog cu cazul anterior.
  \item ${\bf e = \text{case}_{|\sigma|}\ e' \text{ of } \{ \vec{c} \Rightarrow \vec{e} \} }$. Din regula {\scriptsize (T-CASE)} avem c\u a $e': d\app \tau$. Deci singura valoare posibil\u a pentru $e'$ este de forma $c_k\app \tau\app t_1\app t_2\app \dots\app t_n$ caz \^ in care se poate aplica regula de reducere {\scriptsize (E-CASE)}.\qedhere
\end{enumerate*}
\end{proof}
Vom da \^ int\^ ai demonstra\c tia propriet\u a\c tii de conservare folosind o serie de leme al caror enun\c t se gase\c ste la sf\^ ar\c situl demonstra\c tiei. Majoritatea dintre ele sunt leme care apar \^ in orice de\-mon\-stra\-\c ti\-e de conservare precum leme de inversiune a regulilor de tip sau lema de conservare la substitu\c tie. Aceasta demonstra\c tie extinde demonstra\c tia din \citep{967408} pentru tipuri polimorfice.
\begin{proposition}
System \fhat are proprietatea de conservare.
\end{proposition}

\begin{proof}[Demonstra\c tie]
Fie $e_1$ o expresie astfel \^ inc\^ at $\Gamma \vdash e_1 : \sigma $ \c si $e_1 \to e_2$. Trebuie s\u a demonstr\u am c\u a $\Gamma \vdash e_2:\sigma$. Vom demonstra prin induc\c tie structural\u a pe arborele de deduc\c tie al propozi\c tiei $\Gamma \vdash e_1 : \sigma $. \^ In func\c tie de ultima regul\u a de deduc\c tie aplicat\u a, avem mai multe cazuri, dintre care le voi trata doar pe cele interesante:
\begin{description}
  \item[{\scriptsize (T-CASE)}] Atunci $e_1 = \text{case}_{|\sigma|}\ e_1' \text{ of } \{ \vec{c}\Rightarrow \vec{e} \} \text{ si } \Gamma \vdash e_1' : d^{\hat{s}}\app\tau $ .
        \begin{enumerate*}
        \item Dac\u a ultima regul\u a de evaluare aplicat\u a \^in $e_1 \to e_2$ este regula {\scriptsize (CTX-CASE)}, atunci $\exists e_2'. e_1' \to e_2' \text{ si } e_2 = \text{case}_{\sigma}\ e_2' \text{ of } \{ \vec{c}\Rightarrow \vec{e} \}$. Din ipoteza de induc\c tie $\Gamma \vdash e_1' : d^{\hat{s}} \app \tau \Rightarrow \Gamma \vdash e_2': d^{\hat{s}}\app\tau$. Din lema \ref{inversion} pentru $\Gamma \vdash e_2 : \sigma $ \c si regula {\scriptsize (T-CASE)} avem ca $\Gamma \vdash e_2 : \sigma $.
        \item Dac\u a ultima regula de evaluare aplicat\u a este {\scriptsize (E-CASE)} atunci $e_1' = c_k\app[|\tau|]\app t$ si deci $e_2 = e_k \app t$. Pentru simplitate am considerat cazul \^ in care constructorul este unar. Conform regulii de tip {\scriptsize (T-CASE)} avem c\u a $\exists s$ astfel \^ inc\^ at:
                 \begin{equation}
                 \Gamma \vdash e'_1 : d^{\hat{s}} \tau \ \ \wedge \ \
                 \Gamma \vdash e_k : [X \mapsto \tau, \iota \mapsto s] \theta_k \to \sigma \ \ \wedge \ \
                 \Gamma \vdash c_k : \Pi X. \theta_k \to d^{\hat{\iota}}\app X
                 \end{equation}
            Acum aplic\u am lema \ref{inversion} pentru $\Gamma \vdash e_1' = c_k\app[|\tau|]\app t : d^{\hat{s}}\app\tau$ avem c\u a
            \begin{equation} \label{eq1}
            \exists \gamma,\sigma.
                \Gamma \vdash c_k\app [|\tau|] : \gamma \to \sigma  \quad \wedge \quad
                \Gamma \vdash t : \gamma \quad \wedge \quad
                \sigma \sqsubseteq d^{\hat{s}}\app \tau
            \end{equation}
            Aplic\^ and \^ inca odata lema \ref{inversion} pentru $\Gamma \vdash c_k\app [|\tau|] : \gamma \to \sigma$ ob\c tinem
            \begin{equation}\label{eq2}
                \Gamma \vdash c_k\app : \Pi X. \gamma' \quad \wedge \quad
                [X\mapsto \tau] \gamma' \sqsubseteq \gamma \to \sigma
            \end{equation}
            Aplic\^ and \^ inca odata lema \ref{inversion} pentru $\Gamma \vdash c_k\app : \Pi X. \gamma' $ ob\c tinem
            \begin{equation}  \label{eq3}
                \exists r.\gamma' \equiv \gamma'' \to \sigma'' \quad \wedge \quad
                \gamma'' \sqsubseteq [\iota \mapsto r] \theta_k \quad \wedge \quad
                d^{\hat{r}}\app X \sqsubseteq \sigma''
            \end{equation}
            Din rela\c tiile \eqref{eq2} \c si \eqref{eq3} \c si lema \ref{subst_sub} ob\c tinem
            \begin{equation} \label{eq4}
                [X \mapsto \tau] (\gamma'' \to \sigma'') \sqsubseteq \gamma \to \sigma \Rightarrow \gamma \sqsubseteq [X \mapsto \tau] \gamma'' \sqsubseteq [X \mapsto \tau][\iota \mapsto r] \theta_k
            \end{equation}
            Din rela\c tiile \eqref{eq1},\eqref{eq2} \c si \eqref{eq3} \c si lema \ref{stage_inversion} ob\c tinem
            \begin{equation} \label{eq5}
            \begin{split}
                [X \mapsto \tau] (\gamma'' \to \sigma'') \sqsubseteq \gamma \to \sigma
                    &\Rightarrow [X \mapsto \tau] d^{\hat{r}} X \sqsubseteq [X \mapsto \tau] \sigma'' \sqsubseteq \sigma \sqsubseteq d^{\hat{s}} \tau \\
                    &\Rightarrow d^{\hat{r}} \tau \sqsubseteq d^{\hat{s}} \tau\\
                    &\Rightarrow r \le s
            \end{split}
            \end{equation}
            Din \eqref{eq4} \c si \eqref{eq5}, lema \ref{stage_pos_subst} \c si din faptul c\u a tipul de date $d$ fiind inductiv, $d^{\iota}$ \c si deci $\iota$ apare \^ in $\theta_k$ pe pozi\c tii pozitive:
            \begin{equation}\label{eq6}
                \gamma \sqsubseteq [X \mapsto \tau, \iota \mapsto s] \theta_k
            \end{equation}
            Conform \eqref{eq6},\eqref{eq1} \c si {\scriptsize (T-APP)} : $\Gamma \vdash e_2 = e_k \app t : \sigma$.
        \end{enumerate*}
  \item[{\scriptsize (T-APP)}] Atunci $e_1 = t_1 \app t_2$ \c si $\Gamma \vdash t_1 : \tau_1 \to \tau_2 $, $\Gamma \vdash t_2 : \tau_1 $, $\Gamma \vdash e_1 : \tau_2 $.
        \begin{enumerate*}
        \item Dac\u a ultima regul\u a de evaluare aplicat\u a este {\scriptsize (CTX-APP)}, atunci propozi\c tia se demonstreaz\u a folosind ipoteza de induc\c tie ca \c si \^ in cazul anterior.
        \item Dac\u a ultima regul\u a de evaluare aplicat\u a este {\scriptsize (E-APP)}, atunci $t_1 = \lambda x : |\tau_1'| . e$ \c si deci $e_2 = [x \mapsto t_2] e$. Folosind lema \ref{inversion} pentru $\Gamma \vdash t_1 : \tau_1 \to \tau_2$ o\c btinem:
            \begin{equation}\label{eq7}
                \Gamma, x:\tau_1' \vdash e : \tau_2' \quad\wedge\quad \tau_1 \sqsubseteq \tau_1' \quad\wedge\quad \tau_2' \sqsubseteq \tau_2
            \end{equation}
            Din \eqref{eq7} conform regulii {\scriptsize (T-SUB)} avem c\u a $\Gamma \vdash t_2 : \tau_1'$. De aici, conform lemei de substitu\c tie \ref{subst_var} si \eqref{eq7} ob\c tinem:
            \begin{equation}
                \Gamma \vdash [x \mapsto t_2] e : \tau_2' \sqsubseteq \tau_2
            \end{equation}
        \item Dac\u a ultima regul\u a de evalure este  {\scriptsize (E-LETREC)}, avem $t_1 = (\text{letrec}_{d\app|\tau| \to |\sigma|}f := g)$ si $t_2 = c\app [|\tau|] \app e$ si deci $e_2 = ([f \mapsto (\text{letrec}_{d\app|\tau| \to |\sigma|}f := g)]g)\app(c\app [|\tau|] \app e)$. Aplic\^ and lema \ref{inversion} pentru $\Gamma \vdash t_1 : \tau_1 \to \tau_2$ ob\c tinem:
            \begin{gather}
                \label{eq8} \forall \iota. \:\Gamma,f : d^{{\iota}} \tau \to \sigma \vdash g : [\iota \mapsto \hat{\iota}] (d^{{\iota}}\app \tau \to \gamma) \\
                \label{eq9} \exists s. [\iota \mapsto s] (d^{{\iota}} \tau \to \sigma) \sqsubseteq \tau_1 \to \tau_2 \\
                \label{eq10} \iota \text{ {pos} } \sigma \text{ {si} } \iota \text{ {nu apare in} } \Gamma , \tau
            \end{gather}
            Din \eqref{eq9}, \eqref{eq10} \c si regula de subtip pentru func\c tii, avem $\tau_1 \sqsubseteq d^{s} \app \tau$ si $[\iota \mapsto s]\sigma \sqsubseteq \tau_2$ \c si folosind  lema \ref{sub_inversion}
            \begin{equation}\label{eq11}
              \tau_1 \equiv d^p\app \tau' \ \wedge \ p \le s \ \wedge\ \tau' \sqsubseteq \tau.
            \end{equation}
            Folosind \eqref{eq8}, \eqref{eq10} \c si regula {\scriptsize (T-LETREC)} avem c\u a
            \begin{equation}\label{eq12}
                \forall q. \Gamma \vdash (\text{letrec}_{d\app|\tau| \to |\sigma|}f := g) : [\iota \mapsto q] (d^{{\iota}}\app\tau \to \sigma)
            \end{equation}
            \c Tin\^ and cont de \eqref{eq10} alegem pe $q \equiv \iota$. Folosind lema de substitu\c tie \ref{subst_var}, \eqref{eq8} \c si \eqref{eq12} putem concluziona c\u a:
            \begin{equation}\label{eq13}
                \Gamma \vdash ([f \mapsto (\text{letrec}_{d|\tau| \to |\sigma|}f := g)]g) : [\iota \mapsto \hat {\iota}] (d^{\iota} \app \tau \to \sigma)
            \end{equation}
            Ra\c tion\^ and ca \^ in cazul {\scriptsize (T-CASE)} pentru $\Gamma \vdash t_2 = c\app [|\tau|] \app e : d^p \app\tau'$ ob\c tinem c\u a \done\todo{refacut argumentul}
            \begin{equation}\label{eq14}
                \exists r.  d^{\hat{r}} \tau \sqsubseteq d^p \app \tau'
            \end{equation}
            Conform cu lema \ref{sub_inversion} avem $p \ge \hat{r}$, deci exist\u a dou\u a posibilit\u a\c ti
            \begin{enumerate*}
                \item $p=j^n$ cu $n \ge 1$ . Conform lemei \done\todo{lema 3.8} \ref{stage_subst} putem aplica substitu\c tia $\iota \mapsto j^{n-1}$ relatiei \eqref{eq13} \c si \c tinand cont de \eqref{eq10} ob\c tinem
                \begin{equation}
                    \Gamma \vdash ([f \mapsto (\text{letrec}_{d|\tau| \to |\sigma|}f := g)]g) : d^{j^n} \app \tau \to [\iota \mapsto j^n] \sigma
                \end{equation}
                Cum din rela\c tiile \eqref{eq14} \c si \eqref{eq11} avem c\u a $\Gamma \vdash (c [|\tau|] e) : d^p\app \tau' \sqsubseteq d^p\app \tau$, folosind {\scriptsize (T-APP)} ob\c tinem:
                \begin{equation} \label{eq15}
                    \Gamma \vdash([f \mapsto (\text{letrec}_{d|\tau| \to |\sigma|}f := g)]g) \app (c [|\tau|] e) : [\iota \mapsto j^n] \sigma
                \end{equation}
                Dar din rela\c tia \eqref{eq10}, $\iota \text{ \emph{pos} } \sigma$, din \eqref{eq11} $j^n = p \le s$ \c si conform lemei \ref{stage_pos_subst} $[\iota \mapsto j^n] \sigma \sqsubseteq [\iota \mapsto s] \sigma$, deci conform \eqref{eq15} concluzion\u am c\u a
                \begin{equation}
                    \Gamma \vdash e_2 : [\iota \mapsto s] \sigma \equiv \tau_2
                \end{equation}
                \item $p = \infty^m $ cu $m \ge 0$. Folosind {\scriptsize (T-SUB)} deducem $\Gamma \vdash (c [|\tau|] e) : d^{\infty^{m+1}}\app \tau'$ \c si similar cu punctul precedent $\Gamma \vdash e_2 : [\iota \mapsto \infty^{m+1}]\sigma$. Avem $s \ge \infty^{m}$ deci
                \begin{equation}
                    \forall r.r \le \infty \Rightarrow \hat{\infty} \le \infty \Rightarrow \infty^{m+1} \le \infty^{m} \le s
                \end{equation}
                \^ In concluzie , $\Gamma \vdash e_2 : [\iota \mapsto s]\sigma\equiv \tau_2$. \qedhere
            \end{enumerate*}
        \end{enumerate*}
\end{description}
\end{proof}

\done\todo{de facut lemele sa arata ca lumea}
Vom prezenta \^ in continuare lemele folosite \^ in demonstra\c tie \citep{967408}. Majoritatea nu au fost invocate explicit, pentru a nu \^ incarca demonstra\c tia. Demonstra\c tiile implic\u a un procedeu de induc\c tie si constau \^ in manipul\u ari de ecua\c tii de rutin\u a.

\begin{lemma}\label{inversion}
Aceasta lem\u a furnizeaz\u a reciproce pentru regulile de deduc\c tie ale tipurilor.
\begin{enumerate}
\item $\Gamma \vdash x : \sigma \Rightarrow (x : \tau) \in \Gamma \textbf{ si } \tau \sqsubseteq \sigma$
\item $\Gamma \vdash e_1\app e_2 : \sigma \Rightarrow \Gamma \vdash e_1 : \tau_1 \to \tau_2 \textbf{ si }
                                                \Gamma \vdash e_2 : \tau_1 \textbf{ si }
                                                \tau_2 \sqsubseteq \sigma $
\item $\Gamma \vdash e \app[|\tau|] : \sigma  \Rightarrow \Gamma \vdash e : \Pi X . \tau_1 \textbf{ si }
                                                [X \mapsto \tau ] \tau_1 \sqsubseteq \sigma $
\item $\Gamma \vdash c : \sigma  \Rightarrow \exists s. \sigma \equiv \Pi X.\tau_1 \to \tau_2 \textbf{ si }
                                        \tau_1 \sqsubseteq [\iota \mapsto s]\theta_k \textbf{ si }
                                        d^{\hat{s}}\app X \sqsubseteq \tau_2$
\item $\Gamma \vdash \lambda x : {|\tau|}. e :\sigma \Rightarrow \sigma \equiv \tau_1 \to \tau_2  \textbf{ si }
                                                        \Gamma , x:\tau \vdash e : \tau_2' \textbf{ si }
                                                        \tau_1 \sqsubseteq \tau \textbf{ si }
                                                        \tau_2' \sqsubseteq \tau_2$
\item $\Gamma \vdash \text{\emph{case}}_{|\sigma|}\ e \text{ \emph{of} } \{ \vec{c} \Rightarrow \vec{e} \} : \sigma \Rightarrow
                        \Gamma \vdash e : d^{\hat{s}}\tau_1,
                        \Gamma \vdash e_k : [X \mapsto \tau_1, i \mapsto s] \theta_k \to \tau_2 \textbf{si }
                        \tau_2 \sqsubseteq \sigma$
\item $ \Gamma  \vdash (\text{\emph{letrec}}_{d |\tau| \to |\gamma|} f:=g) : \sigma \Rightarrow
        \Gamma,f : d^{{\iota}} \tau \to \gamma \vdash g : [\iota \mapsto \hat{\iota}] (d^{{\iota}}\app \tau \to \gamma) \textbf{ si }$
        $\qquad \iota \text{ \emph{pos} } \gamma \textbf{ si } \iota \notin \Gamma , \tau \textbf{ si } \exists s.[\iota \mapsto s] (d^{{\iota}} \tau \to \gamma) \sqsubseteq \sigma $
\end{enumerate}
\end{lemma}

\begin{lemma}\label{sub_inversion}
Lema de inversiune a reguli de subtip pentru tipuri de date.
$$ \theta \sqsubseteq d^s\app \tau \Rightarrow \theta \equiv d^r \app \sigma \textbf{ si } r \le s \textbf{ si } \sigma \sqsubseteq \tau $$
\end{lemma}

\begin{lemma}
Rela\c tia de subtip este invariant\u a la substitu\c tia de dimensinui.
$$ \tau_1 \sqsubseteq \tau_2 \Rightarrow [\iota \mapsto s]\tau_1 \sqsubseteq [\iota \mapsto s]\tau_2$$
\end{lemma}

\begin{lemma}\label{subst_sub}
Rela\c tia de subtip este invarianta la substitu\c tia de variabile de tip.
$$\tau \sqsubseteq \sigma \Rightarrow [X \mapsto \gamma]\tau \sqsubseteq [X \mapsto \gamma] \sigma $$
\end{lemma}

\begin{lemma}\label{subst_var}
Daca intr-o expresie $e$, o variabila liber\u a $x$ este \^ inlocuit\u a de o expresie de acela\c si tip $e'$, atunci tipul expresiei $e$ se conserv\u a.
$$\Gamma , x : \tau \vdash e : \sigma \wedge \Gamma \vdash e' : \tau \Rightarrow \Gamma \vdash [x \mapsto e']e : \sigma $$
\end{lemma}

\begin{lemma}\label{stage_inversion}
Monotonia func\c tiei $d^{\hat{x}} : s \to \overline{T}$.
$$ d^{\hat{r}} \tau \sqsubseteq d^{\hat{s}} \tau \Rightarrow r \le s$$
\end{lemma}

\begin{lemma} \label{stage_pos_subst}
Covarian\c ta tipurilor \^ in raport cu pozi\c tiile pozitive \c si contravarian\c ta \^ in raport cu cele negative.
\begin{gather*}
 \iota \text{ pos } \theta \wedge s \le r \Rightarrow [\iota \mapsto s] \theta \sqsubseteq [\iota \mapsto r] \theta \\
 \iota \text{ neg } \theta \wedge s \le r \Rightarrow [\iota \mapsto r] \theta \sqsubseteq [\iota \mapsto s] \theta
\end{gather*}
\end{lemma}

\begin{lemma}\label{stage_subst}
Polimorfismul variabilelor de dimensiune libere.
$$\Gamma \vdash e : \sigma \wedge  \iota \notin \Gamma \Rightarrow \forall s. \Gamma \vdash e : [\iota \mapsto s] \sigma$$
\end{lemma}

\begin{corollary}
System \fhat are proprietatea de siguranta.
\end{corollary}

\subsection{Adnot\u ari explicite}

\^ In cazul \^ in care termenii ar fi adnota\c ti explicit cu dimensiuni, proprietatea de conservare \c si deci cea de siguan\c ta nu ar mai avea loc.
\begin{example}[Barthe et al. \citep{DBLP:conf/tlca/BartheGP05}]
Consider\u am expresia
$$ pred = \text{\emph{letrec}}_{\textbf{\emph{Nat}}^{\iota}\to\textbf{\emph{Nat}}^{\iota}} f := \lambda x: \textbf{\emph{Nat}}^{\hat{\iota}}.\text{\emph{case}}_{\textbf{\emph{Nat}}^{\iota}}\ x \text{ \emph{of} } \{ z \Rightarrow z\ |\ s \Rightarrow \lambda z : \textbf{\emph{Nat}}^{\iota}.z \}$$
Conform regulii {\scriptsize (T-LETREC)} pentru orice dimensiune $s$, $pred : \textbf{\emph{Nat}}^s \to \textbf{\emph{Nat}}^s$, deci putem deduce c\u a ra\c tionamentul $y : \textbf{\emph{Nat}}^j \vdash pred \app (s \app y) : \textbf{\emph{Nat}}^j$ este valid. Totodat\u a, \^ in urma reducerii ob\c tinem $y : \textbf{\emph{Nat}}^j \nvdash (\lambda z : \textbf{\emph{Nat}}^i.z)\app y : \textbf{\emph{Nat}}^j$.
\end{example}

Absen\c ta adnot\u arilor cu dimensiuni exprim\u a o cuantificare universala a dimensiunilor peste toata expresia de tip. De exemplu, \^ in cazul func\c tiei $\lambda x : \textbf{Nat} . x$ putem afirma c\u a
$$\forall s . \emptyset \vdash (\lambda x : \textbf{Nat} . x) :  \textbf{Nat}^s \to  \textbf{Nat}^s \Leftrightarrow \emptyset \vdash (\lambda x : \textbf{Nat} . x) :  \forall s. \textbf{Nat}^s \to  \textbf{Nat}^s$$
Practic acest polimorfism la nivel de dimensiune ne spune c\u a func\c tia identitate definita pe numere naturale func\c tioneaza pe toate aproxim\u arile mul\c timii de numere naturale. Adnotarea cu dimensiuni ar defini o func\c tie identitate doar pe o anumit\u a aproximare.

\^ In cazul recursivit\u a\c tii textuale, func\c tia ob\c tinut\u a este prin natura sa definit\u a pe toate aproxim\u arile unui tip de date. E\c secul System \fhat de a avea proprietatea de conservare \^ in prezen\c ta adnot\u arilor cu dimensiuni se datoreaz\u a faptului c\u a o func\c tie lambda monomorifca este o subexpresie a unei defini\c tii textual recursive polimorfice.

\section{Normalizarea puternic\u a}

\^ In aceasta sec\c tiune vom demonstra propietatea de normalizare puternic\u a pentru orice expresie din Systm \fhat \^ in raport cu o rela\c tie de reducere mai generala dec\^ at cea folosit\u a la definirea semnaticii opera\c tionale a System \fhat. Demonstra\c tia prezentat\u a este extinderea demonstratiei din \citep{967408} pentru tipuri polimorfice dupa ideile prezentate \^ in \citep{1614481}.

\begin{theorem}
Secven\c ta de reduceri conform rela\c tiei $\to_{\beta\iota\mu}$ a unei expresii din System \fhat este finita. Unde $e_1 \to_{\beta\iota\mu} e_2$ dac\u a $e_1$ se reduce la $e_2$ prin aplicarea regulilor de reducere {\scriptsize (E-APP),(E-TAPP),(E-CASE)} \c si {\scriptsize (E-LETREC)} pentru orice subexresie (nu doar \^ in cele determinate de regulile {\scriptsize (CTX-APP),(CTX-TAPP),(CTX-CASE)} \c si {\scriptsize  (CTX-LETREC)}).
\end{theorem}

Formal, mul\c timea expresiilor cu proprietatea de normalizare puternic\u a (strongly-normalizing sau $\in \textbf{SN}$) este definit\u a ca cea mai mic\u a mul\c time cu urm\ atoarea proprietate:
\begin{equation}
    \forall e . \quad (\forall e'. \quad e \to_{\beta\iota\mu} e' \Rightarrow e' \in {\textbf{SN}} ) \Rightarrow  e \in {\textbf{SN}}
\end{equation}
Vom spune c\u a orice expresie $e \in {\bf SN}$ este \emph{reductibil\u a}.

Vom \^ incerca s\u a asociem fiecarui tip al System \fhat o mul\c time (\emph{mul\c time saturata} \citep{967408}) de expresii reductibile care au acel tip. Apoi vom demonstra c\u a orice termen apartin\^ and acelui tip face parte din mul\c timea asociata. Acest lucru furnizeaz\u a demonstra\c tia teoremei de normalizare puternic\u a.

\subsection{Mul\c timi saturate}

Pentru \^ inceput vom introduce ni\c ste defini\c tii:
\begin{definition}
Un {\bf context slab} reprezint\u a o loca\c tie intr-o expresie din System \fhat unde ar putea fi f\u acuta o reducere dupa regulile {\scriptsize (CTX-APP),(CTX-TAPP),(CTX-CASE)} si {\scriptsize  (CTX-LETREC)}. Contextele slabe sunt generate de gramatica:
\begin{center}
E[] :=  []   $\:|\:$   E[] e   $\:|\:$   E[] [$|\tau|$]   $\:|\:$   $\text{\emph case}_\tau$ E[] of \{ $\vec{c} \Rightarrow \vec{e}$ \}
\end{center}
Un context se nume\c ste {\bf context slab de baz\u a} dac\u a toate subexpresiile contextului sunt reductibile.
\end{definition}

\begin{definition}
O expresie se nume\c ste \textbf{expresie de baz\u a} dac\u a este de forma E[x] unde E este un context slab de ba\u za, \c si x este o variabil\u a. Mul\c timea expresiilor de baz\u a se noteaz\u a cu \textbf{B}.
\end{definition}

Se poate demonstra prin induc\c tie dup\u a structura contextului, urmatoarea lem\u a:
\begin{lemma}
Orice expresie de baz\u a este reductibil\u a.
\end{lemma}

\begin{definition}
Rela\c tia de \textbf{reducere slab\u a} este rela\c tia de reducere \^ in care regulile {\scriptsize (E-APP),(E-TAPP),(E-CASE)} \c si {\scriptsize (E-LETREC)} se aplic\u a doar pentru subexpresii ce corespund unui context slab. Ea se noteaz\u a cu $\to_k$.
\end{definition}

\begin{definition}
O mul\c time de expresii $X$ System \fhat se nume\c ste \textbf{mul\c time saturat\u a} dac\u a:
\begin{enumerate*}
\item Orice expresie din $X$ este reductibil\u a.
\item Orice expresie de baz\u a face parte din $X$.
\item Daca o expresie $e$ este reductibil\u a \c si $e \to_k e' \in X$ atunci $e \in X$.
\end{enumerate*}
Mul\c timea tuturor multimilor saturate va fi notat\u a cu \textbf{SAT}. Pentru orice mul\c time $X$ de expresii, se noteaz\u a cu $\overline{X} = \{ e \in SN \ |\ \exists e' \in Base \cup X. \quad e \to_k^* e'\}$ \^ inchiderea multimii $X$.
\end{definition}

Pentru fiecare tip din System \fhat va fi construit\u a inductiv o mul\c time saturat\u a. Astfel va trebui s\u a folosim unele propriet\u a\c ti de inchidere a mul\c timii \emph{SAT} relative la operatorii folosi\c ti \^ in acest proces de construc\c tie.

\begin{lemma}
\begin{enumerate*}
\item Pentru orice mul\c time de expresii $X \in SN$ avem $\overline{X} = \bigcap_{Y \supset X, Y \in SAT}Y $.
\item \^ Inchiderea unei mul\c timi este saturat\u a : $\overline{X} \in SAT$.
\item \^ Inchiderea comut\u a cu reuniunea $\overline{X_1 \cup \dots \cup X_n} = \overline{X_1} \cup \dots \cup \overline{X_n}$.
\item Dac\u a $X_i$ sunt mul\c timi saturate pentru orice $i \in I$ atunci $\bigcup_{i \in I} X_i$ este mul\c time saturat\u a.
\end{enumerate*}
\end{lemma}

\done\todo{NU ESTE}
\begin{lemma}
Dac\u a $X,Y$ sunt mul\c timi saturate, atunci $X \to Y = \{ e \: |\: \forall e' \in X.\quad e\app e' \in Y \} $ este mul\c time saturat\u a.
\end{lemma}
\begin{comment}
\begin{proof}[Demonstratie]
Se observa ca orice expresie $e \in X \to Y$ este reductibil\u a:
\begin{equation}
\forall e'.\quad e\app e' \in Y \Rightarrow e\app e' \in SN \Rightarrow e \in SN
\end{equation}
In continuare vom demonstra ca $B \subset X \to Y$. Fie $e \in B$ si $e' \in X $. Din faptul ca $e' \in X \subset SN$, rezulta ca $[]\app e'$ este un context slab, deci $e\app e' \in B \subset Y$. Deci $e \in X \to Y$.

Fie $e \in SN$ cu $e \to_k e'$ si $e' \in X \to Y$. Consideram o expresie $t \in X \subset SN$ arbitrara . Cum contextul $ []\app t $ este context slab, avem ca $e\app t \to_k e'\app t$. Cum $t\in SN$ avem $e'\app t \in Y \subset SN$. Din lema urmatoare \todo{e t in SN} rezulta ca $e \app t \in SN$ si cum $e \app t \to_k e' \app t \in Y$ inseamna ca $Y$ indeplineste si cea de-a treia conditie pentru multimi saturate.
\end{proof}
\end{comment}

\begin{lemma}
Fie $t \in SAT$ un tip System \fhat, atunci mul\c timea $\{ e \: |\: \forall \sigma \in T .\quad e\app \sigma \in t\}$ este mul\c time saturat\u a.
\end{lemma}

\subsection{Interpretarea limbajului}

Vom construi \^ int\^ ai o interpretare pentru dimensiuni, asociind fiecarei dimensiuni un ordinal num\u arabil.

\begin{definition}
Mul\c timea de ordinali num\u arabili $Ord$ este cea mai mica mul\c time nenumar\u abila cu o rela\c tie de ordine totala \fixme{well-founded} pentru care $I_x := \{ y \: |\: y < x \}$ este num\u arabila pentru orice $x$.
\end{definition}

\begin{remark}
Mul\c timea ordinalilor este izomorfa cu "mul\c timea" \^ in care fiecare element este chiar mul\c timea elementelor mai mici decat el. Aceasta mul\c time este total ordonat\u a, iar totalitatea elementelor care reprezint\u a mul\c timi num\u arabile formeaz\u a mul\c timea ordinalilor num\u arabili. Not\u am cu $\Omega$ primul ordinal nenum\u arabil
\end{remark}

Plec\^ and de la o func\c tie care asociaz\u a cate un ordinal num\u arabil fiecarei variabie de dimensiune $v_s : S \to Ord$ definim o interpretare a dimensiunilor astfel:
\begin{align*}
&\langle x \rangle_{v_s} = v_s(x) \\
&\langle \hat{s} \rangle_{v_s} =  succ(\langle s \rangle_{v_s} ) = \inf \{ I_{\langle s \rangle_{v_s} } \cup \{\langle s\rangle_{v_s} \} \} \\
&\langle \hat{\infty} \rangle_{v_s} = \langle \infty \rangle_{v_s} = \Omega
\end{align*}

Plec\^ and de la o interpretare a variabilelor de dimensiune $v_s : S \to Ord$, \c si de la o interpretare a variabilelor de tip $v_T : T \to SAT$ se poate definim interpretarea tipurilor astfel:
\begin{align*}
&\langle X \rangle_{v_s,v_T} = v_T(X) \\
&\langle \tau \to \sigma \rangle_{v_s,v_T} =  \langle \tau \rangle_{v_s,v_T} \to \langle \sigma \rangle_{v_s,v_T} \\
&\langle \Pi X. \tau \rangle_{v_s,v_T} = \{ e \: |\: \forall \sigma \in T .\quad e\app \sigma \in \langle \tau \rangle_{v_s,[X\mapsto \langle \sigma \rangle_{v_s,v_T} ]v_T}\} \\
&\langle d^{s}\app\tau \rangle_{v_s,v_T} = D_d( \tau , \langle s \rangle_{v_s}) = D_d^{def} ( \tau, \langle \tau \rangle_{v_s,v_T}, \langle s \rangle_{v_s})
\end{align*}
unde $D_d(X,s)$ este definita prin induc\c tie transfinit\u a astfel:
\begin{align*}
& D_d^{def}( \tau, S , 0) = \overline{\emptyset} \\
& D_d^{def}( \tau, S, succ(x)) = \overline{\bigcup \{c \app \tau\app e \: | \: e \in \langle \theta \rangle_{[ \iota \mapsto x]v_s,[X \mapsto S] v_T}, c : \theta\to d^{\hat{\iota}} \app X \}} \\
& D_d^{def}( \tau, S, \lambda) = \bigcup\{D_d^{def}(\tau, X,s) \: | \: s < \lambda \} \text{ daca $\lambda$ nu are predecesor }
\end{align*}

\begin{remark}
Definirea printr-o recuren\c t\u a transfinit\u a a fost posibil\u a datorit\u a faptului c\u a rela\c tia de ordine \^ intre ordinali este \fixme{well-founded}. Datorit\u a faptului c\u a \^ in defini\c tia $D_d^{def}( \tau, S, succ(x))$ apare interpretarea lui $\theta$ care la randul s\u au poate con\c tine pe $d$ sau alte tipuri de date $d'$, defini\c tia poate p\u area circular\u a. Totu\c si, faptul c\u a un constructor creeaz\u a o valoare cu dimensiune mai mare dec\^ at argumentul s\u au ne asigur\u a c\u a sunt folosite interpret\u ari ale lui $d$ doar pentru dimensiuni pentru care au fost deja calculate.
\end{remark}
\begin{remark}
Av\^ and \^ in vedere c\u a un tip recursiv $d$ poate avea constructori cu argumente de tip $d'$ doar dac\u a $d'$ apare \^ in program \^ inainte de de $d$ \c si faptul c\u a interpretarea se face \^ in ordinea declara\c tiilor din program, ne asigur\u a ca defini\c tia recursiva pentru interpret\u ari nu este circular\u a.
\end{remark}
\begin{remark}
Atunci c\^ and un ordinal nu are predecesor, el este numit ordinal limit\u a pentru c\u a poate fi v\u azut ca $x = \lim_{y < x} y $. Un exemplu de acest gen este $\infty = \sup_{n \in \mathbb{N}}n$.
\end{remark}
\done\todo{de ce sunt doar aceste cazuri?}

Urm\u atoarea propozi\c tie asigur\u a corectitudinea interpret\u arii dimensiunilor \^ in ceea ce prive\c ste rela\c tia de ordine.
\begin{proposition}
\begin{align*}
s \le \widehat{s}           \: &\equiv \: D_d(S,x) \subseteq D_d(S,succ(x))\\
\widehat{\infty} \le \infty \: &\equiv \: D_d(S,\Omega) = D_d(S, \Omega+1)
\end{align*}
\end{proposition}
\begin{proof}[Demonstra\c tie]
Faptul c\u a se ajunge la un punct fix pentru un ordinal mai mic dec\^ at $\Omega$ se da\-to\-re\-a\-z\u a faptului c\u a la fiecare pas, mu\-l\c ti\-me\-a $D_d(S,x)$ cre\c ste dar fiind mul\c time de expresii System \fhat este numarabil\u a, \^ in timp ce $\Omega$ nu este num\u arabil.
\end{proof}
Analog se poate defini \c si o interpretare a expresiilor prin alte expresii System \fhat pornind de la o interpretare a variabiellor libere.

\subsection{Corectitudinea interpret\u arii}
\begin{definition}
Fie o interpretare dat\u a de tripletul $(\pi,\zeta, \rho)$ - interpret\u ari ale variabilelor libere de dimensiune, tip \c si termeni. Aceast\u a interpretare satisface un context de tip dac\u a $\rho(e) \in \langle \tau \rangle_{\pi,\zeta}$ pentru orice $(x:\tau) \in \Gamma$ \c si se noteaz\u a cu $(\pi,\zeta, \rho) \models \Gamma$. Ea satisface o deduc\c tie $\Gamma \vdash e : \tau$ dac\u a
    $$(\pi,\zeta, \rho) \models \Gamma \Rightarrow \langle e \rangle \in \langle \tau \rangle_{\pi,\zeta}$$
Dac\u a o deduc\c tie este satisf\u acuta\u  de orice interpretare atunci ea se nume\c ste valid\u a \c si vom scrie $\Gamma \models e : \sigma$
\end{definition}

Vom demonstra prin induc\c tie urmatoarea propozi\c tie, proprietatea de normalizare puternica fiind un caz particular al acestei propozi\c tii.
\begin{proposition}\label{soundness}
$$ \Gamma \vdash e : \sigma \Rightarrow   \Gamma \models e : \sigma$$
\end{proposition}
\begin{proof}[Demonstra\c tie]
Se demonstreaz\u a prin inductie structural\u a pe arborele de deduc\c tie al propozi\c tiei $\Gamma \vdash e : \sigma$. Cazurile specifice, care nu apar \^ in alte demonstra\c tii de normalizare puternic\c a, ca de exemplu pentru System F,  sunt pentru care ultima regula aplicata este {\scriptsize (T-CONS), (T-CASE), (T-LETREC), (T-TABS)} sau {\scriptsize (T-TAPP)}. Cazutile corespunz\u atoare regulilor {\scriptsize (T-TAPP)} \c si {\scriptsize (T-TABS)} sunt triviale.
\begin{description}
  \item[{\scriptsize (T-CONS)}] Trebuie s\u a ar\u at\u am c\u a $c = \langle c \rangle_{\rho} \in \langle \Pi X. \theta \to d^{\hat{\iota}}\app X \rangle_{\pi,\zeta} $, deci conform defini\c tiei, c\u a
      \begin{equation}
      \forall \tau. \quad c\app \tau \in \langle [X \mapsto \tau] \theta \rangle_{\pi,\zeta} \to \langle d^{\hat{\iota}} \tau \rangle_{\pi,\zeta}
      \end{equation}
      care este echivalenta cu
      \begin{equation}
      \forall \tau, e \in \langle [X \mapsto \tau] \theta \rangle_{\pi,\zeta} . \quad c\app \tau\app e \in \langle d^{\hat{\iota}} \tau \rangle_{\pi,\zeta}
      \end{equation}
      Lucru adevarat din defini\c tia lui $\langle d^{\hat{\iota}} \tau \rangle_{\pi,\zeta} = \bigcup \{ c\app \tau\app e \:|\: e \in \langle \theta \rangle_{[\iota \to \langle \iota \rangle_{\pi}]\pi, [ X \mapsto \langle \tau \rangle_{\zeta}]\zeta} \} $

  \item[{\scriptsize (T-CASE)}] Trebuie s\u a ar\u at\u am c\u a $\langle \text{case}_{\sigma}\ e \text{ of } \{ \vec{c} \Rightarrow \vec{e} \} \rangle_{\rho} \in \langle \sigma \rangle_{\pi,\zeta} $ \^ in condi\c tiile \^ in care, din ipoteza de induc\c tie avem $\langle e \rangle_{\rho} \in \langle d^{\hat{s}}\tau \rangle_{\pi,\zeta}$ si $\langle e_k \rangle_{\rho} \in \langle [X\mapsto \tau, i\mapsto s] \theta_k \rangle_{\pi,\zeta} \to \langle \sigma \rangle_{\pi,\zeta}$.

  \c Stim c\u a $\langle e \rangle_{\rho} \in \langle d^{\hat{s}}\tau \rangle_{\pi,\zeta}$ \c si din defini\c tia \^ inchiderii trebuie s\u a $\exists e'$ astfel \^ incat $\langle e \rangle_{\rho}\to_k^* e'$ si
  \begin{equation}
  e' \in B \cup \bigcup \{c_k \app \tau\app t \: | \: t \in \langle \theta_k \rangle_{[ \iota \mapsto s]\pi,[X \mapsto \langle \tau \rangle_{\pi, \zeta } ] \zeta}\}
  \end{equation}

  Demonstr\u am acum c\u a $\langle \text{case}_{\sigma}\ e' \text{ of } \{ \vec{c} \Rightarrow \vec{e} \} \rangle_{\rho} \in \langle \sigma \rangle_{\pi,\zeta}$ prin tratarea a dou\u a cazuri
  \begin{enumerate*}
    \item $ e' \in B$. \^ In acest caz
        \begin{equation}
            \text{case}_{\sigma}\ \langle e \rangle_{\rho}\text{ of } \{ \vec{c} \Rightarrow \vec{\langle e\rangle_{\rho}} \} \to_k^*
            \text{case}_{\sigma}\  e' \text{ of } \{ \vec{c} \Rightarrow \vec{\langle e\rangle_{\rho}} \}
            \in B \subseteq \langle \sigma \rangle_{\pi,\zeta}
         \end{equation}
         Deci concluzia este demonstrat\u a av\^ and \^ in vedere c\u a $\langle \sigma \rangle_{\pi,\zeta} $ este \^ inchisa la expansiune slaba (opusul reducerii slabe).
    \item $ e' = c_k \app \tau \app t $ cu $t \in \langle \theta_k \rangle_{[ \iota \mapsto \langle s \rangle_{\pi}]\pi,[X \mapsto \langle \tau \rangle_{\pi, \zeta } ] \zeta} = \langle [X\mapsto \tau, i\mapsto s] \theta_k \rangle_{\pi,\zeta}$. In acest caz ob\c tinem
        \begin{equation}
            \text{case}_{\sigma}\ e' \text{ of } \{ \vec{c} \Rightarrow \vec{\langle e\rangle_{\rho}} \} \to_k
            \langle e_k  \rangle_{\rho}\app t \in  \langle \sigma \rangle_{\pi,\zeta}
        \end{equation}

        Cum $\text{case}_{\sigma}\ e' \text{ of } \{ \vec{c} \Rightarrow \vec{\langle e\rangle_{\rho}} \}$ este reductibil\u a \^ inseamna c\u a $\in  \langle \sigma \rangle_{\pi,\zeta}$. Analog, expresia $\text{case}_{\sigma}\ \langle e \rangle_{\rho} \text{ of } \{ \vec{c} \Rightarrow \vec{\langle e\rangle_{\rho}} \}$ este reductibil\u a (lema \ref{cond_reductibil}) \c si se reduce la un termen din $\langle \sigma \rangle_{\pi,\zeta}$, deci apar\c tine acestei mul\c timi.
  \end{enumerate*}

  \item[{\scriptsize (T-LETREC)}] \^ In acest ultim caz trebuie sa ar\u at\u am c\u a
    \begin{equation} \label{concl}
        \langle \text{letrec}_{d|\tau| \to |\theta|}f := g \rangle_{\rho} \in
        \langle d^s\app \tau \to [\iota \mapsto s] \theta \rangle_{\pi,\zeta} =
        \langle d^s\app \tau  \rangle_{\pi,\zeta}  \to \langle \theta  \rangle_{[\iota \mapsto \langle s \rangle_{\pi}] \pi,\zeta}
    \end{equation}
    Not\u am cu $\pi_0 = [\iota \mapsto \langle s \rangle_{\pi}]\pi$ si $\rho_0 = [f \mapsto f]\rho$. Atunci obiectivul nostru devine
    \begin{equation*}\label{concl_reduced}
        (\text{letrec}_{d|\tau| \to |\theta|} f := \langle g \rangle_{\rho_0}) \app e \in \langle \theta \rangle_{\pi_0,\zeta}
        \text{  pentru  } \forall e \in \langle d^\iota\app \tau \rangle_{\pi_0,\zeta}
    \end{equation*}
    Deoarece $f$ este variabila libera \^ in corpul func\c tiei textual recursive $g$, are loc rela\c tia $\langle f \rangle_{\rho_0} = \rho_0(f) = f \in B \subseteq \langle d^\iota\app\tau\to\theta \rangle_{\pi,\zeta}$. \^ In consecin\c ta, conform ipotezei de induc\c tie, avem c\u a $\langle g \rangle_{\rho_0} \in \langle d^{\hat{\iota}}\app\tau\to [\iota \mapsto \hat{\iota}]\theta \rangle_{\pi,\zeta} \subseteq SN$.

    Vom demonstra \eqref{concl_reduced} prin induc\c tie trasnfinit\u a dup\u a ordinalul $\pi_0(\iota)$
    \begin{equation}
        \forall \pi_0,\zeta,\rho_0,e \in \langle d^\iota\app \tau \rangle_{\pi_0,\zeta}.\:  (\text{letrec}_{d|\tau| \to |\theta|} f := \langle g \rangle_{\rho_0}) \app e \in \langle \theta \rangle_{\pi_0,\zeta}
    \end{equation}
    \begin{description*}
    \item [Caz 1: $\pi_0(\iota)=0$] Fie $e \in \langle d^{\iota} \tau \rangle_{\pi_0,\zeta} = \overline{\emptyset}$. Atunci $\exists e'$ astfel \^ inc\^ at $e \to_k^* e'$ si $e' \in B$. Deci
        \begin{equation}
            (\text{letrec}_{d|\tau| \to |\theta|} f := \langle g \rangle_{\rho_0}) \app e \to_k^* (\text{letrec}_{d|\tau| \to |\theta|} f := \langle g \rangle_{\rho_0}) \app e' \in B
        \end{equation}
        Cum $(\text{letrec}_{d|\tau| \to |\theta|} f := \langle g \rangle_{\rho_0}) \app e$ este reductibil\u a (lema \ref{cond_reductibil}) \c si se reduce la o expresie de baz\u a \^ inseamna c\u a face parte din orice mul\c time saturat\u a deci \c si din $\langle d^\iota\app \tau \rangle_{\pi_0,\zeta}$

    \item [Caz 2: $\pi_0(\iota)=succ(y)$] Fie $\pi' = [\iota \mapsto y]\pi $ si $\rho' = [f \mapsto (\text{letrec}_{d|\tau| \to |\theta|} f := \langle g \rangle_{\rho_0})] \rho$. Din ipoteza de induc\c tie interioar\u a avem c\u a
        \begin{equation}
            \langle f \rangle_{\rho'} = (\text{letrec}_{d|\tau| \to |\theta|} f := \langle g \rangle_{\rho_0}) \in \langle d^{\iota}\app \tau \to \theta \rangle_{\pi',\zeta}
        \end{equation}
        Iar din ipoteza de induc\c tie exterioar\u a, avem \^ in continuare
        \begin{equation}\label{eq16}
        \langle g \rangle_{\rho'} \in \langle d^{\hat{\iota}}\app \tau \to [\iota \mapsto \hat{\iota}]\theta \rangle_{\pi',\zeta} =
        \langle d^{{\iota}}\app \tau \to \theta \rangle_{\pi_0,\zeta}
        \end{equation}
        Cum $e \in \langle d^{\iota} \tau \rangle_{\pi_0,\zeta}$, conform defini\c tiei se reduce la $e'$ care este expresie de baz\u a, caz \^ in care demonstra\c tia este similar\u a cu cea de la cazul anterior, sau are forma $e' = c\app \tau\app t , t \in \langle d^\iota\app \tau \rangle_{\pi_0,\zeta}$. Atunci, conform \eqref{eq16}, avem
        $$
        (\text{letrec}\ f := \langle g \rangle_{\rho_0})\app e' \to_k ([f \mapsto (\text{letrec}\ f := \langle g \rangle_{\rho_0})] \langle g \rangle_{\rho_0}) \app e' = \langle g \rangle_{\rho'}\app e' \in \langle \theta \rangle_{\pi_0,\zeta}
        $$
        Cum expresia $(\text{letrec}\ f := \langle g \rangle_{\rho_0})\app e'$ este reductibil\u a, \^ inseamna c\u a ea face parte din mul\c timea saturat\u a $\langle \theta \rangle_{\pi_0,\zeta}$. Similar, $(\text{letrec}\ f := \langle g \rangle_{\rho_0})\app e$ este reductibil\u a (lema \ref{cond_reductibil}) \c si se reduce la un element din $\langle \theta \rangle_{\pi_0,\zeta}$, deci apar\c tine ea insa\c si lui $\langle \theta \rangle_{\pi_0,\zeta}$.
    \item [Caz 3: $\pi_0(\iota)=\sup_{y<x}y$] \^ In acest caz avem $e \in \langle d^{\iota} \tau \rangle_{\pi_0,\zeta} = \bigcup _{y < x}\langle d^{\iota} \tau \rangle_{[\iota \mapsto y]\pi_0,\zeta}$ deci $e$ apar\c tine unuia din termenii reuniunii, fie el $\langle d^{\iota} \tau \rangle_{[\iota \mapsto y]\pi_0,\zeta}$. Din ipoteza de induc\c tie:
         $$
            (\text{letrec}_{d|\tau| \to |\theta|} f := \langle g \rangle_{\rho_0}) \app e \in \langle \theta \rangle_{[\iota \mapsto y]\pi_0,\zeta}
         $$
        Dar $\iota \text{ pos } \theta$, \c si concluzia este demonstrat\u a pentru c\u a $\langle \theta \rangle_{[\iota \mapsto y]\pi_0,\zeta} \subset \langle \theta \rangle_{\pi_0,\zeta}$. \qedhere
    \end{description*}
\end{description}
\end{proof}

\begin{lemma}\label{cond_reductibil}
\^ In cursul demonstra\c tiei am folosit urmato\u arele condi\c tii suficiente pentru nor\-ma\-li\-za\-re\-a unei expresii:
\begin{enumerate*}
\item Dac\u a $e \in SN$ \c si $e \to_k e'$ \c si $\text{\emph{case}}_{\sigma}\ e' \text{ \emph{of} } \{ \vec{c} \Rightarrow \vec{e} \}  \in SN$, atunci $\text{\emph{case}}_{\sigma}\ e \text{ \emph{of} } \{ \vec{c} \Rightarrow \vec{e} \} \in SN$.

\item Dac\u a $e \in SN$ \c si $e \to_k e'$ \c si $(\text{\emph{letrec}}_{\dots} f :=  g)\app e' \in SN$, atunci $(\text{\emph{letrec}}_{\dots} f :=  g)\app e \in SN$.

\item Dac\u a $t,g, [f \mapsto(\text{\emph{letrec}}_{\dots} f :=  g)] g\app (c\app \tau\app t) \in SN$, atunci $(\text{\emph{letrec}}_{\dots} f :=  g) \app (c\app \tau\app t) \in SN$
\end{enumerate*}
\end{lemma}

\begin{remark}
Urm\u atorul exemplu ilustreaz\u a de ce nu a fost suficient s\u a interpret\u am dimensiunile \^ in numere naturale. Intuitiv, vom codifica ordinalii num\u arabili \^ in System \fhat prin urmatorul tip de date
$$ Datatype \ \textbf{Ord} := oz : \textbf{Ord}^{\hat{\iota}} \:|\: os : \textbf{Ord}^{{\iota}} \to \textbf{Ord}^{\hat{\iota}} \:|\: olim : (\textbf{Nat} \to \textbf{Ord}^{{\iota}}) \to \textbf{Ord}^{\hat{\iota}}$$
\^ In cazul interpret\u arii acestui tip, procedeul iterativ de interpretare pentru tipuri recursive se aplic\u a \^ in mod netrivial pentru fiecare ordinal.
\end{remark}

\begin{remark}
Dup\u a cum se poate observa, nu orice element din interpretarea unui tip corespunde unei expresii cu acel tip. Intuitiv, elementele ap\u arute \^ in plus provin de la aproximarea de dimensiune 0 a tipului, adic\u a elemente construite cu zero constructori.
\end{remark}

\begin{corollary}
System \fhat are proprietatea de normalizare puternic\u a.
\end{corollary}
\begin{proof}[Demonstra\c tie]
Fie $e$ o expresie \^ in System \fhat pentru care $\Gamma \vdash e : \sigma$. Atunci deduc\c tia $\Gamma \vdash e : \sigma$ este satisfacut\u a de orice interpretare, deci \c si de interpretarea care duce variabilele libere din $e$ \^ in ele \^ insele. \^ In consecint\u a avem c\u a $e \in \langle \sigma \rangle \in SAT$, deci expresia $e$ este membr\u a a unei mul\c timi saturate \c si \^ in consecin\c ta reductibil\u a.
\end{proof}

\subsection{Confluen\c ta}
O alt\u a proprietate str\^ ans legat\u a  de cea de normalizare, care atesta existen\c ta unei forme normale pentru fiecare termen, este cea de confluen\c ta care atest\u a unicitatea.

\begin{corollary}
Rela\c tia de reducere pentru System \fhat este confluenta.
\end{corollary}
\begin{proof}[Demonstra\c tie]
Deoarece \^ in procedeul de reducere \^ in care se folosesc ca reguli de context {\scriptsize (CTX-APP),(CTX-TAPP), (CTX-CASE)} la fiecare pas exista maxim o regul\u a care poate fi aplicat\u a,  exista cel mult o form\u a normal\u a la care se poate ajunge.
\end{proof}

\section{Tipuri principale}
\label{tip_princ}
\done\todo{SysF star si de ce are asta tipuri principale}
Asa cum este el definit, System \fhat nu are tipuri principale, adic\u a unele expresii nu au un cel mai general tip \^ in sensul rela\c tiei de subtip.
\begin{example}[Barthe et al. \citep{DBLP:conf/tlca/BartheGP05}]
Un exemplu de expresie care nu are un cel mai general tip este urmatoarea:
$$square := \lambda f : \textbf{Nat} \to \textbf{Nat}. \lambda x : \textbf{Nat}. f \app f\app x$$
Pentru $square$ pot fi asociate tipurile $(\textbf{Nat}^\iota \to \textbf{Nat}^\iota) \to \textbf{Nat}^\iota \to \textbf{Nat}^\iota$ \c si $(\textbf{Nat}^\iota \to \textbf{Nat}^\infty) \to \textbf{Nat}^\iota \to \textbf{Nat}^\infty$. Pentru aceste tipuri nu exist\u a un supertip comun.
\end{example}

Din aceasta cauz\u a s-au introdus tipurile cu pozi\c tie \citep{1614481}, care adnoteaz\u a o parte dintre constructori de tip cu $\star$ \^ in declara\c tiile de tip LETREC. \c Tin\^ and cont de aceste adnot\u ari regula {\scriptsize (T-LETREC)} se schimb\u a astfel

\begin{prooftree}
\AxiomC{$\overline{\Gamma}, f : d^\iota\tau \to \overline{\theta} \vdash e : d^{\hat{\iota}}\tau \to [\iota \mapsto \hat{\iota}] \overline{ \theta}$}
\AxiomC{$\iota \text{ pos } \overline{\theta} $}
\AxiomC{$d^\iota\tau \to \overline{\theta} \approx_{\iota} d^\star|\overline{\tau}| \to |\overline{\theta}|^\star$}

\RightLabel{\scriptsize (T-LETREC$\star$)}
\TrinaryInfC{$\overline{\Gamma} \vdash (\text{letrec}_{d^\star|\overline{\tau}| \to |\overline{\theta}|^\star} f = e) : d^s \overline{\tau} \to [\iota \mapsto s]\overline{\theta}$}
\end{prooftree}

unde $\tau_1 \approx_{\iota} \tau_2$ dac\u a $\tau_1$ si $\tau_2$ au aceea\c si structur\u a \c si un constructor este adnotat cu $\widehat{ \iota }$ (al $k$-lea succesor) \^ in $\tau_1$ dac\u a \c si numai dac\u a este adnotat cu $\star$ in $\tau_2$. Cu ajutorul acestei adnot\u ari se poate sepcifica mai exact tipul dorit al unei expresii. \^ In sec\c tiunea dedicata algoritmului de verificare a tipului se poate vedea ca acesta \^ intoarce o clasa de tipuri, ci nu un singur tip - cel mai general.

 % Principalele caracteristici ale SystemF^

\chapter{Expresivitatea System F\^{} \c si a extensiilor sale}
\label{Capitolul5}
\lhead{Capitolul 5. \emph{Expresivitate}}

\section{Incompletitudinea Turing}

Un neajuns al System \fhat \^ in privin\c ta expresivit\u a\c tii este remarcat de urmatoarea teorem\u a \citep{citeulike:4023285}

\begin{theorem}\label{turing_incomplete}
Dac\u a toate programele ce pot fi scrise \^ intr-un limbaj de programare se termin\u a, atunci exist\u a programe (expresibile cu ma\c sini Turing) care se termin\u a intotdeauna, printre care \c si interpretorul limbajului respectiv, \c si care nu pot fi scrise \^ in acel limbaj.
\end{theorem}

\begin{proof}[Demonstra\c tie]
Vom ar\u ata c\u a nu se poate scrie \^ in System \fhat un interpretor pentru acesta. Mai exact, vom considera doar cazul \^ in care singurul tip de date definit \^ in limbajul interpretat este \textbf{Bool}. Expresiile sunt reprezentate prin arborele lor sintactic ca membrii ai unui tip de date \textbf{Expresie} (Figura \ref{expresie_datatype}) . Am folosit numerotarea deBrujin pentru a re\c tine variabilele din program.
\begin{figure}
\begin{equation*}
\begin{split}
Datatype\ \textbf{Expresie}
                    &:= var : \textbf{Nat} \to \textbf{Expresie}^{\hat{\iota}}\\
                    &|\ lambda : \textbf{Expresie}^{\iota} \to \textbf{Expresie}^{\hat{\iota}} \\
                    &|\ app : \textbf{Expresie}^{\iota}\to \textbf{Expresie}^{\iota}\to \textbf{Expresie}^{\hat{\iota}}  \\
                    &|\ tLambda : \textbf{Expresie}^{\iota} \to\textbf{Tip} \to \textbf{Expresie}^{\hat{\iota}} \\
                    &|\ tApp : \textbf{Nat} \to \textbf{Expresie}^{\iota}\to \textbf{Expresie}^{\iota}\to \textbf{Expresie}^{\hat{\iota}}  \\
                    &|\ case : \textbf{Tip} \to \textbf{Expresie}^{{\iota}} \to \textbf{Expresie}^{{\iota}}\to \textbf{Expresie}^{{\iota}}\to \textbf{Expresie}^{\hat{\iota}}\\
                    &|\ letrec : \textbf{Tip} \to \textbf{Nat} \to \textbf{Expresie}^{{\iota}} \to \textbf{Expresie}^{\hat{\iota}}\\
                    &|\ true : \textbf{Expresie} \\
                    &|\ false : \textbf{Expresie} \\
Datatype\ \textbf{Tip}
                    &:= tvar   : \textbf{Nat} \to \textbf{Tip}^{\hat{\iota}}\\
                    &|\ func   : \textbf{Tip}^{{\iota}}\to \textbf{Tip}^{{\iota}}\to \textbf{Tip}^{\hat{\iota}}\\
                    &|\ pi     : \textbf{Tip}^{{\iota}}\to \textbf{Tip}^{\hat{\iota}} \\
                    &|\ tBool  : \textbf{Tip}^{\hat{\iota}}
\end{split}
\end{equation*}
\caption{Tipul de date Expresie}
\label{expresie_datatype}
\end{figure}
Un interpretor pentru subsetul System \fhat propus mai sus este o func\c tie matematic\u a $interpret$ care prime\c ste reprezentarea unui program, \c si a unei expresii \c si \^ intoarce $true$ dac\u a programul aplicat expresiei respect\u a condi\c tiile legate de tip \c si se evalueaz\u a la $true$ \c si $false$ altfel.

Presupunem c\u a exist\u a o func\c tie \^ in System \fhat, $f : \textbf{Expresie} \to \textbf{Expresie} \to \textbf{Bool}$ care calculeaz\u a rezultatul func\c tiei $interpret$. Consider\u am func\c tia
$$ g := \lambda x : \textbf{Expresie}. \text{case}_{\textbf{Bool}} f\app x\app x \text{ of } \{ true \Rightarrow false | false \Rightarrow true \} $$
\c si reprezentarea acesteia ca valoare a tipului de date \textbf{Expresie}:
$$ e_g := lambda\app (case\app tBool \app  (app\app (app\app f\app (var\app z)) (var\app z))\app false\app true ) $$
Cum expresia $(g\app e_g)$ respect\u a constrangerile de tip cerute, avem c\u a:
$$ (g\app e_g) \downarrow v \in \{true, false\} \Rightarrow (interpret\app e_g\app e_g) = v $$
Din confluen\c ta sistemului de rescriere, avem c\u a:
$$ (g\app e_g) \downarrow v \wedge (g\app e_g) \to not(f\app e_g\app e_g) \Rightarrow (f\app e_g\app e_g) \downarrow not\app v $$
Deci $f$ difer\u a de func\c tia $interpret$.
\done\todo{de demostrat cu interpretorul}
\end{proof}

\begin{corollary}
Limbajul System \fhat este Turing incomplet.
\end{corollary}

\section[Programe expresibile]{Programe expresibile in System F\^{}}
% 1 pag
Toate cele trei tipuri de recursivitate descrise \^ in sec\c tiunea \ref{scheme_rec} pot fi exprimate \^ in System \fhat. Pentru cazul recursivit\u a\c tii primitive putem adapta demonstra\c tia teoremei \ref{compl_sysfrec}. Singurul lucru care trebuie facut este determinarea unui tip pentru operatorului de recursivitate primitiv\u a. Vom prezenta un exemplu de abore de deduc\c tie pentru tipul acestui operator \^ in System \fhat
\begin{align*}
\text{p\_rec}_0     &:= \lambda g:\textbf{Nat}. \lambda h:\textbf{Nat} \to \textbf{Nat}\to \textbf{Nat}.\\
                    & \text{letrec}_{\textbf{Nat}\to\textbf{Nat}}\ f =\lambda y : \textbf{Nat} . \text{case}_{\textbf{Nat}}\ y \text{ of } \\
                    & \qquad z \Rightarrow g \\
                    & \qquad s \Rightarrow \lambda y'. (h\app y'\app (f\app y'))
\end{align*}
Not\u am cu  $\Gamma := g:\textbf{Nat}^{\infty} , h:\textbf{Nat}^{\infty} \to \textbf{Nat}^{\infty}\to \textbf{Nat}^{\infty}, f : \textbf{Nat}^{\iota}\to \textbf{Nat}^{\infty}$. Conform {\scriptsize (T-LETREC)}, este suficient s\u a demonstr\u am
\begin{prooftree}
\AxiomC{$ y :  \textbf{Nat}^{\hat{\iota}} \in \Gamma_1$}
\UnaryInfC{$\Gamma_1 \vdash y :  \textbf{Nat}^{\hat{\iota}}$}
    \AxiomC{$g :  \textbf{Nat}^{\infty} \in \Gamma_1 $}
    \UnaryInfC{$\Gamma_1 \vdash g :  \textbf{Nat}^{\infty}$}

        \AxiomC{\scriptsize \dots continua mai jos}
        \UnaryInfC{$\Gamma_1 \vdash \lambda y': \textbf{Nat}.(h\app y'\app(f\app y')): \textbf{Nat}^{{\iota}} \to \textbf{Nat}^{\infty}$}

\RightLabel{\scriptsize (T-CASE)}
\TrinaryInfC{$\Gamma_1 := \Gamma, y : \textbf{Nat}^{\hat{\iota}} \vdash \text{case}_{\textbf{Nat}}\ y \text{ of } \{z \Rightarrow g \:|\: s
\Rightarrow \lambda y' :\textbf{Nat}. (h\app y'\app (f\app y')) \} : \textbf{Nat}^{\infty}$ }

\RightLabel{\scriptsize (T-ABS)}
\UnaryInfC{$\Gamma \vdash \lambda y : \textbf{Nat} . \text{case}_{\textbf{Nat}}\ y \text{ of } \{z \Rightarrow g \:|\: s \Rightarrow \lambda y' : \textbf{Nat}. (h\app y'\app (f\app y')) \}  : \textbf{Nat}^{\hat{\iota}} \to \textbf{Nat}^{\infty}$}
\end{prooftree}

Aplicarea regulilor de tip continu\u a cu
\begin{prooftree}
\AxiomC{\scriptsize \dots continua mai jos}
\UnaryInfC{$\Gamma_2 \vdash (h\app y'): \textbf{Nat}^{\infty}\to \textbf{Nat}^{\infty}$}

        \AxiomC{$ y' :  \textbf{Nat}^{{\iota}} \in \Gamma_2$}
        \UnaryInfC{$\Gamma_2 \vdash y': \textbf{Nat}^{\iota}$}
            \AxiomC{$ f :  \textbf{Nat}^{\iota}\to \textbf{Nat}^{\infty} \in \Gamma_2$}
            \UnaryInfC{$\Gamma_2 \vdash f : \textbf{Nat}^{\iota}\to \textbf{Nat}^{\infty}$}
        \BinaryInfC{$\Gamma_2 \vdash (f\app y') : \textbf{Nat}^{\infty}$}
\BinaryInfC{$\Gamma_2:=\Gamma_1, y':\textbf{Nat}^\iota \vdash h\app y'\app (f\app y') : \textbf{Nat}^{\infty}$}
\UnaryInfC{$\Gamma_1 \vdash \lambda y': \textbf{Nat}.(h\app y'\app(f\app y')): \textbf{Nat}^{{\iota}} \to \textbf{Nat}^{\infty}$}
\end{prooftree}
Pentru a completa ra\c tionamentul mai trebuie demonstrat c\u a
\begin{prooftree}
\AxiomC{$ y' :  \textbf{Nat}^{{\iota}} \in \Gamma_2$}
    \AxiomC{$ \iota \le \infty$}
    \UnaryInfC{$ \textbf{Nat}^{{\iota}} \sqsubseteq \textbf{Nat}^{{\infty}}$}
\BinaryInfC{$\Gamma_2 \vdash y': \textbf{Nat}^{\infty}$}
    \AxiomC{$ h :  \textbf{Nat}^{\infty}\to \textbf{Nat}^{\infty} \in \Gamma_2$}
    \UnaryInfC{$\Gamma_2 \vdash h : \textbf{Nat}^{\infty}\to \textbf{Nat}^{\infty}$}
\BinaryInfC{$\Gamma_2 \vdash (h\app y'): \textbf{Nat}^{\infty}\to \textbf{Nat}^{\infty}$}
\end{prooftree}

\^ In acelasi mod se aplic\u a regulile de tip \c si pentru demonstra\c tia termin\u arii \c si \^ in celealte scheme de recursivitate. \done\todo{functii primitiv recursive}

\section[Programe (greu /in)expresibile]{Programe (greu /in)expresibile in System F\^{}}
% 1 pag
Cu siguranta c\u a exist\u a \c si programe care se termina dar caro\u ra, scrise \^ in forma naturala, nu li se poate stabili un tip \^ in System \fhat. Un exemplu este dat de parcurgerea \^ in l\u a\c time a unui arbore binar. Aceasta poate fi scris\u a in Haskell astfel:

\begin{lstlisting}[label=bfs,captionpos=b,caption=Parcurgerea \^ in l\u a\c time a unui arbore binar,language=Haskell]
data Tree = Nd Tree Int Tree | Lf deriving Show;
bfs [] o                   = ([] , o)
bfs ((Nd l val r) : nds) o = let ( q' , o') = bfs (nds ++ [l,r]) o  in
                                 ( q' , val : o')
bfs (Lf : nds) o           = bfs nds o
mainBfs root               = snd $ bfs [root] []
\end{lstlisting}

Prin \^ inlocuirea defini\c tiilor prin pattern-mathcing cu expresii bazate pe construc\c tia \emph{case} \c si adnotari cu tipuri, acest program poate fi scris \c si in System \frec. Totusi dimensiunea tipului nici unuia din argumentele func\c tiei \emph{bfs} nu scade. Lista cu noduri ce trebuie parcurse cre\c ste atunci c\^ and este atins un nod \c si scade atunci c\^ and este atins\u a o frunz\u a. Argumentul ce con\c tine lista nodurilor parcurse cre\c ste la fiecare apel al func\c tiei.
\done\todo{ gasit un exemplu care nu poate fi tipat / greu tipat, parcurgerea in latime}
\section{Tipuri de date coinductive}
% 1 pag
\done\todo{ scurta descriere a extensiei + referinta la 1. }
O categorie important\u a de programe care sunt folosite \^ in practica dar nu pot fi exprimate in System \fhat sunt programele reactive, de exemplu: sisteme de operare, programe cu interfa\c ta grafica, etc. Acestea pot fi modelate ca func\c tii de la un flux (posibil infinit) de cereri catre un flux (posibil infinit) de r\u aspunsuri. Pentru introducerea acestui tip de programe \^ in limbaj, acesta ar trebui extins cu \emph{tipuri de date coinductive} \citep{citeulike:4023285}.
\begin{example}
$$ coDatatype \quad \textbf{\emph{coList}} := co\_cons : \textbf{\emph{Nat}}\to \textbf{\emph{coList}} \to \textbf{\emph{coList}} $$
\end{example}

Acestea sunt un concept dual al celui de tipuri de date inductive, \c si sunt folosite pentru a exprima obiecte infinite, dar care au o structur\u a regulat\u a. \^ In cazul func\c tiilor care lucreaz\u a pe astfel de tipuri de date nu se pune problema terminarii. Totu\c si, daca luam drept valori (forme normale) toate expresiile care sunt aplicatii de \emph{co-constructori}, cu o strategie de evaluare lene\c s\u a, se poate stabili proprietatea de normalizare (slaba) \^ in condi\c tii similare cu cele ale System \fhat \citep{DBLP:journals/ita/Abel04}.

Aceast\u a tactic\u a a fost folosita in Haskell pentru a scrie programe interactive \^ inainte de introducerea monadului IO, sub forma \emph{dialogurilor}\citep{158524}. Astfel programul principal trebuia sa aib\u a tipul $\texttt{Dialogue} \equiv \texttt{[Request]} \to \texttt{[Response]}$. Acesta era executat de un wrapper care construia pe baza interac\c tiunii utilizatorului (co)lista \texttt{[Request]} \c si apoi prezenta utilizatorului, pe masur\u a ce erau \^ intoarse de program, elementele din (co)lista \texttt{[Response]}.

\section{Tipuri monadice}\label{tip_monad}
% 3 pag
\newcommand\id[1]{ {#1}_{\text{id}} }
\newcommand\nt[1]{ {#1}_{\text{nt}} }

O alta solu\c tie pentru includerea \^ in limbaj a unor programe care nu se termina este sa le asociem acestora un alt set de tipuri de date disjunct fa\c ta de tipurile System \fhat. O solu\c tie similara este folosita \^ in Haskell pentru separarea \^ intre programele care fac opera\c tii de IO fa\c ta de programele \emph{pure} \citep{158524}. \^ In Haskell o expresie care se evalueaz\u a la un \^ intreg \c si este posibil sa fac\u a opera\c tii de IO are tipul $\textbf{IO Int}$, altfel are tipul $\textbf{Int}$ sau echivalent $\textbf{ID Int}$. \^ In cazul nostru o expresie care se evalueaz\u a la o un \^ intreg, dar a carei evaluare este posibil s\u a nu se termine, o sa aiba tipul $\nt{\textbf{Int}}$, altfel $\id{\textbf{Int}}$.

\subsection{Nota\c tie}

Cu toate c\u a sintaxa declara\c tiilor de tipuri r\u amane neschimbat\u a, semantica acestora este diferit\u a. Astfel, pentru fiecare tip de date inductiv cu dimensiune, $d$, declarat vor exista dou\u a tipuri de date: $\id{d}$ pentru programe care au un tip \^ in System \fhat, \c si $\nt{d}$ pentru programe a c\u aror terminare nu se poate demonstra conform System \fhat dar respecta reguli de tip de genul celor din System \frec.

Tipurile de date sunt formate cu constructorii $\id{d},\nt{d},\to,\Pi$. Vom numi aceste tipuri, \textbf{tipuri monadice} pentru c\u a tipurile de date etichetate cu \emph{nt} sunt similare monazilor folosi\c ti in Haskell. Vom nota cu \textbf{ID} mul\c timea tipurilor care au \^ in componen\c ta doar constructori de tip de forma $\id{d}$, \c si cu \textbf{NT} complementara sa.

Pentru tipuri monadice vom folosi variabile precum $\widetilde{\tau}, \widetilde{\sigma}$, pentru tipuri cu dimensiune vom folosi $\overline{\tau}, \overline{\sigma}$ iar pentru tipuri simple System \frec - $\tau, \sigma$. Nota\c tia $\id{\tau}$ reprezinta adnotarea fiecarui constructor de tip din $\tau$ cu "id". Func\c tia $|.| : \widetilde{T} \to T_{\text{System \frec}}$ va \c sterge \c si dimensiunile tipurilor \c si adnotarile specifice tipurilor monadice.
\done\todo{Notatie}

\subsection{Reguli de tip}

Singura modificare \^ in regulile de subtip este aceea c\u a regula pentru constructori de tip defini\c ti de utilizator este \^ impartita in dou\u a. Pentru tipuri din \textbf{NT} ea face toate aproxim\u arile egale \^ intre ele \c si izomorfe cu tipul de date System \frec corespunz\u ator:

\begin{multicols}{2}
\setlength\columnseprule{.4pt}
\begin{prooftree}
\AxiomC{$s \le s'$}
\AxiomC{$ \widetilde{\tau} \sqsubseteq \widetilde{\sigma}$}
\BinaryInfC{$ \id{d}^s \widetilde{\tau} \sqsubseteq \id{d}^{s'} \widetilde{\sigma}$}
\end{prooftree}
\columnbreak
\begin{prooftree}
\AxiomC{$\widetilde{\tau}\sqsubseteq \widetilde{\sigma}$}
\UnaryInfC{$ \nt{d}^s \widetilde{\tau} \sqsubseteq \nt{d}^{s'} \widetilde{\sigma}$}
\end{prooftree}
\end{multicols}
Regulile de tip sunt inlocuite cu cele de mai jos
\begin{multicols}{2}
\setlength\columnseprule{.4pt}
\begin{prooftree}
\AxiomC{$x : \widetilde{\sigma} \in \widetilde{\Gamma} $}
\RightLabel{\scriptsize (MT-VAR)}
\UnaryInfC{$\widetilde{\Gamma} \vdash x : \widetilde{\sigma} $}
\end{prooftree}
\begin{prooftree}
\AxiomC{$\widetilde{\Gamma}, x : \id{\tau} \vdash e : \widetilde{\sigma} $}
\RightLabel{\scriptsize (MT-ABS)}
\UnaryInfC{$\widetilde{\Gamma} \vdash \lambda x : {\tau} . e : \id{\tau} \to \widetilde{\sigma} $}
\end{prooftree}

\begin{prooftree}
\AxiomC{$\widetilde{\Gamma} \vdash e : \widetilde{\sigma}$}
\RightLabel{\scriptsize (MT-TABS)}
\UnaryInfC{$\widetilde{\Gamma} \vdash \Lambda X . e : \Pi X. \widetilde{\sigma} $}
\end{prooftree}

\begin{prooftree}
\AxiomC{$c \in \mathcal{C}(d)$}
\RightLabel{\scriptsize (MT-CONS)}
\UnaryInfC{$\widetilde{\Gamma} \vdash c : \Pi X. \id{\theta} \to \id{d}^{\hat{\iota}} X $}
\end{prooftree}

\columnbreak

\begin{prooftree}
\AxiomC{$\widetilde{\Gamma} \vdash e : \widetilde{\sigma}$}
\AxiomC{$\widetilde{\sigma} \sqsubseteq \widetilde{\tau} $}
\RightLabel{\scriptsize (MT-SUB)}
\BinaryInfC{$\widetilde{\Gamma} \vdash e : \widetilde{\tau} $}
\end{prooftree}

\begin{prooftree}
\AxiomC{$\widetilde{\Gamma} \vdash e : \widetilde{\tau} \to \widetilde{\sigma}$}
\AxiomC{$\widetilde{\Gamma} \vdash e' : \widetilde{\tau} $}
\RightLabel{\scriptsize (MT-APP)}
\BinaryInfC{$\widetilde{\Gamma} \vdash e\app e' : \widetilde{\sigma} $}
\end{prooftree}

\begin{prooftree}
\AxiomC{$\widetilde{\Gamma} \vdash e : \Pi X.\widetilde{\sigma} $}
\RightLabel{\scriptsize (MT-TAPP)}
\UnaryInfC{$\widetilde{\Gamma} \vdash e \app [{\tau}] : [ X \mapsto \id{\tau}] \widetilde{\sigma} $}
\end{prooftree}

\end{multicols}
\vspace{2pt}
\begin{prooftree}
\AxiomC{$\widetilde{\Gamma} \vdash e : \id{d^{\hat{s}}} \app \id{{\tau}} $}
\AxiomC{$\widetilde{\Gamma} \vdash e_k : [X \mapsto \id{{\tau}}, \iota \mapsto s]\widetilde{\theta}_k \to \id{\sigma} $}
\AxiomC{$\widetilde{\Gamma} \vdash c_k : \Pi X. \widetilde{\theta}_k \to d^{\hat{\iota}} X$}
\RightLabel{\scriptsize (MT-CASE)}
\insertBetweenHyps{\hskip 5pt}
\TrinaryInfC{$\widetilde{\Gamma} \vdash \text{case}_{{\sigma}}\ e \text{ of } \{c_1 \Rightarrow e_1\ |\ \dots \ |\ c_n \Rightarrow e_n\} : \id{\sigma}$}
\end{prooftree}

Regulile de mai sus sunt \^ in principal rescrierea regulilor System \fhat pentru tipuri din \textbf{ID}. Cu toate acestea, regulile {\scriptsize (MT-VAR), (MT-APP)} \c si {\scriptsize (MT-SUB)} sunt aplicabile \c si pentru tipuri din \textbf{NT}. Totu\c si, abstractizarea tipurilor se face doar peste \textbf{ID}, iar func\c tiile lambda se pot defini doar cu argument cu tip din \textbf{ID}.

Pentru a putea avea un calcul suficient de expresiv \c si pentru expresii cu tipuri din \textbf{NT}, regula {\scriptsize (MT-LETREC)} este extinsa \c si pentru acestea. \^ In cazul \^ in care regula este aplicat\u a pentru \emph{m := nt}, datorit\u a noilor reguli de subtip, ea este echivalent\u a regulii {\scriptsize (T-LETREC)} din System \frec.

\begin{prooftree}
\AxiomC{$\widetilde{\Gamma}, f : d^\iota_m \tau_m \to {\theta}_m \vdash e : d^{\hat{\iota}}_m\tau_m \to [\iota \mapsto \hat{\iota}] { \theta}_m $}
\AxiomC{$\iota \text{ pos } {\theta}_m $}
\AxiomC{$m\in \{\text{id, nt}\}$}
\RightLabel{\scriptsize (MT-LETREC)}
\TrinaryInfC{$\widetilde{\Gamma} \vdash (\text{letrec}_{d{\tau} \to {\theta}} f = e) : d^s_m {\tau}_m \to [\iota \mapsto s]{\theta}_m$}
\end{prooftree}

La nivel sintactic, se introduc dou\u a construc\c tii noi: \textbf{unit}, \textbf{bind}. Aceste construc\c tii sunt o\-mo\-lo\-a\-ge\-le lui \texttt{>>=} \c si \texttt{return} din Haskell \c si fac mul\c timea \textbf{NT} s\u a se comporte ca un monad. Pentru a explica roulul lor trebuie facuta distinc\c tia intre expresii (programe) cu tipuri din \textbf{NT} \c si valori cu tipuri din \textbf{ID} \citep{Moggi89notionsof}.
\begin{example}
Pentru o expresie de tip $\nt{\textbf{Bool}}$, valorile posibile sunt $\textbf{true}, \textbf{false} \in  \id{\textbf{Bool}}$, dar este posibil \c si ca evaluarea expresiei s\u a nu se termine.
\end{example}
\begin{itemize*}
\item Construc\c tia \textbf{unit} creeaz\u a pentru o valoare $v : \id{\tau}$, un program $(\text{unit}\app v) : \nt{\tau}$ care nu "face" nimic, dar "intoarce" expresia respectiv\u a.
\item Construc\c tia \textbf{bind} creeaza un program ob\c tinut prin pasarea c\u atre func\c tia $e$ a rezultatului rul\u arii programului $e'$. Dac\u a $e'$ nu se termin\u a, nici programul nou creeat nu se termin\u a.
\end{itemize*}
Semantica operational\u a a acestor dou\u a constructii e dat\u a de urmatoarele reguli de reducere:
\begin{gather*}
(\text{bind}\app e\app e') \to (e\app e') \qquad\text{\scriptsize (E-BIND)}\\
\text{unit}\app e \to e \qquad \text{\scriptsize (E-UNIT)}
\end{gather*}
Rolul acestor construc\c tii este de a permite prin sistemul de tipuri un calcul mai expresiv peste tipurile din \textbf{NT} fara a altera proprietatea de normalizare puternic\u a pentru tipurile din \textbf{ID}.
\begin{prooftree}
\AxiomC{$\widetilde{\Gamma} \vdash e : \id{\overline{\tau}} \to \nt{\widetilde{\sigma}} $}
\AxiomC{$\widetilde{\Gamma} \vdash e' : \widetilde{\tau} $}
\AxiomC{$ |\widetilde{\tau}| = |\id{\overline{\tau}}|$}
\RightLabel{\scriptsize (MT-BIND)}
\TrinaryInfC{$\widetilde{\Gamma} \vdash \text{bind}\app e\app e' : \nt{\widetilde{\sigma}}$}
\end{prooftree}

\begin{prooftree}
\AxiomC{$\widetilde{\Gamma} \vdash e : \widetilde{\tau} $}
\RightLabel{\scriptsize (MT-UNIT)}
\UnaryInfC{$\widetilde{\Gamma} \vdash \text{unit}\app e : \nt{\overline{\tau}}$}
\end{prooftree}


\subsection{Diagrama}
Pentru a ilustra mai bine noul sistem de tipuri vom ar\u ata cum este transformat\u a diagrama \ref{adj_list_nat} pentru a include tipuri monadice.
\done\todo{terminat diagrama}
\begin{center}
\pgfdeclarelayer{background}
\pgfsetlayers{background,main}
\begin{tikzpicture}[node distance=4cm, auto,>=latex', thick]


    \tikzstyle{type} = [shape=circle, fill=white ,draw, thin]
    \tikzstyle{constructor} = [draw, ->,color=blue!50!black ]

% Nat
    \node [type,minimum size=3.2cm,pin={[pin edge={<-,red}]90:$\textbf{\emph{Nat}}_{\text{id}}^{\infty}$}] at (0cm,0cm) (nat) {};
    \node [type,minimum size=2.0cm,label=above:$\vdots$,pin={[pin edge={<-,red}]30:$\textbf{\emph{Nat}}_{\text{id}}^{{{\hat{\iota}}}}$}] at (0cm,0cm) (natsp) {};
    \node [type,minimum size=1.6cm,pin={[pin edge={<-,red}]10:$\textbf{\emph{Nat}}_{\text{id}}^{{{{\iota}}}}$}] at (0cm,0cm) (natp) {};

    % succesor
    \path [constructor] (natp.south east)  node[above] {\scriptsize s} parabola[bend pos=0.5] bend +(0,-1cm) (natsp.south west) ;
    \path [constructor] {(nat.south)+ (0.5cm,0.08cm)}   node[above] {\scriptsize s} parabola[bend pos=0.5] bend +(0,-1cm) (nat.south);

    % zero
    \path [constructor] (0,0) circle (0) node[pin={[pin edge={<-,blue!50!black},pin distance=2cm]180:{\scriptsize z}}] (centru) {};

% Liste
    \tikzstyle{cons} = [shape=rectangle, thin, minimum width=0.03cm, minimum height=0.03cm]

    \node [cons,right of=nat,label=below:{\scriptsize cons}] (conssp) {};
    \node [cons,right of=nat,yshift=+1cm,label=below:{\scriptsize cons}] (consssp) {};

    \tikzstyle{type} = [shape=circle, fill=white, draw, thin, xshift=8cm]

    \node [type,minimum size=3.2cm,pin={[pin edge={<-,red}]90:$\textbf{\emph{List}}_{\text{id}}^{\infty}$}] at (0cm,0cm) (list) {};
    \node [type,minimum size=2.0cm,label=above:$\vdots$,pin={[pin edge={<-,red}]30:$\textbf{\emph{List}}_{\text{id}}^{{{\hat{\iota}}}}$}] at (0cm,0cm) (listsp) {};
    \node [type,minimum size=1.6cm,pin={[pin edge={<-,red}]10:$\textbf{\emph{List}}_{\text{id}}^{{{{\iota}}}}$}] at (0cm,0cm) (listp) {};

    % cons
    \path [constructor,-] (nat.east) edge             (conssp.center);
    \path [constructor,-] (nat.east) edge[bend left] (consssp.center);

    \path [constructor] (conssp.center)  parabola[bend pos=0.5] bend +(0,+0.4cm) (listsp.west);
    \path [constructor] (consssp.center) parabola[bend pos=0.5] bend +(0,-0.5cm) (list.north west);

    \path [constructor,-] (listp.west)      parabola[bend pos=0.5] bend +(0,-0.2cm) (conssp.center);
    \path [constructor,-] (list.north west)  parabola[bend pos=0.5] bend +(0,+0.5cm) (consssp.center);

    % nil
    \path [constructor] (list.center) circle (0) node[pin={[pin edge={<-,blue!50!black},pin distance=2cm]0:{\scriptsize nil}}] (centru) {};

    \begin{pgfonlayer}{background}
        \node [rounded corners,fill=red!20,fit=(list) (nat),pin={[pin edge={<-,black}]above:\textbf{\huge ID}}] {};
    \end{pgfonlayer}


    \begin{scope}[yshift=-4cm]
        \tikzstyle{nttype} = [circle, fill=white,draw,thin,minimum size=2cm,xshift=2.0cm]
        \node [nttype,pin={[pin edge={<-,red}]below:$\textbf{\emph{Nat}}_{\text{nt}}^{\infty}$}] at (0cm,0cm) (natNT) {};
        \node [nttype,pin={[pin edge={<-,red}]below:$\textbf{\emph{List}}_{\text{nt}}^{\infty}$}, right of=natNT] at (0cm,0cm) (listNT) {};
    \end{scope}

    \begin{pgfonlayer}{background}
        \node [rounded corners,fill=black!20,fit=(listNT) (natNT),pin={[pin edge={<-,black}]below:\textbf{\huge NT}}] {};
    \end{pgfonlayer}

    \tikzstyle{monad} = [draw, ->,color=brown!50!black ]
    \begin{scope}[node distance=2.2cm]
        \node [cons,right of=natNT] (consNT) {};
    \end{scope}
    \path [monad]   (nat.center) -- (natNT.center) node[pos=1] {\scriptsize unit z};
    \path [monad]   (list.center) -- (listNT.center) node[pos=0.9] {\scriptsize unit nil};
    \path [monad]   (list.south west) -- (natNT.north east) node[sloped,pos=0.5] {\scriptsize $\lambda l . (\text{unit len }l)$};
    \path [monad]   (listsp.west) node[below,xshift=-2cm,yshift=-0.4cm] {\scriptsize len} parabola[bend pos=0.5] bend +(0,-1cm) (natsp.east)  ;

    \tikzstyle{bind} = [style={draw,dashed}, ->,color=brown!50!red ]
    \path [bind,-] (listNT.center) to[bend right] (list.center);
    \path [bind,-] (list.center)   to[bend right] (list.south west);
    \path [bind] (list.south west) edge[>=stealth',bend right] (natNT.center) node[above,xshift=-3.8cm,yshift=-1cm] {\scriptsize $\text{bind}\app (\lambda l . (\text{unit len }l))\app (\text{unit nil}) $};
    \tikzstyle{bindfinal} = [draw, ->,color=brown!50!red ]
    \path [bindfinal]  (listNT.center) -- (natNT.center);
\end{tikzpicture}


\end{center}
Se observ\u a pe diagram\u a mul\c timile \textbf{ID} \c si \textbf{NT} pentru tipurile de date \textbf{List}, \textbf{Nat}. De fapt func\c tiile $\text{bind}\app (\lambda l . (\text{return len }l))\app (\text{return nil}) $ si $\lambda l . (\text{return len }l)$ au \c si ele tipuri din mul\c timea \textbf{NT}. Cu linie ro\c sie punctat\u a este desenat modul \^ in care operatorul \emph{bind} aplic\u a functia $\lambda l . (\text{return len }l)$ valorii $(\text{return nil})$.

\subsection{Propriet\u a\c ti}

\begin{lemma}
$\widetilde{\Gamma} \vdash e : \widetilde{\tau} \in \textbf{ID}$ atunci \^ in tot arborele de deduc\c tie apar doar reguli cu tipuri din \textbf{ID}.
\end{lemma}
\begin{proof}[Demonstra\c tie]
Se demonstreaz\u a prin induc\c tie structural\u a pe arborele de deduc\c tie al propozi\c tiei $\widetilde{\Gamma} \vdash e : \widetilde{\tau} \in \textbf{ID}$. \^ In afara de cazurile {\scriptsize (MT-APP)} \c si {\scriptsize (MT-SUB)}, toate tipurile care nu sunt adnotate cu \emph{id} \c si apar \^ in premisele unei reguli, apar \c si \^ in concluzie.
\begin{description*}
\item [{\scriptsize (MT-SUB)}] Datorit\u a faptului c\u a tipurile din \textbf{NT} \c si cele din \textbf{ID} sunt incomparabile \^ in raport cu rela\c tia de subtip, nu este posibil ca $\widetilde{\sigma} \in \textbf{NT}$ \c si $\widetilde{\tau} \in \textbf{ID}$
\item [{\scriptsize (MT-SUB)}] Pentru acest caz, este suficient s\u a demonstr\u am c\u a
    \begin{equation}
    \nexists e,\widetilde{\tau} \in \textbf{NT},\widetilde{\sigma} \in \textbf{ID},\widetilde{\Gamma}.\: \widetilde{\Gamma} \vdash e : \widetilde{\tau} \to \widetilde{\sigma}
    \end{equation}
    Aceast\u a propozi\c tie, din punct de vedere intuitiv, spune c\u a nu exist\u a o coresponden\c ta \^ intre programe \c si valori, ci doar \^ intre valori \c si programe, \c si anume prin \textbf{bind}.

    Demonstra\c tia este una prin induc\c tie dup\c a arborele de derivare al propozi\c tiei $\widetilde{\Gamma} \vdash e : \widetilde{\tau} \to \widetilde{\sigma}$, dar se folose\c ste o ipotez\u a de induc\c tie mai puternic\u a.
    \begin{equation}
        \nexists e,\vec{X},\widetilde{\theta},\widetilde{\tau} \in \textbf{NT},\widetilde{\sigma} \in \textbf{ID},\widetilde{\Gamma}.\: \widetilde{\Gamma} \vdash e : \Pi \vec{X}.\widetilde{\theta}\to\widetilde{\tau} \to \widetilde{\sigma}
        \qedhere
    \end{equation}
    \done\todo{terminat demonstratia}
\end{description*}
\end{proof}

\begin{corollary}
Dac\u a $\widetilde{\Gamma} \vdash e : \widetilde{\tau} \in \textbf{ID}$ atunci $\Gamma \vdash e : \overline{\tau}$, unde $\overline{\tau}$ este ob\c tinut din $\widetilde{\tau}$ prin \c stergerea adnotarilor tipurilor monadice, deci evaluarea lui $e$ se termin\u a.
\end{corollary}
\begin{proof}[Demonstra\c tie]
Cum toate regulile sunt aplicate doar pentru tipuri din \textbf{ID}, ele corespund bijectiv regulilor din System \fhat, deci $e$ este o expresie System \fhat, deci evaluarea ei se termin\u a.
\end{proof}
\begin{corollary}[Conservare]
Dac\u a $\widetilde{\Gamma} \vdash t : \widetilde{\tau} \in \textbf{ID}$ \c si $t \to t'$, atunci $\widetilde{\Gamma} \vdash t' : \widetilde{\tau}$ \\
Dac\u a $\widetilde{\Gamma} \vdash t : \widetilde{\tau} \in \textbf{NT}$ \c si $t \to t'$, atunci $|\widetilde{\Gamma}| \vdash_{\text{System \frec}} t' : |\widetilde{\tau}|$.
\end{corollary}
\begin{proposition}[Progres]
Limbajul System \fhat extins cu tipuri monadice, are proprietatea de progres.
\end{proposition}

\begin{proposition}
Limbajul System \fhat extins cu tipuri monadice, la fel ca \c si System \fhat, nu are tipuri principale.
\end{proposition}
\begin{proof}[Demonstra\c tie]
Odat\u a cu ad\u augarea tipurilor monadice, o expresie poate avea un tip din \textbf{ID} \c si unul din \textbf{NT}. Expresia urm\u atoare poate fi considerat\u a de tipul $\nt{\textbf{Nat}^{\infty}}\to\nt{\textbf{Nat}^{\infty}}$ dar \c si de tipul $\id{\textbf{Nat}^{\infty}}\to\id{\textbf{Nat}^{\infty}}$ care nu sunt compatibile \^ in raport cu rela\c tia de subtip.
\begin{equation*}
\text{letrec}_{\textbf{Nat}\to\textbf{Nat}}\: id := \lambda x : \textbf{Nat}. \text{case}_{\textbf{Nat}} e \text{ of } \{ z \Rightarrow z \:|\: s \Rightarrow \lambda px : {\textbf{Nat}}. s (id\app px) \} \qedhere
\end{equation*}
\end{proof}
\begin{proposition}
Limbajul System \fhat extins cu tipuri monadice este turing complet.
\end{proposition}
\begin{proof}[Demonstra\c tie] \label{proof_ntcompl}
Pentru a repeta demonstra\c tia teoremei \ref{compl_sysfrec} este suficient s\u a construim un o\-pe\-ra\-tor echivalent cu $case_\textbf{Nat}$ peste tipul $\nt{\textbf{Nat}}$. Aplicarea de func\c tii \c si recursivitatea textual\u a peste tipuri din \textbf{NT} sunt deja acoperite de regulile de tip {\scriptsize (MT-$\star$)}.
\begin{align*}
& caseNT := \lambda x : {\textbf{Nat}}_{\text{nt}}^\infty. \lambda fz : {\textbf{Nat}}_{\text{nt}}^\infty. \lambda fs : {\textbf{Nat}}_{\text{nt}}^\infty\to {\textbf{Nat}}_{\text{nt}}^\infty.            \\
& (\text{bind } \lambda ax : {\textbf{Nat}}_{\text{id}}^{\hat{\iota}}.                                          \\
& \qquad (\text{bind } \lambda az : \textbf{Nat}_{\text{id}}^\infty.                                            \\
& \qquad \qquad (\text{bind } \lambda as : \textbf{Nat}_{\text{id}}^\iota \to \textbf{Nat}_{\text{id}}^\infty . \\
& \qquad \qquad \qquad (\text{unit case}_{\textbf{Nat}}\ ax \text{ of } \{ z \Rightarrow az | s \Rightarrow as \})     \\
& \qquad \qquad fs)                                                                                             \\
& \qquad fz)                                                                                                   \\
& x) \qedhere
\end{align*}
\end{proof}




 % Expresivitate

% Chapter 1

\chapter{Implementare}
\label{Capitolul6}

Pentru a exemplifica modul \^ in care sistemul de tipuri poate fi folosit pentru a verifica terminarea programelor, am implementat un compilator pentru un limbaj de programare func\c tional\u a - \textbf{TBT} - bazat pe System \fhat. Urmatoarea diagram\u a este o vedere de ansamblu asupra diverselor sisteme de tipuri prezentate p\^ an\u a acum \c si a propriet\u a\c tilor lor.

\done\todo{big picture}
\begin{center}

\newenvironment{good_points}{
    \begin{list}{+}{\labelsep=2pt \leftmargin=1pt \parsep=0pt \itemsep=0pt}
}{
    \end{list}
}
\newenvironment{bad_points}{
    \begin{list}{-}{\labelsep=2pt \leftmargin=1pt \parsep=0pt \itemsep=0pt}
}{
    \end{list}
}

\begin{tikzpicture}[node distance=3cm, auto,>=latex', thick]
    \tikzstyle{typesys} =   [rectangle split,
                            rounded corners,
                            draw=black,
                            rectangle split parts=3,
                            rectangle split draw splits=true,
                            rectangle split part fill={green!20, blue!20, red!20},
                            text width=6.8em]

    \node[typesys] (lambda)
    {\scriptsize {\bf Calcul Lambda}
        \nodepart {second} {\scriptsize
            \begin{good_points}
                \item Turing complet
            \end{good_points}
        }
        \nodepart {third} {\scriptsize
        \begin{bad_points}
            \item F\u ar\u a tipuri
            \item Codificare Church
            \item F\u ar\u a normalizare
        \end{bad_points}
        }
    };

    \node[typesys,right of=lambda,yshift=-3.5cm] (sysf)
    {\scriptsize {\bf System F}
        \nodepart {second} {\scriptsize
        \begin{good_points}
            \item Tipuri
            \item Polimorfism
            \item Normalizare
        \end{good_points}
        }
        \nodepart {third} {\scriptsize
        \begin{bad_points}
            \item Codificare Church
            \item Turing incomplet
        \end{bad_points}
        }
    };

    \node[typesys,above right of=sysf, yshift=1.5cm] (sysfrec)
    {\scriptsize {\bf System \frec}
        \nodepart {second} {\scriptsize
        \begin{good_points}
            \item Turing complet
            \item Tipuri recursive
            \item Polimorfism
        \end{good_points}
        }
        \nodepart {third} {\scriptsize
        \begin{bad_points}
            \item F\u ar\u a normalizare
        \end{bad_points}
        }
    };

    \node[typesys,below right of=sysfrec,yshift=-1.5cm] (sysfhat)
    {\scriptsize {\bf System \fhat}
        \nodepart {second} {\scriptsize
        \begin{good_points}
            \item Tipuri recursive
            \item Polimorfism
            \item Normalizare
        \end{good_points}
        }
        \nodepart {third} {\scriptsize
        \begin{bad_points}
            \item Turing incomplet
            \item F\u ar\u a tipuri prin\-ci\-pa\-le (\sref{tip_princ})
        \end{bad_points}
        }
    };


    \node[typesys,below right of=sysfhat,yshift=5.5cm] (sysfco)
    {\scriptsize {\bf System \fhat + co}
        \nodepart {second} {\scriptsize
        \begin{good_points}
            \item Tipuri (co)recursive
            \item Tipuri recursive
            \item Obiecte infinite
            \item Normalizare slab\u a
        \end{good_points}
        }
        \nodepart {third} {\scriptsize
        \begin{bad_points}
            \item Turing incomplet
        \end{bad_points}
        }
    };

    \node[typesys,below right of=sysfhat, xshift=2cm, yshift=2.1cm] (sysfmon)
    {\scriptsize {\bf System \fhat + mon}
        \nodepart {second} {\scriptsize
        \begin{good_points}
            \item Tipuri recursive
            \item Polimorfism
            \item Normalizare \^ in \textbf{ID}
            \item Turing $\subseteq$ \textbf{NT}
        \end{good_points}
        }
        \nodepart {third} {\scriptsize
        }
    };

    \path [draw,->] (lambda) edge node[pos=0.5,sloped] {$\subset$} (sysf);
    \path [draw,->] (sysf) edge node[pos=0.5,sloped] {$\subset$} (sysfrec);
    \path [draw,->] (sysfrec) edge node[pos=0.5,sloped] {$\subset$} (sysfhat);
    \path [draw,->] (sysfhat) edge node[pos=0.5,sloped] {$\subset$} (sysfco);
    \path [draw,->] (sysfhat) edge node[pos=0.5,sloped] {$\subset$} (sysfmon);

\end{tikzpicture} 
\end{center}

\section{Compilatorul}
% 1 pag
Una din marile diferen\c te \^ intre System \fhat \c si un limbaj de programare func\c tional\u a este sintaxa potrivit\u a mai degraba pentru demonstra\c tii dec\^ at pentru scrierea de programe. Acest lucru se poate rezolva de c\u atre compilator prin diverse manipul\u ari sintactice (syntactic sugar).

\begin{example}
Am folosit declara\c tii globale asem\u an\u atoare celor din Haskell :\\
\texttt{\scriptsize let <X1> f1 := e1 ;} \\
\texttt{\scriptsize let <X2> f2 := e2 ;} \\
\texttt{\scriptsize body}\\
care se translateaz\u a \^ in System \fhat in $(\lambda f_1 : X_1. (\lambda f_2 : X_2. body)\app e_2) \app e_1$. Declara\c tiile de func\c tii sunt exprimate ca \texttt{\scriptsize fn <X> arg => body} \c si se translateaz\u a \^ in $\lambda arg : X.body$.
\end{example}

Un exemplu complet de program TBT este \^ in este cel de mai jos.
\begin{lstlisting}[label=impl_example,captionpos=b,caption=Exemplu de program complet]
-- Liste polimorfice
Inductive List X = nil  : List X
                   cons : X -> List X -> List X ;
-- Numere naturale
Inductive Nat    = z : Nat
                   s : Nat -> Nat  ;
-- Functie de adunare a doua numere naturale
let |<Nat->Nat->Nat>| add :=
        (letrec|<Nat->Nat->Nat>| add_ :=
            fn|<Nat>| x => fn|<Nat>| y =>
                case|<Nat>| x of {
                    z => y
                    s => fn|<Nat>| px => ((add_ px) (s y))
                }
       );
-- Programul principal
(cons |<Nat>|
    (add z (s z))
    (cons |<Nat>|
        (z)
        (nil |<Nat>|)))
\end{lstlisting}

Compilatorul transform\u a programe de genul celui de mai sus \^ in cod Java folosind mecanismul de \^ inchideri func\c tionale \citep{DBLP:books/wi/GruneBJL2002}. Fiecare func\c tie Sytem \fhat este compilat\u a intr-o clas\u a Java ce extinde clasa \texttt{Closure} ce con\c tine o referint\u a c\u atre \^ inchiderea p\u arinte \c si o mapare \^ intre numele argumentului \c si valoarea acestuia (ini\c tial maparea este vid\u a, pan\u a c\^ and func\c tia este aplicat\u a). La aplicarea unei func\c tii, se stabile\c ste leg\u atura \^ intre numele parametrului \c si \^ inchiderea care este reprezentat\u a de acesta. O variabil\u a se caut\u a pe lan\c tul de parin\c ti de inchideri func\c tionale p\^ an\u a c\^ and se gase\c ste o asociere valid\u a pentru numele ei. Un exemplu de \^ inchidere sunt constructorii care pur \c si simplu acumuleaz\u a parametri asupra c\u arora sunt aplica\c ti. \^ In cazul unei expresii \emph{case}, se stabile\c ste tipul construtorului \c si argumentele acumulate de acesta sunt furnizate func\c tiei de pe ramura corespunz\u atoare.

Strategia de evaluare folosit\u a este una \emph{call-by-value} mo\c stenit\u a din Java. Datorit\u a proprieta\c tii System \frec, \c si deci System \fhat, de normalizare puternic\u a \c si de conluen\c ta (folosind rela\c tia de reducere $\to_{\beta\iota\mu}$), termenii au o form\u a normal\u a unic\u a. Formele normale pentru aceast\u a strategie sunt incluse \^ in mul\c timea de forme normale pentru strategia lene\c s\u a. Pentru a afi\c a rezultatul programului acesta trebuie adus \^ in \emph{head normal form}, adic\u a aplica\c tiile de contructori nu mai sunt forme normale. 

Pentru parsare \c si generare de cod am folosit ANTLR \c si StringTemplate, codul fiind scris \^ in Java.

\section{Algoritmul de verificare a tipurilor}
% 1 pag

Regulile de tip prezentate \^ in sec\c tiunea \ref{reguli_sysfhat} au dezavantajul ca \^ in aplicarea lor trebuie s\u a se \emph{ghiceasca} adnot\u arile cu dimensiuni pentru unele tipuri. Solu\c tia acestei probleme \citep{DBLP:conf/tlca/BartheGP05} este a\-se\-m\u a\-n\u a\-to\-a\-re cu cea folosit\u a \^ in sistemului de tipuri Hindley-Milner \citep{ATTAPL}, \c si anume, de fiecare dat\u a c\^ and adnot\u arile trebuiesc \emph{ghicite}, acestea se \^ inlocuiesc cu \emph{variablie de deduc\c tie} cuantificate universal peste toat\u a expresia de tip (func\c tiile annot \c si annotrec din \fref{ver_alg}). Apoi, pe masur\u a ce regulile de tip sunt aplicate, se acumuleaz\u a constr\^ angeri asupra acestor variabile. Tipul unei expresii are forma $C \Rightarrow \overline{\tau}$ cu $\overline{\tau}$ un tip adnotat cu variabile de dimensiune \c si $C$ o multime de constr\^ angeri \^ intre aceste dimensiuni. Noile reguli de tip sunt prezentate \^ in pseudocod \^ in \fref{ver_alg}.


\subsection{Noile reguli de tip}

Pentru claritate, algoritmul este implementat de dou\u a func\c tii mutual recursive: \textbf{Infer} \c si \textbf{Check}. Func\c tia \textbf{Infer} deduce tipul adnotat al unei expresii, \^ in timp ce \textbf{Check} verific\u a dac\u a o expresie are un anumit tip. Ca \c si variabile globale se folosesc: o mul\c time de constr\^ angeri care se acumuleaz\u a pe masura verific\u arilor, contextul de tip \^ in care se desfasoar\u a algloritmul \c si mul\c timea de variabile care au fost deja folosite.

Constr\^ angerile sunt reprezentate de inegalit\u a\c ti \^ intre dimensiuni, constr\^ angerea \emph{false}, sau constr\^ angeri rezultate din rela\c tia de subtip. O solu\c tie pentru o mul\c time de constr\^ angeri este o substitu\c tie a variabilelor de dimensiune cu expresii de dimensiune, pentru care toate inegalit\u a\c tile pot fi demonstrate conform regulilor care dau rela\c tia de ordine \^ intre dimensiuni. De remarcat c\u a dac\u a \^ in setul de constr\^ angeri nu exist\u a \emph{false}, atunci prin \^ inlocuirea tuturor variabilelor cu $\infty$, se ob\c tine o solu\c tie. Deci dac\u a \^ in setul de constr\^ angeri nu apare constr\^ angerea \emph{false}, atunci exista cel pu\c tin un tip care se poate deduce pentru expresia verificat\u a.

\done\todo{algoritm + annot +annotrec + Check + Infer}
\begin{figure}
\begin{align*}
\textbf{Check} (V,\overline{\Gamma},e,\overline{\tau}) =
    &(V_e, C_e \cup \overline{\tau}_e \sqsubseteq \overline{\tau}) \text{ unde } \\
    &(V_e, C_e, \overline{\tau}_e) := \textbf{Infer}(\overline{\Gamma},e)\\
\textbf{Infer}(V,\overline{\Gamma},x) =
    &(V,\emptyset,\overline{\Gamma}(x))\\
\textbf{Infer} (V,\overline{\Gamma}, \lambda x : \tau.e) =
    &(V_e, C_e, \overline{\tau}_1 \to \overline{\tau}_2) \text{ unde } \\
    &(V_1,\overline{\tau}) := \textbf {annot} ( V, \overline{\tau}_1) \\
    &(V_e,C_e,\overline{\tau}_2):= \textbf{Infer}(V_1, \overline{\Gamma};x: \overline{\tau}_1,e)\\
\textbf{Infer} (V,\overline{\Gamma}, \Lambda X.e) =
    & (V_e, C_e,\Pi X.\overline{\tau}) \text{ unde }\\
    & (V_e, C_e, \overline{\tau}_e) := \textbf{Infer}(V,\overline{\Gamma},e)  \text{ daca X nu apare in } \overline{\Gamma} \\
\textbf{Infer}(V,\overline{\Gamma},e_1\app e_2)   =
    & (V_2, C_1 \cup C_2, \overline{\tau}_2) \text { unde } \\
    & (V_1, C_1, \overline{\tau}_1 \to \overline{\tau}_2) := \textbf{Infer}(V,\overline{\Gamma},e_1) \\
    & (V_2, C_2) :=  \textbf{Check}(V_1,\overline{\Gamma},e_2, \tau_1) \\
\textbf{Infer}(V,\overline{\Gamma},e_1\app [\tau])   =
    & (V_e, C_e, [X \mapsto \overline{\tau}]\overline{\tau}_e) \text { unde }\\
    & (V_1,\overline{\tau}) := \textbf{annot}(V,\tau) \\
    & (V_e, C_e, \Pi X.  \overline{\tau}_e)) := \textbf{Infer}(V_1,\overline{\Gamma},e) \\
\textbf{Infer}(V,\overline{\Gamma},c)   =
    &(V \cup \{\alpha\}, [\iota \mapsto \alpha] (\Pi X. \theta \to d^{\hat{\iota}}\app X)) \text{ unde } \alpha \notin V \\
\textbf{Infer}(V,\overline{\Gamma}, \text{case}_\sigma\ e \text{ of } \{ c_i \Rightarrow e_i\}) =
    & (V_n, s \le \hat{\alpha} \cup \bigcup C_i, \overline{\sigma} \text{ unde } \alpha \notin V \\
    & (V_{\sigma}, \overline{\sigma}) := \textbf{annot}(V\cup \{\alpha\}, \sigma) \\
    & (V_0, C_e, d^s\app \overline{\tau}) := \textbf{Infer}(V_\sigma,\overline{\Gamma},e) \\
    & (V_i, C_i) := \textbf{Check}(V_{i-1},\overline{\Gamma}, e_i, [\iota \mapsto \alpha, X\mapsto \overline{\tau}] \theta \to \sigma)\\
\textbf{Infer}(V,\overline{\Gamma}, \text{letrec}_{d^{\star}\tau\to \sigma} f := g) =
    &(V_g,C_r,d^{\alpha}\app\overline{\tau}\to\overline{\sigma})  \text{ unde } \\
    &(V_1, V^\star, d^\alpha\app\overline{\tau}\to\overline{\sigma}) := \textbf{annotrec}(V,d^\star\tau\to\sigma) \\
    &\widehat{\sigma} := [(\iota \mapsto \hat{\iota})_{\iota\in V^\star}]\overline{\sigma}\\
    &(V_g, C_g) := \textbf{Check}(V_1,(\overline{\Gamma},f:d^{\alpha}\app\overline{\tau}\to\overline{\sigma}),g,d^{\hat{\alpha}}
    \app \overline{\tau} \to \widehat{\sigma} )\\
    & C_r := \textbf{RecCheck}(\alpha, V^\star,V^\neq=V_1\setminus V^\star,C_g \cup \overline{\sigma} \sqsubseteq \widehat{\sigma})
\end{align*}
\caption{Algoritmul de verificare a tipului}
\label{ver_alg}
\end{figure}

\subsection{Verificarea pentru letrec}
% enunt cei 8 pasi
\done\todo{pasii si de ce trebuie separate var din letrec}
Pentru a putea fi aplicate, regulile {\scriptsize (T-ABS), (T-TAPP)} \c si {\scriptsize (T-CASE)} au nevoie de existen\c ta unor adnot\u ari cu dimensiuni ale tipurilor specificate \^ in sintaxa lor care s\u a satisfac\u a ipotezele. Acest lucru se traduce prin existen\c ta unei substitu\c tii pentru variabilele de deduc\c tie introduse de algortimul din \fref{ver_alg}, care s\u a satisfac\u a mul\c timea de constr\^ angeri acumulat\u a.

Pentru regula {\scriptsize (T-LETREC)} \^ insa, condi\c tia este s\u a existe o substitu\c tie $\rho$ astfel \^ inc\^ at $[\iota \mapsto s]\rho$ s\u a satisfac\u a $C$ pentru orice $s$ a c\u arei valoare nu este de forma $\hat{\jmath}^k$ cu $\jmath \in \overline{\Gamma}, \overline{\tau}$. Pentru a codifica aceast\u a condi\c tie \emph{universal\u a} prin una \emph{existential\u a}, \textbf{RecCheck} trebuie s\u a modifice mul\c timea de constr\^ angeri $C$ astfel \^ incat s\u a elimine toate constr\^ angerile ce implic\u a variabila $\alpha$ folosit\u a pentru adnotarea argumentului func\c tiei textual recursive \c si a celorlalte pozi\c tii marcate cu $\star$. Acest lucru se face prin substituirea tuturor celorlalte variabile implicate \c in astfel de constr\^ angeri cu $\infty$. Atunci cand acest lucru implica $\infty \le \alpha$, se adauga constr\^ angerea \emph{false} semanl\^ and c\u a nu poate fi stabilit un tip pentru expresia dorit\u a.

\done\todo{pe scurt trebuie ca orice substitutie sa nu combine variabilele}
\emph{Variabila de baz\u a} a unei dimensiuni $\hat{\iota}^k$ este $\iota$. Pa\c sii urma\c ti de algoritm sunt prezenta\c ti \^ in continuare:
\begin{enumerate*}
\item $S_\iota$ := mul\c timea de variabile care trebuie substituite cu dimensiuni care au variabila de baz\u a aceeasi cu a lui $\alpha$. Regula de formare este $V^\star \subseteq S_\iota,\forall \alpha_1 \in S_\iota, \hat{\alpha}^{n_2}_2 \sqsubseteq \hat{\alpha}^{n_1}_1 \Rightarrow \alpha_2 \in S_\iota $.
\item $\alpha$ trebuie sa fie cea mai mic\u a dintre dimensiunile cu aceea\c si variabil\u a de baz\u a. Adic\u a $C_1 := C \cup \bigcup_{s\in S_i}\alpha \sqsubseteq s$.
\item Dac\u a \^ intre variabile exist\u a cicluri negative de genul $\hat{\imath} \le \jmath, \jmath \le \imath $, singura variant\u a de a satisface constr\^ angerile este de a substitui toate variabilele implicate \^ in astfel de cicluri prin $\infty$, adic\u a  putem \^ inlocui toate constr\^ angerile ce formeaz\u a un ciclu negativ cu o multime de constr\^ angeri $\infty \le \imath$ obtin\^andu-se astfel $C_2$.
\item $S_{\iota\le}$ := mul\c timea de variabile care trebuie substituite prin $\infty$ sau o dimensiune cu variabil\u a de baz\u a aceea\c si cu a lui $\alpha$. Adic\u a, $S_\iota \subseteq S_{\iota\le}$ \c si $\forall \alpha_1 \in S_{\iota\le}.\: \hat{\alpha}^{n_1}_1 \sqsubseteq \hat{\alpha}^{n_2}_2 \Rightarrow \alpha_2 \in S_{\iota\le}$.
\item $S_{\neg \iota}$ := mul\c timea de variabile care trebuie substituite prin dimensiuni cu variabil\u a de baz\u a diferit\u a de cea a lui $\alpha$. Adic\u a, $V^\neq \subseteq S_{\neg \iota}$ \c si $\forall \alpha_1 \in S_{\neg \iota}.\: \hat{\alpha}^{n_1}_1 \sqsubseteq \hat{\alpha}^{n_2}_2 \Rightarrow \alpha_2 \in S_{\neg\le}$.
\item $C_3 := C_2 \cup \bigcup_{s \in S_{\neg \iota} \cap S_{\iota\le}} \infty \le s$.
\item $S_\infty$ := mul\c timea variabilelor care trebuiesc substituite cu $\infty$. Adic\u a dac\u a $\infty\le s \in C_3 \Rightarrow s \in S_\infty$ \c si $\forall \alpha_1 \in S_\infty.\: \hat{\alpha}^{n_1}_1 \sqsubseteq \hat{\alpha}^{n_2}_2 \Rightarrow \alpha_2 \in S_\infty$.
\item Dac\u a exist\u a variabile care au variabila de baz\u a $\iota$ (egal\u a cu cea a lui $\alpha$), \c si care trebuie substituite prin $\infty$, atunci \^ inseamna c\u a ipoteza regulii {\scriptsize (T-LETREC)} nu este adevarat\u a pentru $\forall \iota$ ci doar pentru o alegere particular\u a. Atunci setul de constr\^ angeri returnat este $\{false\}$, altfel este $C_3$.
\end{enumerate*}
\done\todo{continuat pasii}
\subsection{Corectitudinea algoritmului}
% algoritmul este corect => exista solutie la constrangeir daca si numai daca se termina
Faptul ca algoritmul de verificare prezentat func\c tioneaza conform regulilor din sec\c tiunea \ref{reguli_sysfhat}, este garantat de urm\u atoarea teorem\u a \citep{DBLP:conf/tlca/BartheGP05}
\done\todo{daca ex o sub a var care sa sat C, at ex o subst si pentru reg initiale}
\begin{theorem}
$\textbf{\emph{Infer}}(\emptyset,\emptyset,e) = (\_,C,\overline{\tau})$ \c si exist\u a o substitu\c tie $\rho$ a variabilelor de dimensiune prin dimensiuni care s\u a satisfac\u a pe $C$, dac\u a \c si numai dac\u a se poate deduce c\u a $\emptyset \vdash e : \rho(\overline{\tau})$ conform regulilor din sec\c tiunea \ref{reguli_sysfhat}.
\end{theorem}

\begin{comment}
\section{Demonstratia in \LaTeX{}}
% 0.5 pag
Compilatorul poate genera un fisier \LaTeX{} care sa contina demonstratia terminarii programului conform regulilor de tip.
\todo{concret cum fac?}
\end{comment}  % Implementare

% Chapter 7

\chapter{Planuri de dezvoltare ulterioar\u a}
\label{Capitolul7}

\^ In urma realiz\u arii acestui proiect apar mai multe posibile direc\c tii de dezvoltare ulterioara:
\begin{description}
\item [ Tipuri monadice ] \^ In ceea ce prive\c ste extensia System \fhat cu tipuri monadice (sectiunea \ref{tip_monad}), desi Turing completa, multe programe sunt complicat de exprimat. Un exemplu bun \^ in acest sens este chiar codificarea construc\c tiei \emph{case} (demonstra\c tia \ref{proof_ntcompl}). O variant\u a pentru aceast\u a problem\u a ar fi relaxarea regulilor de tip. \^ In formularea actual\u a, doar {\scriptsize (MT-APP),(MT-SUB)} \c si {\scriptsize (MT-LETREC)} sunt aplicabile pentru tipuri din \textbf{NT}.

    O alt\u a solu\c tie este adaugarea unor construc\c tii sintactice mai simple decat \textbf{bind} si \textbf{unit} o alternativ\u a viabil\u a ar fi construc\c tia \textbf{do} din Haskell.

\item [ Optimiz\u ari ] Un alt aspect care ar putea fi imbun\u at\u a\c tit este viteza de execu\c tie a programelor scrise in System \fhat. Cel mai mare beneficiu ar  fi adus de folosirea unor implement\u ari de nivel sc\u azut (direct \^ in Java) pentru tipuri de date standard precum \textbf{Int} sau \textbf{String} \c si pentru func\c tiile care opereaz\u a pe aceastea. Pentru a \^ incadra totu\c si \^ in cadrul teoretic studiat aceste implementari putem vedea de exemplu tipul \textbf{Nat} ca av\^ and umr\u atoarea declara\c tie
    $$ Datatype \quad \textbf{Nat} := 0 : \textbf{Nat}^{\hat{\iota}} \:|\: 1 : \textbf{Nat}^{\hat{\iota}} \:| \: \dots $$

\item [ Tipuri coinductive ] Datorit\u a propriet\u a\c tii de normalizare puternic\u a \c si a confluen\c tei, am putut folosi o strategie de evaluare \emph{call-by-value}. Pentru implementarea extensiei System \fhat cu tipuri de date coinductive, ar trebui folosit\u a o strategie de evaluare lene\c s\u a.

\item [ Mai pu\c tine adnot\u ari ]  Un inconvenient al System \fhat este c\u a, necesit\u a adnot\u ari excesive cu tipuri (chiar dac\u a adnot\u arile cu dimensiuni sunt deduse). O posibil\u a solu\c tie ar fi folosirea unui sistem de tipuri Hindley-Milner care s\u a deduc\u a tipuri care apar\c tin System \frec pentru termeni, iar apoi pe baza acestor tipuri deduse sa se foloseasca System \fhat. \^ In func\c tie de setul de constante ales, se poate pierde proprietatea de normalizare puternic\u a : cazul operatorului \emph{fix : $(X \to X) \to X$}, sau se poate diminua expresivitatea limbajului : de exemplu, \^ in HM nu pot fi exprimate numerele naturale \^ in codificare Church. A\c sadar trebuie f\u acut un compromis \^ intre expresivitate \c si cantitatea de adnot\u ari de tip.
\end{description}
 % Planuri de dezvoltare ulterioara

%% ----------------------------------------------------------------
\backmatter

\setstretch{0.9}  % Spatierea intre linii de 0.9 - o singura pagina de bibliografie
\addtocontents{toc}{\vspace{2em}}  % Spatiu in cuprins inainte de bibliografie

\label{Bibliografie}
\lhead{\emph{Bibliografie}}  % Headerul paginii este "Bibliografie"
\bibliographystyle{unsrtnat}  % Stilul "unsrtnat"
\bibliography{bibliografie}  % Fisierul cu bibliografia

% Todo-uri in faza de lucru
%\todos
\end{document}
%% ---------------------------------------------------------------- 