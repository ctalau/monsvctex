\chapter{Conclusions} % Titlul capotilului
\label{Capitolul6}


In this section I will summarize my contributions and I will mention some areas of possible improvement.

\section{Contributions summary}

In this Master thesis project, I contributed to the design and implementation of the {\tt monsvc} monitoring library used in the ATLAS trigger and data acquisition system.

I designed an expressive, easy to use and scalable configuration interface for the library which relies on the OKS \citep{jones1998oks} database for storage and uses regular expressions to specify objects with similar publishing requirements.

In order to improve the publishing latency, the publishing skew among multiple applications and the throughput of the system, I applied real-time scheduling techniques. I designed the subperiod algorithm to assign offsets for the first publication of each histogram in order to minimize jitter and contention while preserving a high throughput. I provided a theoretical analysis of the algorithm as well as simulation and testbed experiments.

Finally, I improved the synchronization between the library and the main application to support new policies for reducing the time for which the main application has is blocked.


\section{Areas of possible improvement}

In order to better support dynamic reconfiguration, we could implement a protocol to allow multiple nodes to make changes in configuration atomically and synchronously. This would allow us to reduce or eliminate the brief periods of time in which the application have different configurations.

Although we showed that the subperiod algorithm works well in practice we did not prove a lower bound for the approximation ratio. In general it cannot be better that $\ln n$, but we think the ratio is smaller if the periods are nice, i.e. have only 2, 3 and 5 as prime factors.

One can also try to extend the offset-assignment algorithm to be able to assign multiple consecutive slots to bigger histogram while maintaining a high throughput. 
